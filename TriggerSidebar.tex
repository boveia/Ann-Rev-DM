%Sidebar (50 words minimum, 200 words maximum) briefly discussing a fascinating adjacent topic; 
%insert below Literature Cited section, but indicate near which section in text the sidebar should be typeset
%Consider swapping with the ERC text below?
\begin{textbox}[!h]
\section{Selection of events at the detector level (trigger)}
%Trying to approximate 10 words per line
%Notes for improvement: 
%this is too long, needs sharpening. 
% the points i want to make are:
% higher thresholds are bad for mediator searches and also in general -> go TLA
% pileup increases MET thresholds -> get track info at the trigger level
%it needs a much clearer motivation: model X gives low mass. 
%probably that needs done in the text because space constraints, but then why using this box-thing (other than tidying things up)? 
The LHC collides protons every 25 $\mathrm{ns}$, producing 40 billion 
of events per second at nominal conditions. This amount of data cannot be 
recorded in its entirety, and not all events are interesting
for the experiments' physics programmes. %programmes or programs?
A trigger system is used to decide whether an event is selected for further analysis. 
Its first level is realized in hardware and only uses
partial detector information for fast decisions in a time of
the order of $\mathrm{\mu s}$, while its second level is software-based
and uses more refined algorithms and information to make a
decision in $\mathrm{ms}$. 

\textbf{Challenge: triggering on low-\pt objects}
Since the rates of SM physics processes decrease
with the transverse momentum of the objects involved, and processes
with a high momentum transfer have a higher chance of containing
interesting features or new particles, the trigger system records
events above a certain threshold e.g. in leading jet \pt or in event \MET. 
Only a fraction of events that do not satisfy these thresholds is recorded. 
Searches for signals with high-rate backgrounds and 
MET or jet \pt below these thresholds are 
therefore penalized unless novel
%not novel anymore?
data recording techniques, such as only recording partial event information 
needed for the search, are employed.

\textbf{Challenge: \textit{pile-up} in trigger} Simultaneous proton-proton interactions occurring within the detector
readout time cannot be completely disentangled from the hard process
of interest, especially if reconstructing the collision vertex
is not possible at the trigger level due to CPU constraints. 
This \textit{pile-up} increases the likelihood of passing 
the minimum threshold to record events, especially in the \MET triggers.
For this reason, the increase in the LHC instantaneous luminosity by virtue
of increasing the number of simultaneous
collisions leads to increases in the trigger thresholds to
keep manageable event recording rates. Reconstruction algorithms that suppress
the effects of pile-up can be employed by ATLAS and CMS directly at the trigger level,
using information on the objects and energy density within the event~\cite{CMS:2014ata,ATLAS-CONF-2014-019}. 
In future LHC runs, track information to disentangle the provenance of the 
energy deposits from the collision vertex will be available for
ATLAS and CMS from dedicated hardware systems (see e.g. Refs.~\cite{Shochet:2013gaw,1748-0221-6-12-C12065}). 
\end{textbox}
