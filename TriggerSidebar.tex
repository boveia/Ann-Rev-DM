\begin{textbox}[!h]
\section{CHALLENGES FOR TRIGGERING AT HADRON COLLIDERS}

The LHC collides protons every 25 $\mathrm{ns}$ in nominal conditions. 
The decision to record collision events for further analysis is made by the trigger system~\cite{Smith:2016vcs,Aaboud:2016leb,Khachatryan:2016bia}. 
Its first hardware-based level uses partial detector information for fast decisions.
Its second software-based level uses more refined algorithms and has access to further detector information.

\textbf{Triggering on low-\pt objects}
The trigger system records events above a certain threshold (e.g. leading jet \pt or event \MET), since energetic processes are likely to contain interesting features. 
Only a small fraction of events below these thresholds is recorded, penalizing signals with lower-energy signatures. 
However, if only final-state objects reconstructed by the trigger system are recorded, instead of full event information, the storage limitations can be overcome~\cite{Aaij:2016rxn,CMS-PAS-EXO-16-056,Aaboud:2016leb}. 

\textbf{\textit{Pile-up} in trigger} 
Pile-up can add energy uncorrelated to the hard process of interest, increasing the event rate for a given trigger threshold: trigger \MET rates grow exponentially with the number of additional interactions. 
For this reason, increases in LHC instantaneous luminosity and dataset size come at the cost of increased thresholds.
Dedicated pile-up suppression algorithms including partial tracking information are used in the trigger reconstruction~\cite{CMS:2014ata,ATLAS-CONF-2014-019}. ATLAS and CMS foresee dedicated hardware systems to obtain full tracking information in future LHC runs~\cite{Shochet:2013gaw,1748-0221-6-12-C12065}. 
 
\end{textbox}
