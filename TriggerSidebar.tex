%Sidebar (50 words minimum, 200 words maximum) briefly discussing a fascinating adjacent topic; 
%insert below Literature Cited section, but indicate near which section in text the sidebar should be typeset
%Consider swapping with the ERC text below?

%Trying to approximate 10 words per line
%Notes for improvement: 
%this is too long, needs sharpening. 
% the points i want to make are:
% higher thresholds are bad for mediator searches and also in general -> go TLA
% pileup increases MET thresholds -> get track info at the trigger level
%it needs a much clearer motivation: model X gives low mass. 
%probably that needs done in the text because space constraints, but then why using this box-thing (other than tidying things up)? 

\begin{textbox}[!h]
\section{Challenges for event triggering}

The LHC collides protons every 25 $\mathrm{ns}$ in nominal conditions. %(30 billion events per second) 
%This amount of data cannot be recorded in its entirety. %programmes or programs?
The decision to record collision events for further analysis is made by the trigger system~\cite{Smith:2016vcs,Aaboud:2016leb,Khachatryan:2016bia}. 
Its first hardware-based level uses partial detector information for fast decisions.% $\mathrm{\mu s}$.
Its second software-based level uses more refined algorithms and has access to further detector information.% to make a decision in $\mathrm{ms}$. 

\textbf{Triggering on low-\pt objects}
The trigger system records events above a certain threshold (e.g. leading jet \pt or event \MET), since energetic processes are likely to contain interesting features. 
Only a small fraction of events below these thresholds is recorded, penalizing signals with lower-energy signatures. 
%This penalizes searches with signals below these thresholds. 
However, if only final-state objects reconstructed by the trigger system are recorded, instead of full event information, the storage limitations can be overcome~\cite{Aaij:2016rxn,CMS-PAS-EXO-16-056,Aaboud:2016leb}. 
%unless non-standard data recording techniques are used.

\textbf{\textit{Pile-up} in trigger} 
Nearly-simultaneous proton-proton interactions (pile-up) add energy uncorrelated to the hard process of interest, increasing the event rate for a given trigger threshold. 
For this reason, increases in LHC instantaneous luminosity and dataset size come at the cost of increased thresholds.% due to increased numbers of simultaneous interactions. 
Tracking information can be used to determine whether energy deposits originate from the primary collision vertex. 
Partial tracking information is used in the trigger reconstruction~\cite{CMS:2014ata,ATLAS-CONF-2014-019}. ATLAS and CMS foresee dedicated hardware systems to obtain full tracking information in future LHC runs~\cite{Shochet:2013gaw,1748-0221-6-12-C12065}. 
 
\end{textbox}
