%Goal: set the stage: why collider searches are needed and interesting. 

%- we know there is DM [review] and it is one of the best motivated observations of BSM physics
%-- relic
%- there are other searches and they haven't found anything yet, or perhaps they have (hints of signals), but nothing conclusive
%- at the LHC we have a tool that has been used to make discoveries of all SM particles so far
%- this article is about what we can learn, keeping in mind that the main running version of that tool we have is the LHC


The existence of dark matter is perhaps the most persuasive experimental evidence for physics beyond the Standard Model of particle physics~\cite{Bertone:2016nfn}. 
If dark matter is indeed a particle~\cite{Steigman:1979kw}, it interacts with normal matter gravitationally. If it has other, non-gravitational interactions, these are relatively rare---dark matter is dark. It also stable, or at least has a decay lifetime that is comparable to the lifetime of the universe. Finally, there is {\it a lot} of it, about five times as much as the normal matter described by the Standard Model of particle physics. The current observed abundance of dark matter in the universe, derived from fits of the Cosmic Microwave Background measured with the Planck satellite~\cite{Ade:2015xua}, is one of the few quantitative measures of it, or indeed of any physics beyond the Standard Model (BSM), available.

The field of particle physics is increasingly keen to understand what Dark Matter (DM) is, if it is indeed a particle. 
Some experiments, termed Direct Detection (DD) experiments, are looking for evidence of galactic DM colliding with underground targets made of Standard Model matter (SM)~\cite{0954-3899-43-1-013001}. Others, termed Indirect Detection (ID) experiments, search for the products of annihilating dark matter concentrated within the gravitational potential wells of the Milky Way and extragalactic objects~\cite{Gaskins:2016cha}. None of these experiments has yet found conclusive evidence of a signal. If the only interaction between DM and SM matter is gravitational, these experiments will never see one. Yet the search for particle DM started relatively recently, and plenty of room for optimism remains.

Colliders, one of the most successful tools in particle physics, have revealed much of what we know about SM matter, and so it is natural to ask what they can contribute to the search.
This review will sketch how colliders may be used, mostly focusing on experiments located at the highest energy collider currently in operation, the Large Hadron Collider at CERN. In the absence of experimental hints on the exact nature of DM-SM interactions, it focuses on simple reactions that could be observed in the near future, on the main experimental challenges presented by searches for them, how these searches fit into the broader field. Finally, it provides some idea of the future of collider DM searches.

%Define Standard Model (SM), Dark Matter (DM). There is also an "acronym" section we could use. 
%Need to decide whether Dark Matter or dark matter.

%Most recent reviews of DD and ID (in absence of anything better?)
%\cite{DMDD_NaturePhysics}
%\cite{DMID_NaturePhysics}
%I prefer these ones also because they have arXivs but they are older , . There is also a VERY old AR: \cite{doi:10.1146/annurev.nucl.54.070103.181244}


%Throughout this review, we will consider the relic density as a rough order-of-magnitude guide for our selection of models and results, rather than as an exact constraint. Most of the models considered that satisfy the relic density only consider a single particle. The DM sector may be much more complex than a single particle with a limited number of interaactions. Nevertheless, if these simple examples dominate over others

%these simple examples may emerge in the searches at the early stages of the LHC  particles  
%Make point that we see simple things first (why simplified models is a good assumption). 

%make this into a sidebar


