Dark matter (DM) is perhaps the most persuasive experimental evidence for physics beyond the Standard Model of particle physics~\cite{Bertone:2016nfn}. 
If DM is indeed a particle~\cite{Steigman:1979kw}, it has gravitational interactions with normal matter.
It may have other, non-gravitational interactions, but these are relatively rare---dark matter is dark and has no electromagnetic charge.
It is also stable, or at least it decays with a lifetime comparable to that of the universe.
Finally, there is {\it a lot} of DM, about five times the standard matter (SM) described by the Standard Model of particle physics, with plenty of room to exceed the SM in its complexity.
The current abundance of dark matter in the universe, derived from measurements of the Cosmic Microwave Background~\cite{Ade:2015xua}, is one of the few quantitative measures of DM, or indeed of any physics beyond the Standard Model (BSM). The main consequences of cosmological observations for particle dark matter are listed in the Sidebar. 

%What we are missing

%A description of MET
%A simplified diagram of an s-channel process at colliders satisfying $Z_2$ symmetry is shown in panel (b) of Fig.~\ref{fig:monoX}.
%the following sentence: 
%The observation of a signal of visible or invisible particles at an LHC experiment that could be identified as being generated by one of the reactions described in this review cannot lead to claims that DM has been discovered. This is not a reason to discount searches for DM at the LHC, as such a signal would still be a groundbreaking discovery, regardless of its interpretation. Instead, we highlight the importance of the comparison of LHC results, where DM would be produced in the lab, with the results of complementary experiments that look for signals of DM coming from space. This comparison can only take place if the same theoretical model is used to interpret both results. This motivates the enumeration of possible models in this chapter. 

%Sidebar (50 words minimum, 200 words maximum) briefly discussing a fascinating adjacent topic; 
%insert below Literature Cited section, but indicate near which section in text the sidebar should be typeset
\begin{textbox}[!h]
\section{Particle properties of Dark Matter}

%Observations from astrophysics can inform experiments on DM particle targets, and whether newly discovered particles can be identified as DM. 
%We list here the most relevant consequences of these observation for the benchmark models used for collider DM searches described in this review. 

\textbf{Stability}
If DM is a particle, it does not seem to decay.
Conservation laws, such as R-parity in Supersymmetry (SUSY) or a $Z_2$ symmetry, can prevent the DM particle from decaying into any lighter even-parity SM particle.
Additionally, pairs of DM particles can be produced by the decay of other particles, charged under the same gauge group as the SM, or singly in the case the parent is a color triplet. 
%We also note that, while DM is stable on a cosmological scale, collider experiments are limited to the observation of particles with a lifetime that is longer than the time needed to escape the detector. 
%For this reason, we use the term "invisible particles" in collider reactions, rather than the term DM particles. 
%(i.e. DM candidate particles could still decay into other particles outside the detector and leave a signal of missing transverse momentum).

\textbf{Darkness} 
DM particles are effectively invisible to traditional collider experiments made of ordinary matter. However, the rest of the event is not. 
Invisible particles can be accompanied by one or more visible recoiling particles, leading to missing momentum in the transverse plane, whose magnitude is termed \MET. This is one of the main signatures of DM at colliders.

%\textbf{Observability of DM}
%Even though models of particle dark matter include SM couplings to satisfy cosmological observations under certain assumptions, these couplings need to be weak.
%~\footnote{We note that the only SM particle that satisfies this requirement of being sufficiently weakly interacting is the neutrino. However, neutrinos cannot make up the totality of DM as they are relativistic particles and cannot explain the galaxy structures that formed in the  universe~\cite{PlehnLecturesDM}.}.
%assumption of thermal freeze-out 
%These couplings determine the reach of collider searches, as they drive the production of new particles and their observability in the detectors. 
%so what
%if the couplings to the (constituents of) the colliding particles are enough to directly . Exceptions exist in the case of models where the SM and the DM interact only through a dark portal.
\end{textbox}


The field of particle physics is increasingly keen to understand what Dark Matter (DM) is, if it is indeed a particle. 
Some experiments, termed Direct Detection (DD) experiments, look for galactic DM colliding with underground targets made of Standard Model matter (SM)~\cite{0954-3899-43-1-013001}.
Others, termed Indirect Detection (ID) experiments, search for the products of annihilating dark matter concentrated within the gravitational potential wells of the Milky Way and elsewhere~\cite{Gaskins:2016cha}.
None of these experiments has yet found conclusive evidence of DM.
If the only interaction between DM and SM matter is gravitational, experiments will never see it.
Yet the search for particle DM started relatively recently, and plenty of room for optimism remains.

Colliders, one of the most successful tools of particle physics, have revealed much about SM matter.
This review will sketch how colliders contribute to the search for DM, focusing on the highest-energy collider currently in operation, the Large Hadron Collider at CERN.
Absent hints for the character of DM-SM interactions, it emphasizes what could be observed in the near future, the main experimental challenges presented, and how collider searches fit into the broader field.
Finally, it underlines a few areas to watch for the future LHC program.

