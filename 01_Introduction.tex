Define Standard Model (SM), Dark Matter (DM). There is also an "acronym" section we could use. 
Need to decide whether Dark Matter or dark matter.

Most recent reviews of DD and ID (in absence of anything better?)
\cite{DMDD_NaturePhysics}
\cite{DMID_NaturePhysics}
I prefer these ones also because they have arXivs but they are older \cite{Gaskins:2016cha}, \cite{0954-3899-43-1-013001}. There is also a VERY old AR: \cite{doi:10.1146/annurev.nucl.54.070103.181244}

History of DM: ~\cite{Bertone:2016nfn}

%Throughout this review, we will consider the relic density as a rough order-of-magnitude guide for our selection of models and results, rather than as an exact constraint. Most of the models considered that satisfy the relic density only consider a single particle. The DM sector may be much more complex than a single particle with a limited number of interaactions. Nevertheless, if these simple examples dominate over others.  

%these simple examples may emerge in the searches at the early stages of the LHC  particles  

We link the observations on DM to its particle properties.

Why WIMP miracle?

\subsection{Observations on DM as a guide for its particle properties}
\label{sec:DMObservations}

%Mention relic

The observations mentioned above require the dark matter particle to be stable on a cosmological timescale. This has important consequences for the prediction and observation of dark matter reactions at colliders, even though collider experiments are unable to measure particles on timescales that are longer than the time it takes them to cross the detector. 

\begin{marginnote}[]
This is because DM is stable on a cosmological scale, while LHC experiments are limited to the observation of particles with a lifetime that is longer than the time needed to escape the detector (i.e. DM candidate particles could still decay into other particles outside the detector and leave a signal of missing transverse momentum). 
\end{marginnote}

Firstly, the DM particle cannot decay. Conservation laws, such as R-parity in Supersymmetry (SUSY),
can prevent this. Another simple theoretical way to stabilize DM, as in Ref.~\cite{Batell:2010bp}, 
is the introduction of a global $Z_2$ symmetry. 
\begin{marginnote}[]
$Z_2$-parity is multiplicative and conserved.  
Under this symmetry, the parity of the DM particle is odd, while the parity of SM particles is even. 
\end{marginnote}
According to the $Z_2$ symmetry an odd-parity DM particle cannot decay into any 
lighter even-parity SM particles and it is therefore stable. 
Additionally, DM particles will be produced in pairs from the decay of other particles
that are charged under the same gauge group as the SM.
A simplified diagram of an s-channel process at colliders
satisfying $Z_2$ symmetry is shown in panel (b) of Fig.~\ref{fig:monoX}.
If the particle mediating the SM-DM interaction is a SM particle, no additional particles beyond the DM need to be invoked, leading to the simplest DM production mode at the LHC. The only theoretically viable SM portal particles within the grounding assumptions of this review are the Z and the Higgs bosons, described in Section~\ref{sec:HZPortalModels}. 

%TODO: add sidebar figure of s-channel. 
%this will become useful when we talk about s-channel mediators. maybe also make a point
%for the t-channel mediator?

Secondly, dark matter particles are invisible to traditional collider experiments. 
However, the rest of the event is not. DM particles can be accompanied by one or more
visible particles that can recoil against the DM, leading to missing
momentum in the transverse plane. 

\begin{marginnote}[]
Transverse momentum is denoted as \pt in this review, 
and the magnitude of the missing transverse momentum is 
termed \MET. 
\end{marginnote}
%CD: CMS uses this in all its plots, but i don't find it too relevant yet
%From: https://arxiv.org/pdf/1106.5048.pdf
%The following notation is used: the vector boson momentum in the transverse plane is ?qT, and the hadronic recoil, defined as the vector sum of the transverse momenta of all particles except the vector boson (or its decay products, in the case of Z candidates), is ?uT. Momentum conservation in the transverse plane requires ?qT + ?uT = 0. The recoil is the negative of the induced ?E/T.

%I don't like how this is linking up. 

%shared context: many possible new physics searches at the LHC
%problem: can't do them all
%solution: strong theoretical motivation, as well as observability
%exposition: particular case of DM

Further assumptions are needed to predict signals of DM at colliders, such
as that there is some form of interaction between DM and SM particles. 
Couplings to SM particles need to feature in the model and be sufficiently large
to produce new particles and observe their signatures in the detectors. 
%everyone thinks of WIMPs, how strong is strong, how weak is weak? quantitative question of coupling, depends on model. in the introduction: need to talk about DM properties. Weak enough that there is no visible EM signal (no light emission or absorption). Relate those properties to what the particle physics properties need to be. Have a model in mind: s-channel mediator between DM and SM, weakness of interaction comes from particle being heavy or coupling being small. DMF models have order=1 couplings. 
Models of particle dark matter include SM couplings to satisfy
cosmological observations in the thermal freeze-out case. 
Under these assumptions, and given the lack of evidence for 
DM interacting strongly with baryonic matter or its emission or absorption of light,
these couplings need to be weak. 
%A typical DM-SM coupling satisfying relic density is of the order of XXX. %CD: isn't this too model-dependent?
The only SM particle that satisfies this requirement of being
sufficiently weakly interacting is the neutrino.
However, neutrinos cannot make up the totality of DM as they 
%are not sufficiently massive 
are relativistic particles and cannot explain the galaxy structures that formed in the universe~\cite{PlehnLecturesDM}. 
%also numbers here http://www.slac.stanford.edu/econf/C040802/papers/L002.PDF
%%CITE FENG AR, BERTONE'S BOOK
%The upper bound on the neutrino content of DM is YYY. Not sure where to find this number

Unlike previous accelerators that either yielded large datasets (e.g. B-factories) or high center-of-mass energy (e.g. Tevatron), the LHC gives unprecedented access to both rare processes and high scale processes at the same time, planning to collect 3/ab by 2035 reaching the design center-of-mass energy of 14 TeV. For this reason, it is worth speculating whether the portal particles could be observed at the LHC for the first time. Models that include one or more very massive new particles beyond the SM in addition to the DM particle are also an LHC search target, and are described in Section~\ref{sec:BSMMediatorModels}. 

Portal models and models of simple BSM mediation only try to explain the presence of particle DM. They keep the SM and the DM sectors separate, and make no claim to being a solution of other shortfalls of the SM. However, the coincidence that hierarchy problem, gauge coupling unification and DM particle nature could be solved with a single theory with observable consequences at the electroweak scale, has been one of the driving reasons to develop and consider SUSY as one of the main search targets for LHC searches. These models are discussed in Section~\ref{sec:SUSYModels}.

Finally, let us return on the concept of observability of the search target mentioned above. Even general purpose particle detectors may miss certain classes of phenomena, as the initial design choices privileged searches for the Higgs boson and for particles that generally decay promptly, as predicted by models discussed so far. However, there is tension when confronting data with SM portal models, BSM mediation models and supersymmetric models compatible with the standard freeze-out scenarios. This encourages us to look for other classes of models, especially those including particles with long lifetimes, as a way to go beyond the traditional WIMP scenario. Reactions including those particles and their connections to DM are sketched in Section~\ref{sec:LLPModels}.

%%I suggest this part goes in the introduction, as it motivates enumeration of models in chapter 2 and comparisons in chapter 4. 
The observation of a signal of visible or invisible particles at an LHC experiment that could be identified as being generated by one of the reactions described in this review cannot lead to claims that DM has been discovered. This is not a reason to discount searches for DM at the LHC, as such a signal would still be a groundbreaking discovery, regardless of its interpretation. Instead, we highlight the importance of the comparison of LHC results, where DM would be produced in the lab, with the results of complementary experiments that look for signals of DM coming from space. This comparison can only take place if the same theoretical model is used to interpret both results. This motivates the enumeration of possible models in this chapter. 