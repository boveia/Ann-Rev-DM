
Finally, let us return on the concept of observability of the search target mentioned above. Even general purpose particle detectors may miss certain classes of phenomena, as the initial design choices privileged searches for the Higgs boson and for particles that generally decay promptly, as predicted by models discussed so far. However, there is tension when confronting data with SM portal models, BSM mediation models and supersymmetric models compatible with the standard freeze-out scenarios. This encourages us to look for other classes of models, especially those including particles with long lifetimes, as a way to go beyond the traditional WIMP scenario. Reactions including those particles and their connections to DM are sketched in Section~\ref{sec:LLPModels}.

%This principle can serve as a guidance, but it still does not narrow down things enough. Therefore we start by postulating the simplest possible models of DM that contain at least one particle fitting that bill, adding from none to a very limited number of particles to the SM content.
%the following sentence is stolen from above



Further assumptions are needed to predict signals of DM at colliders, such as that there is some form of interaction between DM and SM particles. 

%everyone thinks of WIMPs, how strong is strong, how weak is weak? quantitative question of coupling, depends on model. in the introduction: need to talk about DM properties. Weak enough that there is no visible EM signal (no light emission or absorption). Relate those properties to what the particle physics properties need to be. Have a model in mind: s-channel mediator between DM and SM, weakness of interaction comes from particle being heavy or coupling being small. DMF models have order=1 couplings. 
Models of particle dark matter include SM couplings to satisfy cosmological observations in the thermal freeze-out case. 
Under these assumptions, and given the lack of evidence for DM interacting strongly with baryonic matter or its emission or absorption of light, these couplings need to be weak. 
%A typical DM-SM coupling satisfying relic density is of the order of XXX. %CD: isn't this too model-dependent?
The only SM particle that satisfies this requirement of being sufficiently weakly interacting is the neutrino.
However, neutrinos cannot make up the totality of DM as they are relativistic particles and cannot explain the galaxy structures that formed in the  universe~\cite{PlehnLecturesDM}. 
%also numbers here http://www.slac.stanford.edu/econf/C040802/papers/L002.PDF
%%CITE FENG AR, BERTONE'S BOOK
%The upper bound on the neutrino content of DM is YYY. Not sure where to find this number



The stability of DM on cosmological timescales has important consequences for the prediction and observation of dark matter reactions at colliders, even though collider experiments are unable to measure particles on timescales that are longer than the time it takes them to cross the detector. 

\begin{marginnote}[]
\entry{DM stability and collider searches}{DM is stable on a cosmological scale, while LHC experiments are limited to the observation of particles with a lifetime that is longer than the time needed to escape the detector (i.e. DM candidate particles could still decay into other particles outside the detector and leave a signal of missing transverse momentum).} 
\end{marginnote}

Firstly, the DM particle cannot decay. Conservation laws, such as R-parity in Supersymmetry (SUSY), can prevent this. Another simple theoretical way to stabilize DM, as in Ref.~\cite{Batell:2010bp}, is the introduction of a global $Z_2$ symmetry. 
\begin{marginnote}[]
\entry{$Z_2$-parity}{symmetry under which the parity of the DM particle is odd, while the parity of SM particles is even.
It is multiplicative and conserved in the models considered. }
\end{marginnote}
According to the $Z_2$ symmetry an odd-parity DM particle cannot decay into any lighter even-parity SM particles and it is therefore stable. 
Additionally, DM particles will be produced in pairs from the decay of other particles that are charged under the same gauge group as the SM.
A simplified diagram of an s-channel process at colliders satisfying $Z_2$ symmetry is shown in panel (b) of Fig.~\ref{fig:monoX}.


If the particle mediating the SM-DM interaction is a SM particle, no additional particles beyond the DM need to be invoked, leading to the simplest DM production mode at the LHC. The only theoretically viable SM portal particles within the grounding assumptions of this review are the Z and the Higgs bosons, described in Sec.~\ref{sec:HZPortalModels} and the constraints on the rates of these interactions from collider searches can be found in Sec.~\ref{sec:results_ZHSearches}. 

Secondly, 

The presence of these visible particles can be used to search for SM-DM reactions that is mediated by BSM particles. Such models and the corresponding collider searches are described in Sec.~\ref{sec:BSMMediatorModels} and~\ref{sec:results_monoXSearches}. 

\begin{marginnote}[]
\entry{\pt}{transverse momentum} 
\entry{\MET}{magnitude of the missing transverse momentum} 
\end{marginnote}
%CD: CMS uses this in all its plots, but i don't find it too relevant yet
%From: https://arxiv.org/pdf/1106.5048.pdf
%The following notation is used: the vector boson momentum in the transverse plane is ?qT, and the hadronic recoil, defined as the vector sum of the transverse momenta of all particles except the vector boson (or its decay products, in the case of Z candidates), is ?uT. Momentum conservation in the transverse plane requires ?qT + ?uT = 0. The recoil is the negative of the induced ?E/T.

%I don't like how this is linking up. 

%shared context: many possible new physics searches at the LHC
%problem: can't do them all
%solution: strong theoretical motivation, as well as observability
%exposition: particular case of DM


%used already
%Unlike previous accelerators that either yielded large datasets (e.g. B-factories) or high center-of-mass energy (e.g. Tevatron), 
%the LHC gives unprecedented access to both rare processes and high scale processes at the same time, planning to collect 3/ab by 2035
%reaching the design center-of-mass energy of 14 TeV. 
%For this reason, it is worth speculating whether the portal particles could be observed at the LHC for the first time. Models that include one or more very massive new particles beyond the SM in addition to the DM particle are also an LHC search target, and are described in Section~\ref{sec:BSMMediatorModels}. 

Portal models and models of simple BSM mediation only try to explain the presence of particle DM. They keep the SM and the DM sectors separate, and make no claim to being a solution of other shortfalls of the SM. However, the coincidence that hierarchy problem, gauge coupling unification and DM particle nature could be solved with a single theory with observable consequences at the electroweak scale, has been one of the driving reasons to develop and consider SUSY as one of the main search targets for LHC searches. These models are discussed in Section~\ref{sec:SUSYModels}.


%%I suggest this part goes in the introduction, as it motivates enumeration of models in chapter 2 and comparisons in chapter 4. 

Hadron collider experiments provide a complementary tool to direct detection experiment, as the effective couplings to hadrons required for DM production in proton-proton collisions are also required for DM-nucleon scattering. %this assumes we have talked about DM-nucleon scatterin




\end{textbox}








\subsection{Observations on DM as a guide for its particle properties}
\label{sec:DMObservations}

%The guiding principles we will use to guide you apprentice reader to this journey in the world of dark matter at colliders are:








the requirements that a new discovered particle should have to be identified with DM. 


- we have little information about DM and want to make the most of it
- let's not complicate our life start simple





goals: 
- guide reader to find things (like an index, informed by the properties of DM that lead us to write all this shit)
-- describe constraints based on cosmological observations ?
-- start from simple, end in complicated 
--- maybe steal the sentence from the introduction of chapter 2?
- smoothly transition from DM to invisible particles
-- because we don't know that is stable, just collider-stable





[need transition]



%%%%

Further assumptions are needed to predict signals of DM at colliders, such as that there is some form of interaction between DM and SM particles. 
Couplings to SM particles need to feature in the model and be sufficiently large to produce new particles and observe their signatures in the detectors. 
%everyone thinks of WIMPs, how strong is strong, how weak is weak? quantitative question of coupling, depends on model. in the introduction: need to talk about DM properties. Weak enough that there is no visible EM signal (no light emission or absorption). Relate those properties to what the particle physics properties need to be. Have a model in mind: s-channel mediator between DM and SM, weakness of interaction comes from particle being heavy or coupling being small. DMF models have order=1 couplings. 
Models of particle dark matter include SM couplings to satisfy cosmological observations in the thermal freeze-out case. 
Under these assumptions, and given the lack of evidence for DM interacting strongly with baryonic matter or its emission or absorption of light, these couplings need to be weak. 
%A typical DM-SM coupling satisfying relic density is of the order of XXX. %CD: isn't this too model-dependent?
The only SM particle that satisfies this requirement of being sufficiently weakly interacting is the neutrino.
However, neutrinos cannot make up the totality of DM as they are relativistic particles and cannot explain the galaxy structures that formed in the  universe~\cite{PlehnLecturesDM}. 
%also numbers here http://www.slac.stanford.edu/econf/C040802/papers/L002.PDF
%%CITE FENG AR, BERTONE'S BOOK
%The upper bound on the neutrino content of DM is YYY. Not sure where to find this number



The stability of DM on cosmological timescales has important consequences for the prediction and observation of dark matter reactions at colliders, even though collider experiments are unable to measure particles on timescales that are longer than the time it takes them to cross the detector. 

\begin{marginnote}[]
\entry{DM stability and collider searches}{DM is stable on a cosmological scale, while LHC experiments are limited to the observation of particles with a lifetime that is longer than the time needed to escape the detector (i.e. DM candidate particles could still decay into other particles outside the detector and leave a signal of missing transverse momentum).} 
\end{marginnote}

Firstly, the DM particle cannot decay. Conservation laws, such as R-parity in Supersymmetry (SUSY), can prevent this. Another simple theoretical way to stabilize DM, as in Ref.~\cite{Batell:2010bp}, is the introduction of a global $Z_2$ symmetry. 
\begin{marginnote}[]
\entry{$Z_2$-parity}{symmetry under which the parity of the DM particle is odd, while the parity of SM particles is even.
It is multiplicative and conserved in the models considered. }
\end{marginnote}
According to the $Z_2$ symmetry an odd-parity DM particle cannot decay into any lighter even-parity SM particles and it is therefore stable. 
Additionally, DM particles will be produced in pairs from the decay of other particles that are charged under the same gauge group as the SM.
A simplified diagram of an s-channel process at colliders satisfying $Z_2$ symmetry is shown in panel (b) of Fig.~\ref{fig:monoX}.


If the particle mediating the SM-DM interaction is a SM particle, no additional particles beyond the DM need to be invoked, leading to the simplest DM production mode at the LHC. The only theoretically viable SM portal particles within the grounding assumptions of this review are the Z and the Higgs bosons, described in Sec.~\ref{sec:HZPortalModels} and the constraints on the rates of these interactions from collider searches can be found in Sec.~\ref{sec:results_ZHSearches}. 

Secondly, dark matter particles are invisible to traditional collider experiments. However, the rest of the event is not. DM particles can be accompanied by one or more visible particles that can recoil against the DM, leading to missing momentum in the transverse plane. The presence of these visible particles can be used to search for SM-DM reactions that is mediated by BSM particles. Such models and the corresponding collider searches are described in Sec.~\ref{sec:BSMMediatorModels} and~\ref{sec:results_monoXSearches}. 

\begin{marginnote}[]
\entry{\pt}{transverse momentum} 
\entry{\MET}{magnitude of the missing transverse momentum} 
\end{marginnote}
%CD: CMS uses this in all its plots, but i don't find it too relevant yet
%From: https://arxiv.org/pdf/1106.5048.pdf
%The following notation is used: the vector boson momentum in the transverse plane is ?qT, and the hadronic recoil, defined as the vector sum of the transverse momenta of all particles except the vector boson (or its decay products, in the case of Z candidates), is ?uT. Momentum conservation in the transverse plane requires ?qT + ?uT = 0. The recoil is the negative of the induced ?E/T.

%I don't like how this is linking up. 

%shared context: many possible new physics searches at the LHC
%problem: can't do them all
%solution: strong theoretical motivation, as well as observability
%exposition: particular case of DM


%used already
%Unlike previous accelerators that either yielded large datasets (e.g. B-factories) or high center-of-mass energy (e.g. Tevatron), 
%the LHC gives unprecedented access to both rare processes and high scale processes at the same time, planning to collect 3/ab by 2035
%reaching the design center-of-mass energy of 14 TeV. 
%For this reason, it is worth speculating whether the portal particles could be observed at the LHC for the first time. Models that include one or more very massive new particles beyond the SM in addition to the DM particle are also an LHC search target, and are described in Section~\ref{sec:BSMMediatorModels}. 

Portal models and models of simple BSM mediation only try to explain the presence of particle DM. They keep the SM and the DM sectors separate, and make no claim to being a solution of other shortfalls of the SM. However, the coincidence that hierarchy problem, gauge coupling unification and DM particle nature could be solved with a single theory with observable consequences at the electroweak scale, has been one of the driving reasons to develop and consider SUSY as one of the main search targets for LHC searches. These models are discussed in Section~\ref{sec:SUSYModels}.

Finally, let us return on the concept of observability of the search target mentioned above. Even general purpose particle detectors may miss certain classes of phenomena, as the initial design choices privileged searches for the Higgs boson and for particles that generally decay promptly, as predicted by models discussed so far. However, there is tension when confronting data with SM portal models, BSM mediation models and supersymmetric models compatible with the standard freeze-out scenarios. This encourages us to look for other classes of models, especially those including particles with long lifetimes, as a way to go beyond the traditional WIMP scenario. Reactions including those particles and their connections to DM are sketched in Section~\ref{sec:LLPModels}.

%%I suggest this part goes in the introduction, as it motivates enumeration of models in chapter 2 and comparisons in chapter 4. 
The observation of a signal of visible or invisible particles at an LHC experiment that could be identified as being generated by one of the reactions described in this review cannot lead to claims that DM has been discovered. This is not a reason to discount searches for DM at the LHC, as such a signal would still be a groundbreaking discovery, regardless of its interpretation. Instead, we highlight the importance of the comparison of LHC results, where DM would be produced in the lab, with the results of complementary experiments that look for signals of DM coming from space. This comparison can only take place if the same theoretical model is used to interpret both results. This motivates the enumeration of possible models in this chapter. 

Hadron collider experiments provide a complementary tool to direct detection experiment, as the effective couplings to hadrons required for DM production in proton-proton collisions are also required for DM-nucleon scattering. %this assumes we have talked about DM-nucleon scatterin




%%Precision
Precision measurements of the Higgs boson properties and the comparison with SM theory also play a role in constraining the possible contributions to new physics, as 





This stuff tests Higgs portals.



$qq \rightarrow H qq$, $qq/gg \rightarrow VH$, $gg \rightarrow Hg$ Higgs production modes. 
%CD: This section is just CMS for now. Need to add ATLAS, but also cut as we're using too much space for this:
%https://atlas.web.cern.ch/Atlas/GROUPS/PHYSICS/PAPERS/HIGG-2015-03/fig_09.png
In all cases, in addition to a requirement of sizable missing transverse momentum, auxiliary visible objects are used to select the events. 
%Loop-induced signals are important. 
%For more info on importance and calculation of loops: 1605.08039, but we run out of citations and space
The events are divided in exclusive categories targeting specific production modes. The associated boson (VH) searches target the decays of Z bosons to electrons, muons light or heavy flavour quarks, while the W bosons can decay into light-flavour jets. 
The $qq \rightarrow H qq$ production mode is dominated by Vector Boson Fusion (VBF) processes, where the Higgs boson is produced in association with two hadronic jets that have a large pseudorapidity ($\eta$)
%CD: assume eta?
separation in the detector, and a large invariant mass. This topology is used to select events and discriminate between signal and background. 
%The large QCD backgrounds are suppressed by requiring that the missing transverse momentum recoils against the jets in the event. 
%If the missing transverse momentum was in the direction of the jets, there would be a chance of it coming from mismeasured jets. %CD: hmmm written this in a rush
The jet+MET search, described in more detail in the next section, is reinterpreted for the $gg \rightarrow Hg$ mode. 


can be probed through direct observation  

only direct measurements of the Higgs width are feasible, because [reasons]

%Pre-LHC constraints on the invisible Higgs width are derived from measurements of the ZH production channel at LEP in searches for new neutral Higgs-like bosons, where only the visible decays of the Z are observed. This is a common procedure to select events in LHC invisible particles searches. 
- how do we measure it now: LHC 
-- ofuck a hadron collider: how?
---- try to measure directly -> most recent results that are upper bounds, highlight the ZH mode
---- use precision measurements
-- how to connect: Dobrescu: make a bunch of assumptions because you don't have total width, measure coupling ratios, not model-independent anymore (but ratios make it less model-dependent)

In the SM, the Z boson can decay to a neutrino-antineutrino pair, while the Higgs boson decays into a pair of Z bosons each decaying to neutrinos. Additional decays of the Z and Higgs boson to particles beyond the SM modify the properties of the vector boson, such as width and couplings. 

%%MonoH

\textbf{Invisible decays of the H boson} within the SM only contribute to less than 0.1\% of the total decay width. For this reason, an observation of even a small contribution to the Higgs width from invisible particles would signal the presence of new physics phenomena that could be linked to invisible particles if 2\minvisible particles $< m_H$~\footnote{For the case of heavier invisible particles particles, see Ref.~\cite{Djouadi:2011aa}.}. 



%CD: need a number! 
It is not feasible to directly or indirectly measure the total and partial Higgs widths at a hadron collider and then extract the invisible contribution as done for the Z at LEP, as some of the decays (e.g. gluons and lighter quarks) have too large a background to be measured, the experimental resolution even for leptonic decays is large compared to the intrinsic Higgs SM width, and the kinematics of the ZH process is not fully determined as in lepton colliders. %CD: this is ambiguous also because I am not sure I fully understand the first and third points completely - need AB's help, page 2 of Dobrescu/Lykken. 
%Lykken/Dobrescu, 1210.3342: Total theoretical SM width/mass for H125: 3.2 * 10^-5 MeV, due to small Yukawa of b quark and suppression of WW*. From rates and couplings,  can extract upper and lower limits on the exotic Higgs branching fractions, which come from the upper/lower limit on the total width. This paper ignores exp uncertainties. 
%Wagner Dark Side of the Higgs boson: omit because we don't care about non-SM Higgses
%The width can be extracted from the lineshape in the low-background channel $Z \rightarrow ZZ \rightarrow 4l$, assuming a SM width. This is what CMS has done. 
%If one does not want to assume the SM width, one can still extract the width
%above 190 GeV where the experimental resolution is better. 
%what we want to see is a larger total width with less normalization because of the invisible decays


%The upper limit on the invisible BR from Higgs decays is 25%. 
%ATLAS Abstract
%Direct searches for invisible Higgs boson decays in the vector-boson fusion and associated production of a Higgs boson with W/Z (Z ? ??, W/Z ? jj) modes are statistically combined to set an upper limit on the Higgs boson invisible branching ratio of 0.25. The use of the measured visible decay rates in a more general coupling fit improves the upper limit to 0.23, constraining a Higgs portal model of invisible particles.
%%Precision
Precision measurements of the Higgs boson properties and the comparison with SM theory also play a role in constraining the possible contributions to new physics, as decays into invisible particles would reduce the SM Higgs production and decay coupling strengths~\cite{Khachatryan:2016vau,Englert:2011yb,Aad:2015pla}. 
%For the Higgs boson, the upper limit on the branching fraction to visible and/or invisible non-SM particles only using precision measurements is 34\%
%In case we want to say what limits these
%The main limitation for the measurement of the invisible width of the Higgs at the LHC is due to QCD uncertainties the Higgs production cross-section, which limits the sensitivity of these searches to roughly 10\% of the SM value. 




%CD: do we need to answer "what if not"? No one seems to care, but one could maybe think of using monojet off-shell (tiny tiny region) and precision constraints for the off-shell region too, a la dijet. Main point for the moment: DD covers this region so we don't have to. 















% leading to a measurement of the number of light neutrino families compatible with cosmology; if the partial widths of the decays into visible particles are subtracted from the total width, the invisible width can be measured to


%Searches for invisible particles at high-energy colliders are successful, since the Z boson branching fraction to light and weakly interacting particles is sizable. 
%The cross-section of Z to neutrinos at the LHC is [blah]. While included in the SM, these processes constitute a [testbed] and a background to further search for new invisible particles. 




%Higgs and Z portal models decay more in MET than they would do in the SM only



%%monojet

%Other benchmark scenarios such as compressed SUSY scenarios, 
%maybe explain?, 
%squark pair production, 
%who ordered that
%non-thermal singly-produced invisible particles, 
%and Large Extra Dimensions (ADD) are also constrained by the ATLAS and CMS searches, in some cases providing the most stringent constraints to date. 

%%What it means for the models we talked about
%Since no significant excess is found in any of the signal regions, limits are set on the parameter space of Higgs portal models described in Sec.~\ref{sec:HZPortalModels} and simplified models described in Sec.~\ref{sec:BSMMediatorModels}, namely where the SM-invisible particles interaction is mediated by $s-$channel vector (V), axial vector (AV), scalar (S) and pseudoscalar (P) and colored scalar mediators.  

%Sidebar (50 words minimum, 200 words maximum) briefly discussing a fascinating adjacent topic;
%insert below Literature Cited section, but indicate near which section in text the sidebar should be typeset

%\subsubsection{Searches with jets}
%monojet

%%CD: I am not sure I would want to read this summary of monojet search. But maybe I'm just jaded. Anyway, if we can, we should make it more interesting / give it a slightly different spin than just a plain description. 

%and  QCD subprocesses matter - too much detail too little space https://cds.cern.ch/record/159861/files/198507018.pdf


%%Sidebar (50 words minimum, 200 words maximum) briefly discussing a fascinating adjacent topic; insert below Literature Cited section, but indicate near which section in text the sidebar should be typeset
%\begin{textbox}[!h]
%\section{Precision estimation of background for \MET+X searches}
%In order to relate the number of events in the jet+\MET
%signal regions (where $Z\rightarrow \nu\nu$ dominates) and control
%regions (where events with jets produced in association with
%$Z\rightarrow ll$, $W\rightarrow l\nu$ and $\gamma$ are used to maximise
%the statistical power of the background estimation),
%one needs to rely on a precise theory  
%prediction of the ratio of the V+jets cross-sections. 
%This is why this is important, this is 10 words. 
%This is why this is important, this is 10 words. 
%This is why this is important, this is 10 words. 
%This is why this is important, this is 10 words. 
%This is why this is important, this is 10 words. 
%This is why this is important, this is 10 words. 
%This is why this is important, this is 10 words. 
%This is why this is important, this is 10 words. 
%This is why this is important, this is 10 words. 
%This is why this is important, this is 10 words. 
%%\subsubsection{Precise background estimation}
%%\label{sub:precision}
%
%%Cite AR on this topi (but it's 2009): \cite{doi:10.1146/annurev.nucl.56.100704.122617}
%%from ooutline
%%- LHC results				
%%	citations for most recent
%%	- Differences between ATLAS and CMS:				
%%		- CMS starts with only Z, ATLAS uses Z and W, CMS uses everything including gamma				
%%		- theoretical issues with using Z and W				
%%		- Pozzorini paper: shit is complicated, W and Z are one thing but if you want to do photon it's a different story				
%%			- dependence of result on analysis cuts				
%%			- QCD correction				
%%			- EW corrections				
%\end{textbox}

%in OOutline we wanted to quote example numbers, but there is a lot of eyeballing
%>1000 GeV: EM10: 226+/-16 events predicted, 245 observed
%WIMP minvisible particles 400, mmed 1000: 0.2 (eyeballed)*200 GeV  



%CD: Maybe we put this in the reactions chapter? It is quite an important statement


%monoH come from CMS but maybe ATLAS has a better reinterpretation. 
The most stringent 95\% C.L. observed (expected) upper limits on the invisible branching
fraction from jet+\MET searches are 53\% (40\%).  TODO look for ATLAS?
%(CMS, combining jet and vector boson radiation categories). 
%V/AV come from CMS search, ATLAS is less sensitive as it's 1.55 TeV
Vector and axial vector mediators are excluded by LHC searches at values of \minvisible particles up to 700 and 400 GeV respectively with \mmed up to 1.8 TeV. This choice of model and couplings produces a relic density that is lower than the Planck measurement and it is still unconstrained by LHC searches for \minvisible particles$>$0.3 TeV at \minvisible particles$=$1.8 TeV for the vector mediator, and for 0.65$<$\minvisible particles$<$0.75 TeV at \minvisible particles=1.8 TeV for the axial vector mediator\footnote{Here and in the following, we quote observed limits at 95\% C.L. and refer to the bibliography for expected limits and 90\% C.L. limits.}. 
%CD: Not sure this is interesting for anyone? A bit complicated to project a 2D plot in words
%pseudoscalar comes from CMS
The LHC limit on the pseudoscalar mediator mass is lower due to the Yukawa-like couplings suppressing the cross-section with respect to spin-1 mediators, and it is 0.4 TeV in the CMS search for \minvisible particles up to roughly 150 GeV. 
Jet+\MET searches are not yet sensitive to scalar mediators with the chosen couplings. 
%t-channel comes from ATLAS
%CMS
%Colored scalar mediators with masses up to 1.4 TeV at values of \minvisible particles = 60 GeV are excluded.
%ATLAS
%CD TODO: check what parameters are people using?
Colored scalar mediators with masses up to 1.7 TeV at values of \minvisible particles = 10 GeV are excluded for \ginvisible particlesq=1 and \minvisible particles=100 GeV. Considering this exclusion limit, this model still provides a viable invisible particles relic density for \mmed \minvisible particles above roughly 500 GeV at \mmed=1.7 TeV\footnote{The ATLAS and CMS results do not use the same parameters, here we report the ATLAS result.}.




This observable is corrected for detector effects

Moreover, the ATLAS Collaboration has used the ratio of cross sections of events containing a jet and \MET and events containing a jet produced in association with an opposite-sign same-flavour dilepton pair from the decay of a Z/$\gamma*$ boson~\cite{Aaboud:2017buf}, corrected for detector effects. 
%in a fiducial phase space - saving space, leaving it unsaid
This is an observable sensitive to the anomalous production of events with jets and \MET, and uses many of the techniques from the the jet+\MET search described above to estimate background. The constraints derived are comparable to those of the jet+\MET search with the equivalent dataset. Unlike most other searches for new physics described in this review, detector effects are already accounted for (\textit{unfolded}) when presenting results, so that there is no need to implement a detector simulation to reinterpret this search. 


\subsection{monophoton/V}



differences:
+ Higgs to invisible (not really because monojet also)
+ the EFTs 

%i'm ok with Zll just being cited earlier, kayla will be sad but 

%not sure we need this marginnote
%\begin{marginnote}[]
%\entry{Jet substructure}{techniques employed to extract information from the radiation %pattern of a jet, by analyzing its constituents or its reconstruction history. For a review, see ~\cite{Larkoski:2017jix}. 
%}
%\end{marginnote}

% needs transition now what? backkkkkkkkkkkkkkkkkkkkkkkkkkkkkkkkkkkkkkkkkk-edkkkkiskkinyourofficenownogoodathomerightiforgot

ok let's cut this shit


%Don't want Zgamma resonances so restrict mllgamma < 1 TeV, but maybe too much info
For the \textbf{photon+\MET search} at ATLAS and CMS~\cite{Aaboud:2017dor,CMS-PAS-EXO-16-014}, jets and leptons faking photons are estimated directly from data.
The $\gamma$+jet background where the jet is mismeasured and produced \MET is suppressed by the requirement that the photon and the direction of the \MET vector do not overlap in the azimuthal plane. 
The total uncertainty is dominated by the statistical uncertainty in the control regions, ranging from 4\% to 10\%. 
%Total ATLAS uncertainty in SR1:
%post-fit
%>>> 160./2600
%0.06153846153846154
%pre-fit
%>>> 200./2400
%0.08333333333333333

%Detail detail detail
%The \textbf{photon+\MET searches} use a photon to trigger the events to be recorded for analysis, and selects events containing an isolated photon above 150 GeV and no leptons. The number of events from Z decays to neutrinos in association with a photon can be estimated in events where the lepton veto is inverted and the contribution of visible Z and W boson decays is removed from the transverse momentum balance of the event, and transferred to the signal region. 

ATLAS and CMS \textbf{W/Z+\MET searches} where the vector boson (V) decays to a quark-antiquark pair specifically select events where the decay products from the high-\pt{} boson are collimated, to better discriminate signal and background. 
QCD jets will not present any \textit{substructure}, while the decay products of vector bosons grouped into large-radius jets have a typical two-prong pattern from the hadronization of the quark-antiquark pair. 
The dominant background is still Z decays to neutrinos in association with jets, followed by W decays where the lepton is not identified, and top quark decays. 
The shapes of these backgrounds are estimated using simulation, while the normalization is determined in control regions, similarly to the jet+\MET search. 
%In the CMS search, the V+\MET backgrounds are estimated in a simultaneous fit together with the jet+\MET backgrounds. 
The main uncertainty for this search (up to 9\%~\cite{Sirunyan:2017jix} and 13\%~\cite{Aaboud:2016qgg}) %ATLAS, CMS is 9\% due to the tagging requirements
is due to the modeling of the substructure observables. 

\begin{marginnote}[]
\entry{Jet substructure}{a set of techniques employed to extract information from the radiation pattern of a jet, by analyzing its constituents or its reconstruction history. For a review, see ~\cite{Larkoski:2017jix}.}
\end{marginnote}

The \textbf{Z+\MET searches} where the Z boson decays leptonically are sufficiently general to be sensitive to simplified models of invisible particles with a Z radiation, as well as to Higgs decaying into new invisible particles and produced in association with a Z~\cite{Sirunyan:2017qfc, Aaboud:2017bja}. 
The event selection includes a constraint on the dilepton invariant mass, which limits the backgrounds to diboson, leptonically decaying top quarks, Drell-Yan production and a small amount of triboson processes. 
The estimation of the main $ZZ\rightarrow 2\nu 2l$ background (about 60\% of the total backgrounds) uses simulation, as the data sample that could be used to constrain the normalization as in the jet+\MET search is statistically-limited. The main uncertainty for this search (10\% on the background estimation) comes from the theoretical uncertainties on this background. 
%who gives a shit fuck machine learning really
%The CMS search uses a Boosted Decision Tree (BDT) applied to events with the missing transverse momentum between 100 and 130 GeV, to enhance the sensitivity to invisible Higgs decays. 

%~\footnote{Leptonic decays W bosons have also been employed in the past for this kind of searches, but due to the additional experimental challenges (e.g. the presence of an additional invisible particle, the neutrino in the W decay) and the reduced sensitivity with respect to the hadronic decays, they have not been specifically pursued as invisible particles searches for the LHC Run-2.}

%ATLAS and CMS have pursued searches for missing transverse momentum produced in association with a photon%monophoton - if we're running low on citation space, remove CMS monophoton
%\cite{Aaboud:2017dor,CMS-PAS-EXO-16-014},%less lumi, unpublished, ,
%vector boson decaying hadronically %monoV, had, 2015+2016 (CMS) and 2015 (ATLAS)
%\cite{Sirunyan:2017jix,Aaboud:2016qgg} or leptonically %monoV, lep, 2015+2016
%\cite{Aaboud:2017bja, Sirunyan:2017qfc}. 

%One of the advantages of these search signatures over the jet+\MET one
%is the lower event selection threshold, thanks to the additional handles to suppress background
%provided by either the photon ISR or the leptonic decays. 

%Searches for invisible particles produced in association with a jet are the most sensitive among the searches employing an object radiated in the initial state, due to the large signal rates from the radiation of a gluon as opposed to the radiation of a photon or a W/Z/Higgs boson. The jet+\MET final state is however also affected by the largest SM and instrumental background, and only covers signals producing a high \MET to comply with data-taking limitations at the trigger level, due to the high-rate backgrounds producing signal-like signatures. It is therefore worth considering other objects as ISR, as those searches will be subject to different backgrounds, different kinds of systematic uncertainties, lower \MET thresholds, and can provide confirmation in case of an excess in the jet+\MET final state~\cite{Birkedal:2004xn,Petriello:2008pu}. 
%Gershtein:2008bf 2nd monophoton paper, save cites
%The sensitivity hierarchy of \MET+X searches does not necessarily privilege the jet+\MET final state if there is a direct new physics coupling between a vector boson and the invisible particles, as in the case of the EFT model mentioned in Section~\ref{sub:EFT}, or if the radiated object is a new particle~\cite{Autran:2015mfa}. %mono-Z-prime
%This latter signal motivates searches in the \MET+generic resonance final state CD: keep for later. 

[transition? or we've said that before?]

In absence of any excess over the estimated background, the photon+\MET searches provide the next-to-most stringent constraints on invisible particle production, with 95\% CL cross-section limit constrained to be below 2.5 and 7~\fb depending on the \MET threshold~\cite{Aaboud:2017dor} ranging from 150 to 300 GeV. 

%we need to qualify because otherwise it looks like it's better than monojet, but then isn't it?
These results are interpreted in the context of EFT models where the invisible particles couple directly to the photon~\cite{Petrov:2013nia,Berlin:2014cfa}, excluding EFT scales between 150 and 750 GeV for \mdm=100 GeV, assuming the maximal coupling value allowed by perturbativity. %is this right? the "direct" vertex is a consequence of the EFT after all. should we give the coupling value?

axial-)vector simplified models for invisible particles,
The rates of 

Mediators 







JUNK: Run-1 LHC searches privileged EFT operators, while Run-2 searches are limited to using EFT models with a SM singlet and a boson pair, coupled to invisible particles through a contact interaction (see e.g.~\cite{Petrov:2013nia,Berlin:2014cfa}). These models provide a direct SM-invisible particles interaction and motivates searches in final states with a vector boson accompanied by \MET. 


in the region where the mediator can decay to invisible particles. %Not mentioning off-shell, otherwise it's a closed region. 
The ATLAS photon+\MET search also sets limit on the EFT model where the invisible particles couples directly to the photon, 
excluding EFT scales between 150 and 750 GeV for \minvisible particles=100 GeV, assuming the maximal coupling value allowed by perturbativity. 
%The excluded region decreases to 150-600 GeV for a coupling value of 3. 
Searches in the \MET+hadronic Z final state constrain vector and axial vector mediator
masses of \mmed $<$ 650 GeV for \minvisible particles=100 GeV\footnote{As a side note that is useful to compare results from different LHC datasets, 
Run-1 V+\MET searches used a version of the vector simplified model
that enhanced W radiation because of the constructive interference 
due to different up- and down-quark mediator couplings,
but that was not gauge invariant~\cite{Bell:2015sza,1475-7516-2016-01-051}.}. 
Searches in the \MET+leptonic Z final state provide constraints on 
\mmed $<$ 650 GeV for \minvisible particles=100 GeV for vector and axial vector mediators. 
W/Z+\MET searches are also uniquely sensitive to the radiation of a boson 
from the mediator particle in the case of colored scalar models~\cite{Bell:2012rg}, 
but the Run-2 searches do not present this interpretation. 
%CD: maybe we add a link to the 8 TeV search but it's worse than monojet 


%- however other models. eg.. eFT

%In absence of any excess over the estimated background, 

%Other things we aren't mentioning:
%- Bell's "ISR from mediator" (but no one checks that; this is the 'from the dark interaction') 
%- the monoW fake strong limits from Run-1 but they were wrong anyway
- (for before -> going there and adding)
-- UA1 monojets, can stick it in "difficulties of precision"
---- question here: the paper to cite is here:
https://cds.cern.ch/record/159861/files/198507018.pdf
and basically says "UA1 has no fucking clue about backgrounds in general" 
QUESTION HERE FLASH FLASH BLINK BLINK -> do we want to use citations to make fun of Rubbia?
read this: https://books.google.se/books?id=iguOgUlYMHoC&pg=PA22&lpg=PA22&dq=UA1+monojet&source=bl&ots=gbHPTHKsi7&sig=dI4slozaz4Wj1LBqDWliEVb0Ps0&hl=en&sa=X&ved=0ahUKEwiUl56PmY3ZAhXJh6YKHUiSB3YQ6AEINzAC#v=onepage&q=UA1%20monojet&f=false
ok so we use this to introduce the important of precise predictions
ok
not now though, i put a pointer 

-- t-channel results
maybe we should put this in (fuck that whitepaper i have no bandwidth)
yes
we should
but in: experimental challenges: how hard it is when you have this cocktail of stuff 
you mean experimental? it seems like SUSY does fine. the MC generation i'd say too hard to explain
however, though this wa first a problem in monojets, it is now a standard problem and the reason why we all use MC
THIS IS A 30 year old problem that is solved now
define this?
the ellis paperaha
i thought you were talking tchannel
ok what about tchannel

https://atlas.web.cern.ch/Atlas/GROUPS/PHYSICS/PAPERS/EXOT-2016-27/fig_08.png
"this analysis sets limits on colored scalar mediators"
tbh they are the Worlds Best Records


%%MonoHiggs

%CD: we already said that before
The newly discovered Higgs boson is of particular interest for invisible particles searches at the LHC. 
Higgs radiation is kinematically and PDF suppressed, but searches for a mono-Higgs have
other strong theoretical motivations. 
Higgs portal models are the simplest incarnation of theories where the coupling
between the dark sector and the SM is realized through a Higgs boson. Higgs couplings
to at least a new scalar are necessary for the gauge invariance of simplified models described in
the previous chapters towards more complete theories. 
%The rates of signatures including Higgs bosons are small, but 
Gauge symmetries link Higgs+\MET signatures and signatures including W, Z bosons
or jets as well as two-body mediator searches~\cite{Liew:2016oon}. 
%so results from Higgs searches are not to be taken in isolation.  


derail somewhere then do monob then do SUSY which is pretty much done by Frederik and we keep LLP for tomorrow?
question 
mark
i still want to be awake until 5' of superbowl because i am curious about your commentary
'''';;;;; ok
i need: coffee + a walk or something
walking is good
can you walk to coffee 
 could go to the lake
 then tell me if the resident heron is there

it is america: no walking to coffee unless the city
it is snowing and icy and an alert about the weather buzzed my phone
hence i did not go in today
so maybe no driving anywhere to walk
also i function for another couple of hours 
so you have say 20' to go to the lake
take pictures?
i don't know if it's iced
iced latke


not enough time
coffee alone takes 10'
how do you make coffee
it is probably a complicated procedure
involving lambics
no that is a beer
alambicchi
https://naturaecultura.files.wordpress.com/2015/01/scansione0003.jpg
your coffee apparatus
A Large Alambic Coffee Apparatus (ALACA)

so i will walk t
o throw out trash
careful about ladies in cars throwing out trash and waitressing or trashing or running over 
probably waitressing is the least dangerous
tell me if lake is icy
the fish are still under there but the heron can't reach them

do not endanger yourselfs
good
do we want to get raaackedn
letsgetletsgetletsgetletsgetracked



i like this commercial def leppard album
adrenalize
KEY CHANGE
babythatistrue
%Events are categorized according to their missing transverse momentum for CMS (50$<$\MET$<$130 GeV and \MET$>130$ GeV)
%or according to specifications optimized for different signal categories. 



and the ability to probe low \MET 



thresholds compared to other Higgs decay channel as the trigger rates are low. 
The main uncertainty for the $H \rightarrow \gamma\gamma$ searches using the 2015+2016
LHC Run-2 dataset is statistical. 


% see below, need to prune a lot

what are commonalities and differences in the two mono-H channels
how are they differently analyzed from the others
- H->bb and H->gamgam
- known higgs mass
- looking for peak on smooth background
they are completely different 
we could introduce here the fit to smooth distribution

https://open.spotify.com/track/54lmjC4P1kWvN38LgmBd1a?si=mwHA-NPoQCq3hposUjPgLQ
have you ever tried so hard that the words blah
i gotta heavy babe



ok problem: the ATLAS search is so complicated because of the overoptimization for the mandala boson that we can't really get away without the NNNNNN signal categories 
CMS has an easier way of doign thigns

%https://open.spotify.com/track/1habYP727aDxAH11D9JmuJ?si=UpFPdECNToGySuEPCeEbcg
%obvious chords i like it

- results 
-- there are so many models we don't really want to talk about them (also ATLAS and CMS use different parameters and it's awkward)
-- 2HDM??????????
-- the mandala boson is shit public statement?




%CD: I kinda want to say this but we have no space so why bother
%the mandala boson is shit, misguided attempts at combining Higgs discovery with every other excess


%where do we use the assumption of this dark interaction in the kinematics? does this mean stuff is less model independent? 
%yesthis would be a nice transition to SUSY after but then i would move the bbar+MET above this, not su.re why  oikt's below


wait though
where do we put dijets then? logically

maybe we have 1) searches with MET 2) searches without MET
or 
se
archseuss yss
s

susy -> more powerful searches when you see other consequences of the interaction -> dijets
SUSY = MET
MET = SUSY


yes that is another chapter altogether
dijet after SUSY?
not sure
people forgot monojet by then

right now outline is:

- searches for H and Z portals
- generic searches with MET
-- MET+jet
-- MET+V
(first transition about what we know about the interaction helps us, but still linked to earlier because no assumption on the interaction, only about the production)
-- MET+HF
(second transition about what we know about the interaction helps us)
-- MET+H
- generic searches without MET (dijet goes here)
- SUSY searches
- LLP searches


(aside: olegplan for whitepaper will need corrections, they are less sensitive than monoZ at not-particle-level)
--- here we don't use monojet techniques, we do shittons of fits that i don't really want to talk about

%monoH, H to invisible detailed earlier on

%SUSY

%how to do
%As discussed in Sec.~\ref{sec:SUSYModels}, predictive benchmark models lead to specific signatures where the invisible particles particle (often leading to \MET) is accompanied by a cascade of other objects. 

%If a search addresses specific final states in its signal regions, then it will benefit from an increased sensitivity with respect to a more generic \MET+X search designed to lay hold of a number of less specific models.  
%For this reason, SUSY searches apply a more stringent event selection for their signal regions than the generic searches above. %The cost of this increased sensitivity is a larger number of possible searches that must be done to have a broad coverage of the various specific possibilities in which new SUSY particles would manifest. 

%R-parity conserving SUSY searches can be broadly categorized according to the features of the signal they would produce at colliders, specifically whether:
%this way we're not going to talk about RPV? but not sure this apply, see RPV gravitino
%\begin{itemize}

%The mass reach of a selection of SUSY searches at the LHC is shown in Fig.~\ref{fig:SUSYSummary}. Figure~\ref{fig:SUSYSummary} (a) shows that the constraints on strongly produced superpartners approach 2 TeV for neutralino masses up to the TeV.  
%https://twiki.cern.ch/twiki/pub/CMSPublic/PhysicsResultsSUS/T1tttt_limits_summary_cms_Moriond17.pdf


%The mass reach of a representative selection of searches using simplified models, usually inspired by the MSSM, is shown in Fig.~\cite{fig:SUSYSummary}. The figure on the left shows that the limits on gluino production for a range of neutralino masses between [blah] 

%\item the new particles sought are strongly or weakly produced, leading to decay chains containing strongly or weakly interacting SM particles; %strong SUSY and electroweak SUSY 
%\item heavy superpartners of the top and bottom quarks are present, leading to final states with heavy flavor quarks;%Stop and 3rd gen
%\item whether the particle spectrum of LSP and NLSP is compressed, leading to either soft or long-lived objects. %LLP SUSY



%CD question: do we want to talk about the search methodology, VR, CR, SR? I would say no but it depends on what we do with the monojet part. 
%Many SUSY searches use simplified models as benchmark signals. 

%These models are usually inspired by restricted versions of the MSSM and capture the kinematics of only a subsection of the particle spectrum, 
%and can subsequently be combined to constrain full, self-consistent models~\cite{Kraml:2013mwa}. 

%The mass reach of ATLAS searches using simplified models is shown in Fig.~\ref{fig:SUSYSummary}, with a similar reach for CMS searches. 
%The lower limit on the masses of the strongly produced partners of quarks and gluons %for 100 GeV neutralino masses
%are approaching 2 TeV (see e.g.~\cite{Aaboud:2017bac}). 
%picked the strongest gluino limit from https://atlas.web.cern.ch/Atlas/GROUPS/PHYSICS/CombinedSummaryPlots/SUSY/ATLAS_SUSY_Strong_all/ATLAS_SUSY_Strong_all.png
%Heavy flavor supersymmetric partners are excluded below 1 TeV for light neutralinos (see e.g.~\cite{Sirunyan:2017wif}). %picked the latest stop 0L from CMS

%Maybe we shouldn't call them gauginos because that is more terminology? 

%Notes from the SUSY talk
%No evidence for Jet+MET SUSY (strong plain SUSY)
%Specifically target scenarios of gauge mediated: add photons to the final state and push sensitivity
%SUSY 2017 Gluino up to 2 TeV range. 
%There are parallels between the models
%Use exclusive combinations of objects
%Pick the right signal, very specific also in the mass spectrum 
%RPV: gravitino invisible particles 

%Higgsino searches nice to find projections for those. going to take a lot of luminosity.

%\subsubsection{pMSSM scans}

%tied to other 

%anyone can tie to other such products to construct global constraints for the invisible particle models of interest. 

%A FUCKLOAD OF RE-interpretation work has been done since the start of the LHC ERA. Many of these tools would make building such maps more efficient, if people would only use them. 
%They aim to encapsulate search results in some sort of abstract statistical product like the final likelihood function, which anyone can tie to other such products to construct global constraints for the invisible particle models of interest. For example, GAMBIT~\cite{Athron:2017ard} (and for CMS Mastercode~\cite{Bagnaschi:2017tru}) can produce a likelihood function for the viability of sets of parameters of a SUSY model given the available results from from both searches at colliders as well as Standard Model precision measurements and direct and indirect Dark Matter detection experiments.

%CD: I am not 100% sure I can convey this in three lines.

%global fitting code for generic Beyond the Standard Model theories, designed to allow fast and easy definition of new models, observables, likelihoods, scanners and backend physics codes.
%code that allows to fit different versions of the Minimal Supersymmetric Standard Model (MSSM) to currently existing experimental data.

%(cite battaglia etc? and ATLAS paper with relic?). Mention that these assumptions are of course very tenuous,
%as they deal with physics at potentially much higher energy scales. But invisible particles density is one
%of the only clues we have about BSM, so more, not less, of this sort of thing is interesting to do (i.e. vary the assumptions).

%Other tools to do this. GAMBIT.

%Tools exist to approach

%\subsubsection{Models with long-lived particles}
%This serves as transition for the LLP chapter

%, and why those regions have so far remained untouched. %CD: unclear how you can tell this from SUSY scans?
%CD: I would leave this discussion for the relic section later. Otherwise we can start with:
%These assumptions are of course not to be taken as more than a guiding principle... 

%dijet

%TLA and challenges, point to sidebar


%Describe dilepton and remind of why we want to include dilepton
%Searches for new particles decaying in opposite-sign, same-flavor lepton pairs can also be interpreted in terms of the simplified models of invisible particles, 
%if the mediator particle has sizable couplings to leptons. 





%An interesting feature of scalar and pseudoscalar particles decaying to $t\bar{t}$ 
%is their interference with SM $gg \rightarrow t\bar{t}$ production~\cite{Djouadi:2016ack}
%that has to be explicitly accounted for when estimating the background for these searches
%~\cite{Aaboud:2017hnm}. 

%can overcome the data taking constraints 
%at masses above roughly 500 GeV and have a sensitivity comparable to inclusive 
%dijet searches for vector and axial vector mediators.



%but the sensitivity is reduced by this requirement with
%respect to selecting the leading order dijet process. The ISR object can be either a jet and a photon,
%and it recoils against a dijet pair.
%The dijet pair can be either resolved~\cite{ATLAS:2016bvn} or
%collimated and reconstructed within a large-radius jet tagged with substructure techniques~\cite{Sirunyan:2017nvi},
%depending on the ratio between mass and transverse momentum of the resonance
%that boost the decay products. 



%sidecar, not sure how to call this but we can ask for help
%An example of such a technique is recording only \textbf{partial event information} for later analysis directly 
%from the trigger system (called Data Scouting in CMS~\cite{},
%Trigger-object Level Analysis in ATLAS~\cite{}, Turbo Stream in LHCb~\cite{Aaij:2016rxn}),
%to overcome the data storage constraints. 
%Dijet resonance searches that use this technique~\cite{CMS-PAS-EXO-16-056,ATLAS:2016xiv}
%can record the full rate of dijet events to much lower dijet invariant masses than the standard dijet
%searches, but have to overcome a number of challenges that go beyond
%a seemingly simple search. The first challenge is demonstrating that the
%performance of the physics objects reconstructed at the trigger level is sufficiently good
%to perform a physics analysis and not just take a decision on whether to keep the event.
%Secondly, the extremely large background rates (above 10$^5$ events/GeV) grant
%a sufficient statistical precision to observe signals of the order of a few thousand events,
%but also per mille-level detector and SM contamination effects. 




%his minimizes modelling
%and theoretical uncertainties. Localized excesses are sought atop %woo atop!
%the fitted background estimation. 
%Where dijet lose sensitivity: wide signals, angular 
%the fitted background estimation will be biased by the presence of signal. 
%In this case, the \textbf{scattering angle of dijet events} can be exploited as a discriminating variable, 
%since the QCD background is dominated by $t-$channel processes that privilege
%large angular separations between the two jets, as opposed to signals with an isotropic angular distribution
%in the center-of-mass frame that translates to the presence of more central jets in the
%detector~\cite{CMS-PAS-EXO-16-046,Aaboud:2017yvp}.  



%Describe dijet search
%$s-$channel scattering processes are more in the center-of-mass frame
%are not 
%in case of wide resonances, 
%and below the trigger thresholds .  



%Collider experiments are well-prepared for, with a wealth of generic searches for
%two-body resonances (see e.g.).
%In the following, we will describe two of the most general examples,
%the searches for dijet and dilepton resonances, their challenges and the implication of their results
%for models of SM-invisible particles mediation. % including a Z'-like mediator. 



%Decays into pairs of SM particles are an inevitable consequences of models where invisible particles mediators are exchanged in the $s-$channel and have a SM coupling. 
%This possibility to probe the SM-invisible particles interaction through the visible 
%decays of mediator is a unique feature of 


%As mentioned in the earlier section, 
%For this reason, it is worthwhile that collider experiments not only search for
%invisible particles, but also probe directly the interaction between Standard Model and 
%invisible particles by searching for the visible decays of the particles that mediate it, as shown 
%in Figure~\ref{fig:monoX} (c) and summarized in e.g. Refs.. 

%At the LHC, such mediators would be produced by colliding two partons,
%and thus one could studying them via their decays to jets. 
%Dijet searches are sensitive to vector and axial vector invisible particles mediators decaying 
%exclusively to jets with couplings that would satisfy relic density constraints~\cite{Chala:2015ama}.
%but also to new, unknown particles that might be created when crossing 
%the threshold of a new energy scale. 

%it is not possible to claim a discovery of invisible particles mediators at the LHC without
%corresponding excesses in invisible channels and non-collider experiments. 



%All citations from invisible particlesF, probably want to save citations for later
%They are sometimes necessary in order to construct a consistent theory, for example in minimal completions of the axial-vector model [11, 12] or in models with extended Higgs sectors [13, 14]. They often appear in anomaly-free spin-1 mediator models [15], see also Section 3.3.2 of [7]. They may also be induced through radiative corrections (e.g. through quark loops that lead to Z??Z mixing). The near-ubiquity of lepton couplings in full theories motivates including them when searching for visibly-decaying spin-1 mediators.

%Generic resonance searches sensitive to a broad range of theoretical models 
%are already the focus of LHC. 

%END JUNK
