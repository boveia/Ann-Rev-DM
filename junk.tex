
%%Precision
Precision measurements of the Higgs boson properties and the comparison with SM theory also play a role in constraining the possible contributions to new physics, as 





This stuff tests Higgs portals.



$qq \rightarrow H qq$, $qq/gg \rightarrow VH$, $gg \rightarrow Hg$ Higgs production modes. 
%CD: This section is just CMS for now. Need to add ATLAS, but also cut as we're using too much space for this:
%https://atlas.web.cern.ch/Atlas/GROUPS/PHYSICS/PAPERS/HIGG-2015-03/fig_09.png
In all cases, in addition to a requirement of sizable missing transverse momentum, auxiliary visible objects are used to select the events. 
%Loop-induced signals are important. 
%For more info on importance and calculation of loops: 1605.08039, but we run out of citations and space
The events are divided in exclusive categories targeting specific production modes. The associated boson (VH) searches target the decays of Z bosons to electrons, muons light or heavy flavour quarks, while the W bosons can decay into light-flavour jets. 
The $qq \rightarrow H qq$ production mode is dominated by Vector Boson Fusion (VBF) processes, where the Higgs boson is produced in association with two hadronic jets that have a large pseudorapidity ($\eta$)
%CD: assume eta?
separation in the detector, and a large invariant mass. This topology is used to select events and discriminate between signal and background. 
%The large QCD backgrounds are suppressed by requiring that the missing transverse momentum recoils against the jets in the event. 
%If the missing transverse momentum was in the direction of the jets, there would be a chance of it coming from mismeasured jets. %CD: hmmm written this in a rush
The jet+MET search, described in more detail in the next section, is reinterpreted for the $gg \rightarrow Hg$ mode. 


can be probed through direct observation  

only direct measurements of the Higgs width are feasible, because [reasons]

%Pre-LHC constraints on the invisible Higgs width are derived from measurements of the ZH production channel at LEP in searches for new neutral Higgs-like bosons, where only the visible decays of the Z are observed. This is a common procedure to select events in LHC invisible particles searches. 
- how do we measure it now: LHC 
-- ofuck a hadron collider: how?
---- try to measure directly -> most recent results that are upper bounds, highlight the ZH mode
---- use precision measurements
-- how to connect: Dobrescu: make a bunch of assumptions because you don't have total width, measure coupling ratios, not model-independent anymore (but ratios make it less model-dependent)

In the SM, the Z boson can decay to a neutrino-antineutrino pair, while the Higgs boson decays into a pair of Z bosons each decaying to neutrinos. Additional decays of the Z and Higgs boson to particles beyond the SM modify the properties of the vector boson, such as width and couplings. 

%%MonoH

\textbf{Invisible decays of the H boson} within the SM only contribute to less than 0.1\% of the total decay width. For this reason, an observation of even a small contribution to the Higgs width from invisible particles would signal the presence of new physics phenomena that could be linked to invisible particles if 2\minvisible particles $< m_H$~\footnote{For the case of heavier invisible particles particles, see Ref.~\cite{Djouadi:2011aa}.}. 



%CD: need a number! 
It is not feasible to directly or indirectly measure the total and partial Higgs widths at a hadron collider and then extract the invisible contribution as done for the Z at LEP, as some of the decays (e.g. gluons and lighter quarks) have too large a background to be measured, the experimental resolution even for leptonic decays is large compared to the intrinsic Higgs SM width, and the kinematics of the ZH process is not fully determined as in lepton colliders. %CD: this is ambiguous also because I am not sure I fully understand the first and third points completely - need AB's help, page 2 of Dobrescu/Lykken. 
%Lykken/Dobrescu, 1210.3342: Total theoretical SM width/mass for H125: 3.2 * 10^-5 MeV, due to small Yukawa of b quark and suppression of WW*. From rates and couplings,  can extract upper and lower limits on the exotic Higgs branching fractions, which come from the upper/lower limit on the total width. This paper ignores exp uncertainties. 
%Wagner Dark Side of the Higgs boson: omit because we don't care about non-SM Higgses
%The width can be extracted from the lineshape in the low-background channel $Z \rightarrow ZZ \rightarrow 4l$, assuming a SM width. This is what CMS has done. 
%If one does not want to assume the SM width, one can still extract the width
%above 190 GeV where the experimental resolution is better. 
%what we want to see is a larger total width with less normalization because of the invisible decays


%The upper limit on the invisible BR from Higgs decays is 25%. 
%ATLAS Abstract
%Direct searches for invisible Higgs boson decays in the vector-boson fusion and associated production of a Higgs boson with W/Z (Z ? ??, W/Z ? jj) modes are statistically combined to set an upper limit on the Higgs boson invisible branching ratio of 0.25. The use of the measured visible decay rates in a more general coupling fit improves the upper limit to 0.23, constraining a Higgs portal model of invisible particles.
%%Precision
Precision measurements of the Higgs boson properties and the comparison with SM theory also play a role in constraining the possible contributions to new physics, as decays into invisible particles would reduce the SM Higgs production and decay coupling strengths~\cite{Khachatryan:2016vau,Englert:2011yb,Aad:2015pla}. 
%For the Higgs boson, the upper limit on the branching fraction to visible and/or invisible non-SM particles only using precision measurements is 34\%
%In case we want to say what limits these
%The main limitation for the measurement of the invisible width of the Higgs at the LHC is due to QCD uncertainties the Higgs production cross-section, which limits the sensitivity of these searches to roughly 10\% of the SM value. 




%CD: do we need to answer "what if not"? No one seems to care, but one could maybe think of using monojet off-shell (tiny tiny region) and precision constraints for the off-shell region too, a la dijet. Main point for the moment: DD covers this region so we don't have to. 















% leading to a measurement of the number of light neutrino families compatible with cosmology; if the partial widths of the decays into visible particles are subtracted from the total width, the invisible width can be measured to


%Searches for invisible particles at high-energy colliders are successful, since the Z boson branching fraction to light and weakly interacting particles is sizable. 
%The cross-section of Z to neutrinos at the LHC is [blah]. While included in the SM, these processes constitute a [testbed] and a background to further search for new invisible particles. 




%Higgs and Z portal models decay more in MET than they would do in the SM only



%%monojet

%Other benchmark scenarios such as compressed SUSY scenarios, 
%maybe explain?, 
%squark pair production, 
%who ordered that
%non-thermal singly-produced invisible particles, 
%and Large Extra Dimensions (ADD) are also constrained by the ATLAS and CMS searches, in some cases providing the most stringent constraints to date. 

%%What it means for the models we talked about
%Since no significant excess is found in any of the signal regions, limits are set on the parameter space of Higgs portal models described in Sec.~\ref{sec:HZPortalModels} and simplified models described in Sec.~\ref{sec:BSMMediatorModels}, namely where the SM-invisible particles interaction is mediated by $s-$channel vector (V), axial vector (AV), scalar (S) and pseudoscalar (P) and colored scalar mediators.  

%Sidebar (50 words minimum, 200 words maximum) briefly discussing a fascinating adjacent topic;
%insert below Literature Cited section, but indicate near which section in text the sidebar should be typeset

%\subsubsection{Searches with jets}
%monojet

%%CD: I am not sure I would want to read this summary of monojet search. But maybe I'm just jaded. Anyway, if we can, we should make it more interesting / give it a slightly different spin than just a plain description. 

%and  QCD subprocesses matter - too much detail too little space https://cds.cern.ch/record/159861/files/198507018.pdf


%%Sidebar (50 words minimum, 200 words maximum) briefly discussing a fascinating adjacent topic; insert below Literature Cited section, but indicate near which section in text the sidebar should be typeset
%\begin{textbox}[!h]
%\section{Precision estimation of background for \MET+X searches}
%In order to relate the number of events in the jet+\MET
%signal regions (where $Z\rightarrow \nu\nu$ dominates) and control
%regions (where events with jets produced in association with
%$Z\rightarrow ll$, $W\rightarrow l\nu$ and $\gamma$ are used to maximise
%the statistical power of the background estimation),
%one needs to rely on a precise theory  
%prediction of the ratio of the V+jets cross-sections. 
%This is why this is important, this is 10 words. 
%This is why this is important, this is 10 words. 
%This is why this is important, this is 10 words. 
%This is why this is important, this is 10 words. 
%This is why this is important, this is 10 words. 
%This is why this is important, this is 10 words. 
%This is why this is important, this is 10 words. 
%This is why this is important, this is 10 words. 
%This is why this is important, this is 10 words. 
%This is why this is important, this is 10 words. 
%%\subsubsection{Precise background estimation}
%%\label{sub:precision}
%
%%Cite AR on this topi (but it's 2009): \cite{doi:10.1146/annurev.nucl.56.100704.122617}
%%from ooutline
%%- LHC results				
%%	citations for most recent
%%	- Differences between ATLAS and CMS:				
%%		- CMS starts with only Z, ATLAS uses Z and W, CMS uses everything including gamma				
%%		- theoretical issues with using Z and W				
%%		- Pozzorini paper: shit is complicated, W and Z are one thing but if you want to do photon it's a different story				
%%			- dependence of result on analysis cuts				
%%			- QCD correction				
%%			- EW corrections				
%\end{textbox}

%in OOutline we wanted to quote example numbers, but there is a lot of eyeballing
%>1000 GeV: EM10: 226+/-16 events predicted, 245 observed
%WIMP minvisible particles 400, mmed 1000: 0.2 (eyeballed)*200 GeV  



%CD: Maybe we put this in the reactions chapter? It is quite an important statement


%monoH come from CMS but maybe ATLAS has a better reinterpretation. 
The most stringent 95\% C.L. observed (expected) upper limits on the invisible branching
fraction from jet+\MET searches are 53\% (40\%).  TODO look for ATLAS?
%(CMS, combining jet and vector boson radiation categories). 
%V/AV come from CMS search, ATLAS is less sensitive as it's 1.55 TeV
Vector and axial vector mediators are excluded by LHC searches at values of \minvisible particles up to 700 and 400 GeV respectively with \mmed up to 1.8 TeV. This choice of model and couplings produces a relic density that is lower than the Planck measurement and it is still unconstrained by LHC searches for \minvisible particles$>$0.3 TeV at \minvisible particles$=$1.8 TeV for the vector mediator, and for 0.65$<$\minvisible particles$<$0.75 TeV at \minvisible particles=1.8 TeV for the axial vector mediator\footnote{Here and in the following, we quote observed limits at 95\% C.L. and refer to the bibliography for expected limits and 90\% C.L. limits.}. 
%CD: Not sure this is interesting for anyone? A bit complicated to project a 2D plot in words
%pseudoscalar comes from CMS
The LHC limit on the pseudoscalar mediator mass is lower due to the Yukawa-like couplings suppressing the cross-section with respect to spin-1 mediators, and it is 0.4 TeV in the CMS search for \minvisible particles up to roughly 150 GeV. 
Jet+\MET searches are not yet sensitive to scalar mediators with the chosen couplings. 
%t-channel comes from ATLAS
%CMS
%Colored scalar mediators with masses up to 1.4 TeV at values of \minvisible particles = 60 GeV are excluded.
%ATLAS
%CD TODO: check what parameters are people using?
Colored scalar mediators with masses up to 1.7 TeV at values of \minvisible particles = 10 GeV are excluded for \ginvisible particlesq=1 and \minvisible particles=100 GeV. Considering this exclusion limit, this model still provides a viable invisible particles relic density for \mmed \minvisible particles above roughly 500 GeV at \mmed=1.7 TeV\footnote{The ATLAS and CMS results do not use the same parameters, here we report the ATLAS result.}.




This observable is corrected for detector effects

Moreover, the ATLAS Collaboration has used the ratio of cross sections of events containing a jet and \MET and events containing a jet produced in association with an opposite-sign same-flavour dilepton pair from the decay of a Z/$\gamma*$ boson~\cite{Aaboud:2017buf}, corrected for detector effects. 
%in a fiducial phase space - saving space, leaving it unsaid
This is an observable sensitive to the anomalous production of events with jets and \MET, and uses many of the techniques from the the jet+\MET search described above to estimate background. The constraints derived are comparable to those of the jet+\MET search with the equivalent dataset. Unlike most other searches for new physics described in this review, detector effects are already accounted for (\textit{unfolded}) when presenting results, so that there is no need to implement a detector simulation to reinterpret this search. 


