Now that we have a handle on the reactions of DM observable at collider experiments, we turn to a description of the searches and experimental constraints for DM at colliders, privileging LHC searches as they generally set the most stringent constraints. For a detailed description of the LHC and the ATLAS, CMS and LHCb experiments, we refer to~\cite{LHC2008,ATLAS2008,CMS2008}. %CD: blurb more? I'd say no. 
The first period of LHC running (2010-2012) at 7 and 8 TeV center-of-mass energy ($\sqrt{s}$) is termed Run-1, while the second period (2015-2018) is called Run-2. 
The categorization of these searches follows loosely the description of the benchmark models. We start describing searches for DM interacting through SM bosons~\ref{sec:results_ZHSearches}, then move to generic searches for signals with missing transverse momentum~\ref{sec:results_monoXSearches}, and outline the searches for complete models with DM candidates in Section~\ref{sec:results_SUSYSearches}. Throughout this chapter and in Section~\ref{sec:experimentalChallenges} we will highlight the experimental challenges and the novel experimental techniques used to overcome them, motivated by the strong interest in dark matter searches. 
We then conclude with searches for long-lived particles within models of DM in
Section~\ref{sec:results_LLPSearches}. %CD: need to rewrite this sentence, but the idea is: if we hadn't had DM as a motivation motivation we wouldn't have done this difficult stuff. 

\subsection{Searches for DM in interactions mediated by SM-boson}
\label{sec:results_ZHSearches}

The invisible decays of the Z and Higgs boson are the main direct targets of searches for SM-boson-mediated interactions between SM and DM particles, if the DM particle is lighter than half the mass of the boson. Above this region, Direct Detection experiments are generally more sensitive than collider experiments. 
%CD: do we need to answer "what if not"? No one seems to care, but one could maybe think of using monojet off-shell (tiny tiny region) and precision constraints for the off-shell region too, a la dijet. Main point for the moment: DD covers this region so we don't have to. 
In the SM, the Z boson can decay to a pair of neutrinos, while the Higgs boson decays into a pair of Z bosons each decaying to neutrinos. Additional decays of the Z and Higgs boson to particles beyond the SM modify the properties of the vector boson, such as width and couplings. 

%MonoZ

%CD only mentioning below because it's like a monophoton
%Simple (but somehow messy) explanation in https://cds.cern.ch/record/1750933/files/CERN-THESIS-2013-330.pdf, Hugo's student
%Hugo did it, unpublished: https://www-cdf.fnal.gov/physics/ewk/2007/ZnunuWidth/
%CDF direct: 466 pm 42
\textbf{Decays of the Z boson into invisible particles} can be constrained using the invisible Z width. It can be measured directly in Z decays in association with a photon emitted as initial state radiation. Events are selected containing a single photon, missing transverse momentum and no other sizable event activity. This selection is also used for identifying events from possible DM reactions at colliders.
%LEP combined: 503 $\pm$ 16 MeV
The total Z width has been measured indirectly at LEP~\cite{ALEPH:2005ab} leading to a measurement of the number of light neutrino families compatible with cosmology; if the partial widths of the decays into visible particles are subtracted from the total width, the invisible width can be measured to 499.1 $\pm$ 1.5 MeV~\cite{Patrignani:2016xqp}. 
%The precision of the indirect measurement is better than that of the direct measurement, due to the higher statistics and the relative ease of selection and background subtraction for the visible Z decays. %CD: omitted, no space
%The main systematic uncertainty in this case comes from the theoretical uncertainties in the simulation. CD: Carena seems to think it is an uncertainty on fast simulation
New physics effects modify direct and/or indirect Z width~\cite{Carena:2003aj}.
\begin{marginnote}[]
Direct and indirect Z width measurements must agree if the decay of the Z to a pair of invisible new particles is to be the main mechanism responsible for the deviation from the SM values. 
\end{marginnote}%CD: I think this is important to mention in the same sense as the caveats on the s-channel resonances, but it can be omitted
%in this case, an analysis of the mass of the system recoiling against the photon would provide a handle to distinguish between different BSM processes. %CD: omitted because this is possibly too handwavy but how can we summarize 4 pages of Carena in a sentence?
%Carena quantitative: At present, measurements at LEP and CHARM II are capable of constraining the left-handed Z\nu\nu-coupling, 0.45 <~ g_L <~ 0.5,  while the right-handed one is only mildly bounded, |g_R| <= 0.2.
The LEP precision measurements~\footnote{Bounds on Z to invisible decays obtained from LHC searches are not yet competitive~\cite{deSimone:2014pda}.}, as well as direct detection experiments, rule out the majority of the Z-mediated DM scenarios~\cite{Arcadi:2014lta,Escudero:2016gzx}. The LEP invisible width is well below the width one would expect if vector and axial vector models of DM were realized, for all couplings satisfying the relic density with a DM mass below 25 GeV. Direct detection experiments such as Xenon1T~\cite{Aprile:2017iyp} 
%CD: take figure 2 of Escudero:2016gzx and compare with the results of Aprile:2017iyp
rule out most of the other simplified model scenarios compatible with freeze-out relic density up to multi-TeV DM masses. 
%DM mass above 6 TeV for the vector couplings, while for axial the plot is truncated. 

%%MonoH

\textbf{Invisible decays of the H boson} within the SM only contribute to less than 0.1\% of the total decay width. For this reason, an observation of even a small contribution to the Higgs width from invisible particles would signal the presence of new physics phenomena that could be linked to DM if 2\mdm $< m_H$~\footnote{For the case of heavier DM particles, see Ref.~\cite{Djouadi:2011aa}.}. 

Pre-LHC constraints on the invisible Higgs width are derived from measurements of the ZH production channel at LEP in searches for new neutral Higgs-like bosons, where only the visible decays of the Z are observed. This is a common procedure to select events in LHC DM searches. %CD: need a number! 
It is not feasible to directly or indirectly measure the total and partial Higgs widths at a hadron collider and then extract the invisible contribution as done for the Z at LEP, as some of the decays (e.g. gluons and lighter quarks) have too large a background to be measured, the experimental resolution even for leptonic decays is large compared to the intrinsic Higgs SM width, and the kinematics of the ZH process is not fully determined as in lepton colliders. %CD: this is ambiguous also because I am not sure I fully understand the first and third points completely - need AB's help, page 2 of Dobrescu/Lykken. 
%Lykken/Dobrescu, 1210.3342: Total theoretical SM width/mass for H125: 3.2 * 10^-5 MeV, due to small Yukawa of b quark and suppression of WW*. From rates and couplings,  can extract upper and lower limits on the exotic Higgs branching fractions, which come from the upper/lower limit on the total width. This paper ignores exp uncertainties. 
%Wagner Dark Side of the Higgs boson: omit because we don't care about non-SM Higgses
%The width can be extracted from the lineshape in the low-background channel $Z \rightarrow ZZ \rightarrow 4l$, assuming a SM width. This is what CMS has done. 
%If one does not want to assume the SM width, one can still extract the width
%above 190 GeV where the experimental resolution is better. 
%what we want to see is a larger total width with less normalization because of the invisible decays
Instead, searches at the LHC either attempt to directly observe the invisible decays of the Higgs boson, or compare measurements with precise theoretical calculations of SM parameters, to reveal
%unearth? there are no worms in this article so we should change this word
discrepancies signaling new physics or indirectly place constraints on new physics phenomena. 
Higgs to invisible LHC searches using Run-1 and Run-2 data~\cite{Khachatryan:2016whc,Aad:2015pla} employ and combine the $qq \rightarrow H qq$, $qq \rightarrow VH$, $gg \rightarrow HZ$, $gg \rightarrow Hg$ Higgs production modes. 
%CD: This section is just CMS for now. Need to add ATLAS, but also cut as we're using too much space for this:
%https://atlas.web.cern.ch/Atlas/GROUPS/PHYSICS/PAPERS/HIGG-2015-03/fig_09.png
In all cases, in addition to a requirement of sizable missing transverse momentum, auxiliary visible objects are used to select the events. 
%Loop-induced signals are important. 
%For more info on importance and calculation of loops: 1605.08039, but we run out of citations and space
The events are divided in exclusive categories targeting specific production modes. The associated boson (VH) searches target the decays of Z bosons to electrons, muons light or heavy flavour quarks, while the W bosons can decay into light-flavour jets. 
The $qq \rightarrow H qq$ production mode is dominated by Vector Boson Fusion (VBF) processes, where the Higgs boson is produced in association with two hadronic jets that have a large pseudorapidity ($\eta$)
%CD: assume eta?
separation in the detector, and a large invariant mass. This topology is used to select events and discriminate between signal and background. 
%The large QCD backgrounds are suppressed by requiring that the missing transverse momentum recoils against the jets in the event. 
%If the missing transverse momentum was in the direction of the jets, there would be a chance of it coming from mismeasured jets. %CD: hmmm written this in a rush
The jet+MET search, described in more detail in the next section, is reinterpreted for the $gg \rightarrow Hg$ mode. 

%The upper limit on the invisible BR from Higgs decays is 25%. 
%ATLAS Abstract
%Direct searches for invisible Higgs boson decays in the vector-boson fusion and associated production of a Higgs boson with W/Z (Z ? ??, W/Z ? jj) modes are statistically combined to set an upper limit on the Higgs boson invisible branching ratio of 0.25. The use of the measured visible decay rates in a more general coupling fit improves the upper limit to 0.23, constraining a Higgs portal model of dark matter.
%%Precision
Precision measurements of the Higgs boson properties and the comparison with SM theory also play a role in constraining the possible contributions to new physics, as decays into invisible particles would reduce the SM Higgs production and decay coupling strengths~\cite{Khachatryan:2016vau,Englert:2011yb,Aad:2015pla}. 
%For the Higgs boson, the upper limit on the branching fraction to visible and/or invisible non-SM particles only using precision measurements is 34\%
%In case we want to say what limits these
%The main limitation for the measurement of the invisible width of the Higgs at the LHC is due to QCD uncertainties the Higgs production cross-section, which limits the sensitivity of these searches to roughly 10\% of the SM value. 

The most stringent observed upper limit on the fraction of invisible decays of the Higgs boson, combining direct and precision measurements is 23\%.
%What does it mean for Higgs portal models: DD is always better
In the case of light fermion DM with scalar couplings to the Higgs, direct detection experiment rule out most of the parameter space where the model can provide the measured relic density~\cite{Escudero:2016gzx,Djouadi:2011aa}. Due to the suppression of the cross-section for DD in the pseudoscalar case, the model is still not constrained around a small region for DM masses corresponding to half the Higgs mass and above. %CD: maybe we have to say why this is the case - essentially rates are too small, see paper by Plehn. 

\subsection{Generic searches for DM with missing transverse momentum}
\label{sec:results_monoXSearches}

WIMP DM particles at colliders escape detection, and their observation requires one or more visible objects in the same event. Searches that only rely on this feature are for the most part model-agnostic, as they only need to detect an excess of missing transverse momentum left by the DM particles recoiling against SM objects, without making any extra assumption on the DM particles or on their production mechanism. The general search strategy has essentially remained unchanged as center-of-mass energy and dataset size increased, from LEP to Tevatron to the most recent LHC searches~\cite{Fox:2011fx,Beltran:2010ww,Bai:2010hh}. 

We begin this section by describing the LHC searches for missing transverse momentum in association with one or more hadronic jets. This choice allows us to illustrate many of the techniques used in invisible particle searches, and it is one of the most powerful to constrain BSM-mediated simplified models of dark matter. We then move on to outlining searches using different associated objects, and continue with searches for visible mediators that are the consequences of the DM production mechanism. Finally, we compare and discuss the sensitivity of invisible DM and visible mediator searches at the LHC. 

\subsubsection{Searches with jets}
%monojet

%%CD: I am not sure I would want to read this summary of monojet search. But maybe I'm just jaded. Anyway, if we can, we should make it more interesting / give it a slightly different spin than just a plain description. 

%%Intro and event selection

Events containing invisible particles can be identified and selected at colliders if initial state radiation (ISR) is present. For $e^+e^-$ colliders, the most frequently radiated object is a photon, while for hadron colliders gluon radiation dominates. These searches have been called "Mono-X", where X is the radiated object, although the radiation of a single object is only the leading process in a SM-DM $s-$channel interaction~\cite{Haisch:2013ata}. For this reason, the most recent LHC searches for MET with jets~\cite{Sirunyan:2017jix,Aaboud:2017phn} allow for events containing more than one jet in the final state. Since the presence of highly energetic invisible particles would manifest as an excess of events with a significant \MET,  the main observable for this search is the number of events in \MET regions, either exclusive (in bins of \MET) or inclusive (considering all events above a given \MET threshold). 

%up to 3 jets with pT>30 GeV for ATLAS
Events are selected to enter the \textit{signal region} if they contain at least one jet in the central region of the detector ($\eta<2.4$) with \pt $>$ 250 GeV (ATLAS) or \pt $>$ 100 GeV (CMS) and \MET $>$ 250 GeV (ATLAS) or \MET $>$ 200 GeV (CMS). This selection ensures that all events with these characteristics are recorded by the trigger system for further analysis, as described in more detail in Sec.~\ref{sub:trigger}. A lepton veto is used to suppress background from leptonically decaying W bosons. 
%CMS excludes taus, while ATLAS does not. Too much detail imo. 
QCD background where large \MET originates from mismeasured jets is rejected by requiring that the $\phi$ direction of the missing transverse momentum vector does not align with the direction of the four-momentum of the jets with the highest \pt (leading jets). The remaining QCD background estimated from data amounts to a maximum of 0.4\% of the total background. Events containing fake \MET due to non-collision background (e.g. cosmic rays, beam-gas interactions, calorimeter problems) is rejected with specific quality criteria discussed in Sec.~\ref{sub:MET}. 
%CMS: up to 4 leading jets 
The CMS analysis also applies specific vetoes for photons and heavy flavour jets, to reject events with photon ISR or containing top quarks. The CMS analysis also includes a signal region targeting hadronic decays of the W and Z bosons using substructure techniques, which is considered separately in the case of ATLAS and will be discussed in Sec.~\ref{subsub:monoV}. 

%%Backgrounds

The main background contributions that remain after the event selection are
invisible decays of the Z boson into neutrinos (approximately 55-70\% of the total background) 
%numbers in Livia's talk: https://indico.cern.ch/event/682235/contributions/2817876/attachments/1576792/2490208/DMWG-2017_V2.pdf and in Francesca's talk:
%https://indico.cern.ch/event/682235/contributions/2817877/attachments/1576793/2490236/171218_atlas_ungaro.pdf
and leptonic decays of the W boson where the lepton is not
reconstructed (approximately 20-35\% of the total background), in association with jets. 
%
In order to reduce theoretical and experimental uncertainties on the main V+jet backgrounds, the number of events in the signal region from each of these backgrounds are derived from data in signal-free \textit{control regions} selecting V+jet processes where the W and Z bosons decay into visible particles ($Z\rightarrow ll, W\rightarrow l\nu+jets$, where $l$ = $e, \mu$). 
%not sure we should say that the bin by bin estimates are used, here it's ambiguous
The event selection follows that of the signal region, substituting a lepton requirement to the lepton veto. The visible decay products in events selected for the control regions are subtracted from the total transverse momentum balance, providing an estimate of the contribution of these backgrounds in the signal region. CMS also uses a $\gamma$+jet control region following the same procedure to increase the statistical precision of the background estimate. 
%
The distribution of events in the main observable used for the search, the shape of the \MET distribution, is simulated and reweighted for each of the control regions using the most recent perturbative calculations for NLO QCD and QED~\ref{Lindert:2017olm}. This step is particularly important for the consistent treatment of the different processes used in the background estimation. The full information on the theoretical and experimental uncertainties and their correlations from this procedure is used in a simultaneous fit to control and signal regions, to determine the overall background estimate in each of the \MET regions considered. 
%and leads to an improvement of 40 to 50% in the search
Backgrounds from top processes in ATLAS are estimated using a dedicated control region with a requirement of a $b-$jet that is included in the fit, while CMS takes this background from simulation. Smaller diboson backgrounds are estimated from simulation. 

%%Experimental uncertainties
The systematic uncertainties on the background estimate for the jet+MET search range from 2 to 7\% (CMS) and 2 to 10\% (ATLAS), depending on the \MET region. The main uncertainties are due to the identification of leptons (CMS) and the understanding of the jet and \MET calibration (ATLAS). 

%%What it means for the models we talked about

Since no significant excess is found in any of the signal regions, limits are set on the parameter space of Higgs portal models described in Sec.~\ref{sec:HZPortalModels} and simplified models described in Sec.~\ref{sec:BSMMediatorModels}, namely where the SM-DM interaction is mediated by $s-$channel vector (V), axial vector (AV), scalar (S) and pseudoscalar (P) and colored scalar mediators. 
%CD: Maybe we put this in the reactions chapter? It is quite an important statement
Since the simulation of the entire parameter space for these models by the experiments is computationally intensive, the Dark Matter Forum had agreed on a limited set of benchmark parameters to be tested~\cite{Abercrombie:2015wmb}, privileging those that change the LHC kinematics of the search (e.g. give a harder \MET spectrum) rather than those that only change the cross-section of the process and can therefore be reinterpreted from the search results. For example, the kinematics and cross-section of the vector and axial vector mediators is very similar at the LHC, while the DD and ID cross-sections change. 
The parameter values used as benchmarks (e.g. couplings) have been selected considering the sensitivity of early Run-2 searches, precision constraints and general simplicity arguments. As described more in detail in Section~\ref{sub:comparisonVisibleInvisible}, the plane \mdm, \mmed is scanned fixing the couplings to \gq=0.25 and \gdm=1.0 for vector and axial vector mediated models, and \gq=\gdm=1.0 for scalar and pseudoscalar models. 

%monoH come from CMS but maybe ATLAS has a better reinterpretation. 
The most stringent 95\% C.L. observed (expected) upper limits on the invisible branching fraction from jet+\MET searches are 53\% (40\%).  TODO look for ATLAS?
%(CMS, combining jet and vector boson radiation categories). 
%V/AV come from CMS search, ATLAS is less sensitive as it's 1.55 TeV
Vector and axial vector mediators are excluded by LHC searches at values of \mdm up to 700 and 400 GeV respectively with \mmed up to 1.8 TeV. This choice of model and couplings produces a relic density that is lower than the Planck measurement and it is still unconstrained by LHC searches for \mdm$>$0.3 TeV at \mdm$=$1.8 TeV for the vector mediator, and for 0.65$<$\mdm$<$0.75 TeV at \mdm=1.8 TeV for the axial vector mediator. 
%CD: Not sure this is interesting for anyone? A bit complicated to project a 2D plot in words
%pseudoscalar comes from CMS
The LHC limit on the pseudoscalar mediator mass is lower due to the Yukawa-like couplings suppressing the cross-section with respect to spin-1 mediators, and it is 0.4 TeV in the CMS search for \mdm up to roughly 150 GeV. 
Jet+\MET searches are not yet sensitive to scalar mediators with the chosen couplings. 
%t-channel comes from ATLAS
%CMS
%Colored scalar mediators with masses up to 1.4 TeV at values of \mdm = 60 GeV are excluded.
%ATLAS
%CD TODO: what parameters are people using?
Colored scalar mediators with masses up to 1.7 TeV at values of \mdm = 10 GeV are excluded. Considering this exclusion limit, this model still provides a viable DM relic density for \mmed \mdm above roughly 500 GeV at \mmed=1.7 TeV. 

Other benchmark scenarios such as compressed SUSY scenarios, squark pair production, 
%maybe explain?, 
non-thermal singly-produced DM, 
%who ordered that
and Large Extra Dimensions (ADD) are also constrained by the ATLAS and CMS searches, in some cases providing the most stringent constraints to date. 

%%Reinterpretation - this is the only thing that I think fits well here
Given the importance of this search to constrain a wide variety of reactions for invisible particles, various approaches have been taken to allow model-builders and phenomenologists to more easily reinterpret ints results. All experimental data published from the two collaborations is provided on the HEPData platform~\cite{Maguire:2017ypu}. 

Mention R ratio as example of unfolded search, simplified likelihood. 

%s-channel and t-channel 

\subsubsection{Searches with photons and vector bosons}
\label{subsub:monoV}
%monophoton, monoV

%Cite MonoZ: https://arxiv.org/pdf/0803.4005.pdf, also for ZZDark
%Compare ATLAS and CMS searches for MonoV(had)

\subsubsection{Searches with heavy-flavor quarks}
%ttbar+MET
%reinterpretation of SUSY

\subsubsection{Search signatures including the Higgs boson}
%monoH, H to invisible detailed earlier on

\subsubsection{Two-body mediator searches}

%Dijet and dilepton
%Mention TLA

\subsubsection{Comparison of sensitivity of visible and invisible LHC searches}
\label{sub:comparisonVisibleInvisible}
%It would be nice to make tables of lowest mediator/DM searches+refs, as a poor-person approximation of a summary plot we can't make because ATLAS data not public. 

%%%SUMMARY TABLE FOR MONOX SEARCHES
\begin{table}[h]
\tabcolsep7.5pt
\caption{Summary of searches for BSM mediators at the LHC}
\label{tab:BSMSearchesSummary}
\begin{center}
\begin{tabular}{@{}l|c|c|c|c@{}}
\hline
Signature & Model& \mmed limit & \mdm limit  & Cit.\\
 &  & (\mdm=100 GeV) & (\mmed=100 GeV)  &  \\
%{(}units)$^{\rm a}$ &Head 2 &Head 3 &Head 4 &{(}units)\\
\hline
Jets+\MET & $s-$channel, AV$^{\rm a}$ & Column3 & Column4 & \cite{Sirunyan:2017jix,Aaboud:2017phn} \\
Jets+\MET & $s-$channel, V$^{\rm a}$ & Column3 & Column4 & \cite{Sirunyan:2017jix,Aaboud:2017phn} \\
Jets+\MET & colored scalar & Column3 & Column4 & \cite{Sirunyan:2017jix,Aaboud:2017phn} \\
Photon+\MET & Column 2 & Column3 & Column4 & Column\\
W,Z (had)+\MET 1 & Column 2 & Column3 & Column4 &Column\\
W,Z (lep)+\MET 1 & Column 2 & Column3 & Column4 &Column\\
Higgs+\MET &Column 2 & Column3 & Column4 &Column\\
\hline
\end{tabular}
\end{center}
%\begin{tabnote}
$^{\rm a}$ Coupling values: \gq=0.25, \gdm=1.0; $^{\rm b}$second table footnote.
%\end{tabnote}
\end{table}



%The CMS analysis also scans the coupling-mass plane by fixing the ratio between \mdm and \mmed to ensure perturbativity - nice but maybe we put it later. 
%CMS sentence: Quark couplings down to 0.05 for mediator masses at 50 GeV are excluded for the spin- 1 simplified models as shown in Fig. 12. 

\subsection{Searches for SUSY DM}
\label{sec:results_SUSYSearches}

\subsubsection{Searches for sparticles}

\subsubsection{pMSSM scans}

%mention GAMBIT

\subsubsection{Searches for electroweakinos}


\subsection{Experimental challenges for DM searches at the LHC}
\label{sec:experimentalChallenges}

Now that we have seen the searches for invisible particles and their sensitivity to DM models, we will cover the experimental challenges more in detail, as those will be the key to fully exploit %bleurgh
the Run-2 and Run-3 datasets. 

\subsubsection{Missing transverse momentum}
\label{sub:MET} 

Main points:
\begin{itemize}
\item The measurement of \MET relies on the precise measurement of all reconstructed physics objects. 
\item Some description of \MET significance may be needed, but it may also be too academic. 
\item Fake \MET is rejected using quality cuts.  
\item Pile-up needs specific techniques because of the soft terms. 
\item \MET at the trigger level is the driving reason why we can't go lower, see next section.
\end{itemize}

%Valerio's talk for relevant plots 
%https://indico.cern.ch/event/466934/contributions/2590281/attachments/1489278/2314178/20170706_EPS_DMatATLAS.pdf

%MET significance: in VBF CMS search
%For the 8 TeV dataset, an additional requirement is set on an approximate missing transverse energy significance variable S(Emiss) defined as the ratio of Emiss to the square root of the scalar sum of the transverse energy of all PF objects in the event [62]. Selected events are required to satisfy S(Emiss) > 4?GeV.

\subsubsection{Event selection: triggering on visible and invisible particles}
\label{sub:trigger}

\subsubsection{Precise background estimation}

\subsection{Searches for DM in association with long-lived particles}
\label{sec:results_LLPSearches}

%beginning with the searches that illustrate many of the experimental 
%
%
% and then signals of MET. 
%
% description has a historical and ordering
%We begin with introducing 
%
%start with introducing the searches, mostly  
%
% of DM searches, we will turn to a [categorization] of the searches done so far. After a summary of searches for interactions through SM bosons, we turn to the experimental results 


%As described in the previous chapter, DM itself is not visible at colliders and has
%to be observed indirectly in association with other visible particles. 
%
%Signals of DM can come from MonoX and diX searches. 
%
%State of the art MC:
%
%Vector and scalar models are known at NLO~\cite{Neubert:2015fka,Haisch:2013ata}
%NLO corrections for vector and scalar models in monoZ and monojet
%
%t-channel
%
%From Millie's talk (see Reaction OmniOutliner):
%Since the DM-mediator-quark vertices allow for simultaneous FS partons with different hard scale, particular prescription needs to be used for the generation of samples with different partons that splits samples in number of mediators. Interference between the diagrams is neglected following Papucci et al. 

%%%Bunch of text from ERC that could be useful for TLA part

%This is what I wrote in ERC, not sure if it's usable
%Not knowing the exact nature of these new particles, it is not possible
%to pin down at which energy scales they are going to be produced, 
%and at which rates. This calls for generic searches that are sensitive 
%to a broad range of theoretical benchmarks, and whose results can be easily
%re-interpreted in different frameworks. The absence of any New Physics discoveries 
%during the first LHC run also points to the need for the study of more elaborate 
%and rare processes that can only be achieved with new analysis techniques. 

%At the LHC, proton beams with a center of mass energy of 13 \tera\electronvolt \ will collide up to 
%every 25 \nano\second \ at the four interaction points, where experiments are installed. 
%The ATLAS (\textbf{A} \textbf{T}oroidal \textbf{L}HC \textbf{A}pparatu\textbf{S}) experiment 
%is a general purpose detector located at the Interaction Point 1.
%From the Spring of 2010 to Fall 2012 the LHC has recorded a total of 25 \invfb of collision data 
%at 7 and 8 \tera\electronvolt. 
%
%It is the high interaction rate of proton-proton collisions at the LHC that 
%allows the collection of the large dataset necessary to search for rare processes. 
%The design and operation of the ATLAS detector is described in 
%References~\cite{DetectorPaper}. The key subsystems employed in this projects are the calorimeters, 
%where energy deposits from the collimated sprays of particles originated by 
%quarks and gluons are detected. The energy deposits are used as inputs of 
%\textit{jet algorithms}~\cite{Salam:2009jx}: 
%the experimental output used for the analysis is a reconstructed jet that, after calibration, 
%contains information on the kinematics of the original physics process. 
%The system and technologies employed to select and collect interesting data are the crucial components for the 
%innovative Trigger Level Analysis searches outlined in WP2 and WP3, 
%and they are described in more detail in Section~\ref{subsub:datacollection}. 
%
%The LHC will restart operations in Summer 2015, increasing the center of mass energy to 13 TeV. 
%The dataset collected over the course of the planned three years of operations (called \textit{Run 2}, 
%ongoing until mid-2018) corresponds to 100 \invpb~\cite{KEKRossi}. 
%Upgrades to further increase the data rate and the center of mass energy to 14 TeV will be 
%undertaken during the course of 2016, and completed during the 1.5 year technical stop (called LS2).
%LS2 will last until the beginning of 2020, leading to the third LHC run 
%that will collect an additional 200 \invpb of data. 
%
%During LS2, the LHC experiments will undergo major upgrades to their hardware (called Phase-I upgrades). 
%In the final two years of this project that coincide with the LHC Run 3, 
%I will exploit the new gFEX trigger board~\cite{Bartoldus:1602235} to extend the Trigger-Level Analysis 
%method to obtain an increased sensitivity for four-jet final states. 
%
%\subsubsection{Data collection and data analysis in ATLAS at the LHC}
%\label{subsub:datacollection}
%
%The data rate delivered by the LHC (up to 40 million collisions per second in nominal conditions) exceeds the 
%current capabilities for recording events offline, both in terms of recording speed and storage space. 
%The rare signal events sought are buried in an overwhelming number of background events.
%It is therefore necessary to have a \textit{trigger} system that in a very short timescale 
%selects the interesting collision events based on the presence of high transverse momentum physics objects 
%(muons, electrons, photons, jets and tau leptons). 
%
%The ATLAS data collection rate at the end of the trigger selection still
%surpasses other ``Big data'' 
%challenges both in academic research and in industry. 
%As an example, ATLAS and the three other main detectors at the LHC produced and recorded 
%13 petabytes of data just in 2010~\cite{NaturePetabyte}. It is clear
%that with such large dataset new ideas and new analysis techniques are needed,
%especially in the case of small deviations over large backgrounds. 
%One such idea consists of moving the data analysis as close
%to the data taking as possible, ultimately removing the need for storage at all. 
%This proposal moves the first steps for the ATLAS detector in this direction, 
%proposing to only retain a subset of the event information, and eventually 
%to move an entire search to be done in real-time as soon as the data
%is collected. 
%
%\paragraph{The ATLAS Trigger system}
%
%The ATLAS trigger system is subdivided in two levels: Level-1 and High Level Trigger (HLT)
%The first, fast selection is made at L1 using coarse detector information. 
%Its logic is mostly hardwired in the readout electronics, 
%given that the decision needs to be made in less than 2.5 $\mu$ \second. 
%The bandwidth available for data transfer to the HLT and the HLT computing power
%limit the rate that could be accepted by the L1 to 75 kHZ from the initial 40 MHz 
%provided by the LHC~\cite{Sfyrla:1510140}. The events accepted by the L1 trigger 
%are passed on to the HLT trigger, a software subsystem. The longer latency allows 
%to perform a more refined reconstruction of the objects that are used for the selection.
%Jets reconstructed within the HLT system will be called \textit{trigger jets} in the following,
%while \textit{offline jets} are those reconstructed after the event has passed the full trigger chain.
%During the 2012 data taking, ATLAS recorded and reconstructed data at a rate of 
%400 Hz, leaving an additional 200 Hz of recorded data for later reconstruction
%(\textit{delayed} data).  
%
%Since the rate for certain signatures (e.g. multi-jet events that are backgrounds of the 
%searches outlined in this proposal) would saturate the limited bandwidth that needs to be shared 
%by all triggers, some triggers are \textit{prescaled}. 
%This means that only a fraction of the events accepted 
%are effectively passed onto the next level.
%A prescale of 1 means that all events 
%selected by the trigger are accepted, while larger prescales mean that only a fraction 
%1/prescale is accepted. This in turn harms the sensitivity of searches, as signal events
%will be rejected by the trigger system as well. 

%Blurb from ERC

%Hadronic jets are a particularly promising final state for both the Dark Matter mediators 
%produced at the LHC, but also for new, unknown particles that might be created when crossing 
%the threshold of a new energy scale such as in the upcoming LHC run. 
%Not knowing the exact nature of these new particles, it is not possible
%to pin down at which energy scales they are going to be produced, 
%and at which rates. This calls for generic searches that are sensitive 
%to a broad range of theoretical benchmarks, and whose results can be easily
%re-interpreted in different frameworks. The absence of any New Physics discoveries 
%during the first LHC run also points to the need for the study of more elaborate 
%and rare processes that can only be achieved with new analysis techniques. 

%A wide range of theoretical models attempt to incorporate Dark Matter. 
%Many of these models postulate the presence of a new massive subatomic particle
%that interacts only feebly with SM particles as a Dark Matter candidate.
%The presence of these Weakly Interacting Massive Particles (WIMPs) can be inferred in 
%in a variety of experiments. \textit{Direct Detection} 
%experiments detect the interaction between an incoming 
%Dark Matter particle and target nuclei within the detector, by measuring nucleon recoil.
%\textit{Indirect Detection} experiments detect the fluxes of SM particles that 
%are produced from the annihilation of DM particles~\cite{Bertone:2004pz, Bauer:2013ihz}. 
%Dark Matter searches at colliders are supported by the consistency
%of the thermal relic density and the annihilation rates of WIMP candidates with masses
%in the GeV-TeV range, compatible with the energy regime of the LHC~\cite{Bertone:2004pz}. 
%The complementarity of those searches in terms of the WIMP-SM interaction of interest 
%is shown in Figures~\ref{fig:complementarity} (a-c). 
%
%\begin{wrapfigure}{L}{0.7\textwidth}
%% \begin{figure}[!h]
%\centering
%    \includegraphics[width=0.65\textwidth]{figures/SimplifiedModels}
%  \caption[Complementarity of DM searches]{\label{fig:complementarity} Sketch showing the complementarity 
%  between different experiments searching for Dark Matter (a-c). The difference between
%  the EFT approach and the simplified model approach is depicted for collider searches in (c-d).}
%% \end{figure}
%\end{wrapfigure}
%
%Indirect detection experiments such as 
%FERMI~\cite{Hooper:2010mq}, AMS 
%~\cite{PhysRevLett.113.121101} and DAMA~\cite{Bernabei:2003za}
%have observed tantalizing signals 
%with a possible Dark Matter explanation.
%There some tension between direct detection experiments: the signal-like excesses 
%and characteristics of events in the CDMS and CoGENT experiments~\cite{Agnese:2013rvf,Hooper:2010uy}
%are not confirmed by other experiments such as LUX~\cite{Akerib:2013tjd}. 
%
%The flagship searches for Dark Matter at the LHC and at the Tevatron 
%exploit the recoil of undetected pair-produced WIMPs against a jet radiated 
%by one of the initial-state quarks or gluons. Such searches have yet to find 
%evidence for WIMPs (see e.g. Refs. ~\cite{Aad:2013oja,ATLAS:2014wra,Aad:2014vka,Aad:2014vea,Aad:2014tda,ATLAS-CONF-2012-147}
%for the ATLAS Collaboration). 
%
%Results from all these experiments need to be connected in a coherent framework for a 
%successful program of study of Dark Matter in the coming years. 
