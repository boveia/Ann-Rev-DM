In this chapter, we identify a set of future directions for searches of dark matter at colliders that will become interesting in the coming years. 
We note in any case that this is not an exhaustive list, and gives only a personal perspective. 

Idea here: have a couple of sentences in future directions divided by chapter. One page max. 

%Higgs and Z portals
%For future: mention future e+e- colliders in VH production. 
%New: https://arxiv.org/abs/1712.07237
Mention inclusive vs exclusive searches, monojet is the basic event selection then one selects more things on top of that. 
%Link to:
%Moreover, the sensitivity hierarchy of \MET+X searches does not privilege the jet+\MET final state if there is a direct new physics coupling between a vector boson and the DM, as in the case of the EFT model mentioned in Section~\ref{sub:EFT}, or if the radiated object is a new particle~\cite{Autran:2015mfa}. 
%This latter signal motivates searches in the \MET+generic resonance final state CD: keep for later. 

A generic event selection for excesses of \MET also provides an inclusive sample for more targeted searches as it will be discussed in Chapter~\ref{sec:05_Future}. 

Chapter 2

What models are we developing? 
- the scalar has low x-sec, will be more interesting when we can probe with mu < 1 
- LLPs and why and how
- t-channel are going away from the dominance of s-channel
- SUSY electroweak is the future (Mike Hance's talk?)

%too LLP? "we need this, an example of a way to do it is here"
Motivated by the wide variety of theoretically possible, experimentally challenging long-lived particle signatures, various efforts are ongoing to derive a prioritized set of benchmark models for collider searches in a similar fashion as Ref.~\cite{Abercrombie:2015wmb}. Among the various possibilities, we note the bottom-up approach adopted in~\cite{Buchmueller:2017uqu}, choosing masses and couplings for the models described in~\ref{sub:simplifiedModels} so that they must include a long-lived particle. The categorization of the models by production operator and final state permits to adopt a more systematic set of benchmarks for this kind of signatures. These models can then be mapped onto more complete theories. However, no attempts have yet been made to connect the models in~\cite{Buchmueller:2017uqu} to cosmological history. 

Chapter 3

Remind people of how TLA was developed out of a hole, and in the same way develop simplified model connection to dark photons, are there holes?
- search for other scalars below the Higgs mass, lighter than the Z (1612.07282)
- this will need special triggering strategies, especially track triggers and triggerless readout (cite LHCb's MikeW)

Try and do searches with MET+ISR+weird stuff (or ISR is a new resonance like the Z' but do we really have to cite Whiteson?)

HL-LHC and future colliders a couple of sentences
- phil's stuff: under a particular set of assumptions, only the scalar will be left (i don't like that message)
- making use of highest lumi/CoM requires precise understanding of backgrounds
- some LLP (cite Curtin)

Chapter 4

Next generation of ID and DD, and maybe directional, will give us a lot more results on thermal wimps

%"We advocate that": not sure i want to put that in
A further step towards a more informative comparison between these results is to convey an idea of the impact of those uncertainties in the plots. 
The main uncertainties for LHC searches have been outlined in Sec.~\ref{03_ExperimentalResults}, and they are traditionally displayed as one-sigma bands around the LHC constraints. 
Additional uncertainties in the procedure that translate collider results in the DD and ID planes stem from the matrix elements of the nuclei, but this is an uncertainty that
affects DD as well. %Could make this clearer and cite? 
The situation is less straightforward when comparing collider results to measurements subject to astrophysical uncertainties (for a summary of DD and ID uncertainties, see~\cite{Feldstein:2014ufa,d300ef23986a49099715e661295a4d72} and references therein) as those are difficult to estimate and can have a large impact that affects different kinds of experiments differently. 
The experimental and theoretical communities have not yet agreed on a common presentation of experimental uncertainties, but we expect the discussion will continue in the future. 


It is also worth mentioning that, even though comparisons between collider, DD and ID are becoming the standard for these communities,
the comparison with LHC, DD and ID results with results from astrophysics beyond the relic density is a growing field. 
This is particularly important since all the observational evidence of DM that we have is gravitational,
so DM properties such as mass and density in our and other galaxies
can be inferred from galaxy simulations or deviations in astrophysical observables, 
and signatures of DM self-interactions in rotation galaxies or cluster collisions can complement and verify any DD 
observations. Even though we do not cover this topic in detail in this review we refer to~\cite{Buckley:2017ijx}. 