Here we highlight some topics to watch in the coming decade of dark matter searches at colliders.

% Future Issues
\begin{issues}[FUTURE ISSUES]
\begin{enumerate}

\item The obvious directions for LHC searches are towards lower-rate processes and processes that are more difficult to detect.

\begin{itemize} 

\item For such processes, extended scalar sectors and electroweak SUSY are among possible benchmarks. As Run-2 progresses, LHC searches are becoming sensitive to the simplest (pseudo-)scalar simplified models, opening the door to more realistic models.
On a longer timescale, the HL-LHC dataset will bring sensitivity to up to 3 TeV in scalar mediator masses for unit couplings~\cite{CMS-PAS-FTR-16-005} and precision knowledge of the SM Higgs sector, e.g. at the level of 10, and the mass reach for electroweak production of SUSY partners increases by a few hundred GeV~\cite{Campana:2016cqm}.

\item With data arriving at a slightly less frantic pace, experimentally challenging long-lived particle signatures are a growing field, and benchmarks similar to Ref.~\cite{Abercrombie:2015wmb} are needed to help guide the design of these searches. Among many ongoing efforts, we note the bottom-up approach adopted in Ref.~\cite{Buchmueller:2017uqu}, which connects such models with the long-lived-particle limit of those in Chapter~\ref{sub:simplifiedModels}. 

\end{itemize} 

\item Precision searchers and efficient triggering of rare signals buried in large backgrounds are key to fully exploiting the HL-LHC dataset.
  LHCb will make use of a novel triggerless detector readout to perform dark photon searches at unprecedented sensitivities~\cite{Ilten:2016tkc}.
  These sorts of efforts should inspire ATLAS and CMS to search for rare processes involving relatively light new particles, a subject they've left largely unexplored~\cite{Alves:2016cqf}.
  In the familiar jet+\MET search, precision estimates of the V+jet backgrounds, and of the inputs to these predictions, will be crucial. Efforts are ongoing in that direction~\cite{Blumenschein:2018gtm}. 

\item Future hadron and electron-positron colliders have immense potential (see e.g. the studies towards a Future Hadron Collider~\cite{Golling:2016gvc}). Nevertheless, present studies largely continue the approaches already in use at the LHC. A new hadron collider would be built to discover new physics, and therefore qualitatively different experimental design, benchmark models, and analysis strategies should be considered.

\item More useful working comparisons between results from colliders, underground searches, and observatories should take into account the uncertainties on each type of result, and on the extrapolations between them.
  The main uncertainties for LHC searches have been outlined in Sec.~\ref{sec:03_ExperimentalResults} and in the experimental references.
  For a summary of DD and ID uncertainties, see Ref.~\cite{Feldstein:2014ufa,d300ef23986a49099715e661295a4d72} and references therein.

\item Comparisons among collider and non-collider particle physics experiments are becoming standard, relating particle physics to astrophysical observables is crucial to exploit the few clues that dark matter can provide about particle physics beyond the SM. We highly encourage further work on this subject, e.g. Ref~\cite{Buckley:2017ijx}

\end{enumerate}
\end{issues}
