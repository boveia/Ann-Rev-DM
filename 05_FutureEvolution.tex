In the following, we give a brief, personal list of future directions for search of dark matter at colliders in the coming decade. 

% Future Issues
\begin{issues}[FUTURE ISSUES]
\begin{enumerate}

\item In absence of new physics signals with large couplings with the LHC data so far, the obvious direction 
for invisible particle searches at colliders is towards lower rate reactions
and signatures that are more difficult to detect. 

\begin{itemize} 

\item Extended scalar sectors and electroweak SUSY are an example of such benchmark models. LHC Run-2 searches are only starting to be sensitive to the simplest (pseudo-)scalar models, and the development of further extensions is motivated by the expected sensitivity to this type of interactions. On a longer timescale, the HL-LHC dataset will bring a sensitivity of up to 3 TeV in scalar mediator masses for unit couplings. Full-dataset HL-LHC results~\cite{Campana:2016cqm} will bring considerable improvements in the precision knowledge of the SM Higgs sector, with a sensitivity to Higgs to invisible decays from couplings measurements down to 10\%. %biggg dealllll  
The sensitivity to the electroweak production of SUSY particles will be increased, gaining from 50\% to 100\% in the superpartner mass reach. %(from 50 to 100\%)
%http://cms-results.web.cern.ch/cms-results/public-results/preliminary-results/FTR-16-005/CMS-PAS-FTR-16-005_Figure_008.png

\item Motivated by the wide variety of theoretically possible, experimentally challenging long-lived particle signatures, various efforts are ongoing to derive a prioritized set of benchmark models for collider searches in a similar fashion as Ref.~\cite{Abercrombie:2015wmb}. Such a categorization will provide a map for more a efficient coverage of a wide and varied territory as well as aid reinterpretation of searches. Joint efforts between the experimental and theory community are crucial for this purpose. Among the various possibilities, we note the bottom-up approach adopted in~\cite{Buchmueller:2017uqu}, choosing masses and couplings for the models described in~\ref{sub:simplifiedModels} so that they must include a long-lived particle. From the experimental side, complementing  generic event selections targeting invisible particles and adding a requirement will increase the sensitivity of these searches to specific unconventional signatures. 

%The categorization of the models by production operator and final state permits to adopt a more systematic set of benchmarks for this kind of signatures, that can then be mapped onto more complete theories. 

\end{itemize} 

\item Efficient data taking and precision searches will be among the key challenges in the near future for fully exploiting the LHC dataset, starting with the upcoming LHC run and continuing towards the HL-LHC. Here we bring the example of dark photon searches at LHCb, which from Run-3 will make use of a novel triggerless detector readout and reach unprecedented sensitivities for models of dark boson mediation with kinetic mixing~\cite{Ilten:2016tkc}. This can also inspire other experiments to search for new particles below half the Higgs mass, in a region of phase space that is still largely unexplored by ATLAS and CMS~\cite{Alves:2016cqf}. Another example is the familiar jet+\MET search with the full HL-LHC dataset: after experimental uncertainties are tamed, precision measurements of the V+jet backgrounds and understanding of the parton distribution functions will be crucial to search for even rarer signals. Efforts are ongoing in that direction~\cite{Blumenschein:2018gtm}. 

%https://arxiv.org/pdf/1802.02100.pdf

\item The potential of future electron-positron precision machines and higher-energy hadron machines in terms of searches for visible and invisible particles that may be connected to dark matter is immmmmmense (see e.g. the studies towards a Future Hadron Collider~\cite{Golling:2016gvc}), but they will need to be informed by findings at the LHC. Nevertheless, studies are ongoing, as the preparations have to start soon enough. 

\item A further step towards a more informative comparison between these results is to convey an idea of the impact of those uncertainties in the plots. 
The main uncertainties for LHC searches have been outlined in Sec.~\ref{03_ExperimentalResults}, and they are traditionally displayed as one-sigma bands around the LHC constraints. 
Additional uncertainties in the procedure that translate collider results in the DD and ID planes stem from the matrix elements of the nuclei, but this is an uncertainty that affects DD as well. %Could make this clearer and cite? 
The situation is less straightforward when comparing collider results to measurements subject to astrophysical uncertainties (for a summary of DD and ID uncertainties, see~\cite{Feldstein:2014ufa,d300ef23986a49099715e661295a4d72} and references therein) as those are difficult to estimate and can have a large impact that affects different kinds of experiments differently. 
The experimental and theoretical communities have not yet agreed on a common presentation of experimental uncertainties, but we expect the discussion will continue in the future. 

\item It is also worth mentioning that, even though comparisons between collider, DD and ID are becoming the standard for these communities,
the comparison with LHC, DD and ID results with results from astrophysics beyond the relic density is a growing field. 
This is particularly important since all the observational evidence of DM that we have is gravitational,
so DM properties such as mass and density in our and other galaxies
can be inferred from galaxy simulations or deviations in astrophysical observables, 
and signatures of DM self-interactions in rotation galaxies or cluster collisions can complement and verify any DD 
observations. Even though we do not cover this topic in detail in this review we refer to~\cite{Buckley:2017ijx}. 


\end{enumerate}
\end{issues}
