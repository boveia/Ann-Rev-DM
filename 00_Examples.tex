% Heading 2
\subsection{Second-Level Heading}
This is dummy text. This is dummy text. This is dummy text. This is dummy text.

% Heading 3
\subsubsection{Third-Level Heading}
This is dummy text. This is dummy text. This is dummy text. This is dummy text. 

% Heading 4
\paragraph{Fourth-Level Heading} Fourth-level headings are placed as part of the paragraph.

%Example of a Figure
\section{ELEMENTS\ OF\ THE\ MANUSCRIPT} 
\subsection{Figures}Figures should be cited in the main text in chronological order. This is dummy text with a citation to the first figure (\textbf{Figure \ref{fig1}}). Citations to \textbf{Figure \ref{fig1}} (and other figures) will be bold. 

\begin{figure}[h]
\includegraphics[width=3in]{SampleFigure}
\caption{Figure caption with descriptions of parts a and b}
\label{fig1}
\end{figure}

% Example of a Table
\subsection{Tables} Tables should also be cited in the main text in chronological order (\textbf {Table \ref{tab1}}).

\begin{table}[h]
\tabcolsep7.5pt
\caption{Table caption}
\label{tab1}
\begin{center}
\begin{tabular}{@{}l|c|c|c|c@{}}
\hline
Head 1 &&&&Head 5\\
{(}units)$^{\rm a}$ &Head 2 &Head 3 &Head 4 &{(}units)\\
\hline
Column 1 &Column 2 &Column3$^{\rm b}$ &Column4 &Column\\
Column 1 &Column 2 &Column3 &Column4 &Column\\
Column 1 &Column 2 &Column3 &Column4 &Column\\
Column 1 &Column 2 &Column3 &Column4 &Column\\
\hline
\end{tabular}
\end{center}
\begin{tabnote}
$^{\rm a}$Table footnote; $^{\rm b}$second table footnote.
\end{tabnote}
\end{table}

% Example of lists
\subsection{Lists and Extracts} Here is an example of a numbered list:
\begin{enumerate}
\item List entry number 1,
\item List entry number 2,
\item List entry number 3,\item List entry number 4, and
\item List entry number 5.
\end{enumerate}

Here is an example of a extract.
\begin{extract}
This is an example text of quote or extract.
This is an example text of quote or extract.
\end{extract}

\subsection{Sidebars and Margin Notes}
% Margin Note
\begin{marginnote}[]
\entry{Term A}{definition}
\entry{Term B}{definition}
\entry{Term C}{defintion}
\end{marginnote}

\begin{textbox}[h]\section{SIDEBARS}
Sidebar text goes here.
\subsection{Sidebar Second-Level Heading}
More text goes here.\subsubsection{Sidebar third-level heading}
Text goes here.\end{textbox}



\subsection{Equations}
% Example of a single-line equation
\begin{equation}
a = b \ {\rm ((Single\ Equation\ Numbered))}
\end{equation}
%Example of multiple-line equation
Equations can also be multiple lines as shown in Equations 2 and 3.
\begin{eqnarray}
c = 0 \ {\rm ((Multiple\  Lines, \ Numbered))}\\
ac = 0 \ {\rm ((Multiple \ Lines, \ Numbered))}
\end{eqnarray}