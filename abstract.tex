The field of particle physics is increasingly keen to understand what Dark Matter (DM) is, if it is indeed a particle. 
Some experiments, termed Direct Detection (DD) experiments, look for galactic DM colliding with underground targets made of Standard Model matter (SM)~\cite{0954-3899-43-1-013001}.
Others, termed Indirect Detection (ID) experiments, search for the products of annihilating dark matter concentrated within the gravitational potential wells of the Milky Way and elsewhere~\cite{Gaskins:2016cha}.
None of these experiments has yet found conclusive evidence of DM.
If the only interaction between DM and SM matter is gravitational, experiments will never see it.
Yet the search for particle DM started relatively recently, and plenty of room for optimism remains.

Colliders, one of the most successful tools of particle physics, have revealed much about SM matter.
This review will sketch how colliders can contribute to the search for DM, focusing on the highest-energy collider currently in operation, the Large Hadron Collider at CERN.
Absent hints for the character of DM-SM interactions, it emphasizes what could be observed in the near future, the main experimental challenges presented, and how collider searches fit into the broader field.
Finally, it underlines a few areas to watch for the future LHC program.

