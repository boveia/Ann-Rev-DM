Colliders, one of the most successful tools of particle physics, have revealed much about ordinary matter.
This review will sketch how colliders contribute to the search for particle dark matter, focusing on the highest-energy collider currently in operation, the Large Hadron Collider at CERN.
Absent hints for the character of interactions between dark matter and ordinary matter, it emphasizes what could be observed in the near future, the main experimental challenges presented, and how collider searches fit into the broader field.
Finally, it underlines a few areas to watch for the future LHC program.
