%What we are missing

%A description of MET
%A simplified diagram of an s-channel process at colliders satisfying $Z_2$ symmetry is shown in panel (b) of Fig.~\ref{fig:monoX}.
%the following sentence: 
%The observation of a signal of visible or invisible particles at an LHC experiment that could be identified as being generated by one of the reactions described in this review cannot lead to claims that DM has been discovered. This is not a reason to discount searches for DM at the LHC, as such a signal would still be a groundbreaking discovery, regardless of its interpretation. Instead, we highlight the importance of the comparison of LHC results, where DM would be produced in the lab, with the results of complementary experiments that look for signals of DM coming from space. This comparison can only take place if the same theoretical model is used to interpret both results. This motivates the enumeration of possible models in this chapter. 

%Sidebar (50 words minimum, 200 words maximum) briefly discussing a fascinating adjacent topic; 
%insert below Literature Cited section, but indicate near which section in text the sidebar should be typeset
\begin{textbox}[!h]
\section{Particle properties of Dark Matter}

%Observations from astrophysics can inform experiments on DM particle targets, and whether newly discovered particles can be identified as DM. 
%We list here the most relevant consequences of these observation for the benchmark models used for collider DM searches described in this review. 

\textbf{Stability}
If DM is a particle, it does not seem to decay.
Conservation laws, such as R-parity in Supersymmetry (SUSY) or a $Z_2$ symmetry, can prevent the DM particle from decaying into any lighter even-parity SM particle.
Additionally, pairs of DM particles can be produced by the decay of other particles, charged under the same gauge group as the SM, or singly in the case the parent is a color triplet. 
%We also note that, while DM is stable on a cosmological scale, collider experiments are limited to the observation of particles with a lifetime that is longer than the time needed to escape the detector. 
%For this reason, we use the term "invisible particles" in collider reactions, rather than the term DM particles. 
%(i.e. DM candidate particles could still decay into other particles outside the detector and leave a signal of missing transverse momentum).

\textbf{Darkness} 
DM particles are effectively invisible to traditional collider experiments made of ordinary matter. However, the rest of the event is not. 
Invisible particles can be accompanied by one or more visible recoiling particles, leading to missing momentum in the transverse plane, whose magnitude is termed \MET. This is one of the main signatures of DM at colliders.

%\textbf{Observability of DM}
%Even though models of particle dark matter include SM couplings to satisfy cosmological observations under certain assumptions, these couplings need to be weak.
%~\footnote{We note that the only SM particle that satisfies this requirement of being sufficiently weakly interacting is the neutrino. However, neutrinos cannot make up the totality of DM as they are relativistic particles and cannot explain the galaxy structures that formed in the  universe~\cite{PlehnLecturesDM}.}.
%assumption of thermal freeze-out 
%These couplings determine the reach of collider searches, as they drive the production of new particles and their observability in the detectors. 
%so what
%if the couplings to the (constituents of) the colliding particles are enough to directly . Exceptions exist in the case of models where the SM and the DM interact only through a dark portal.
\end{textbox}
