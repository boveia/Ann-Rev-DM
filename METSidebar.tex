\begin{textbox}[!h]
\section{MEASURING INVISIBLE PARTICLES: \MET RECONSTRUCTION}

Precise measurements throughout each detector systems are crucial to measure \MET in experiments at hadron colliders, 
as its calculation should include all particles in the event.
Contributions that are not attributed to physics objects form the soft component of the \MET~\cite{Aad:2016nrq,CMS-PAS-JME-16-004}. 
One of the main challenges for \MET measurements is excluding contributions from additional proton-proton interactions whose debris are detected at nearly the same time as the hard scatter (pile-up).
Searches for invisible particles also need to reject events with large \MET if the visible energy is due to non-collision backgrounds. 

\textbf{Challenge: pile-up in \MET reconstruction.} 
In addition to suppressing pile-up suppression within the calorimeters, tracking information can be used to determine whether energy deposits originate from the primary collision vertex. 
The combination of this information is used to remove pile-up both in the physics objects used for \MET calculation and in the overall event energy balance~\cite{CMS-PAS-JME-16-004,ATLAS-CONF-2014-019}. 

\textbf{Challenge: fake \MET rejection.} 
Non-collision backgrounds, such as cosmic rays, beam background and detector noise have a significant contribution to the tails of the \MET spectrum, as shown in Fig.~\ref{fig:fakeMET}
Specific quality cuts, based on the presence of tracks associated to the deposited energy and on energy deposited in the various calorimeter layers are applied to reject these events~\cite{ATLAS-CONF-2015-029}. The number of events passing the jet+\MET analysis selection before these quality cuts is about ten times larger than the SM contribution~\cite{Aaboud:2016tnv}. 
\end{textbox}