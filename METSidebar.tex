\begin{textbox}[!h]
\section{Measuring invisible particles: \MET reconstruction}

%10 words per line, 20 lines
While at lepton colliders the initial center of mass energy is known and 
can be used as an additional constrain to measure the momentum imbalance due to escaped invisible particles in the final state, measuring missing transverse momentum at hadron colliders
% TJ: Not sure why you make this distinction?
% I tend to think about the lepton collider case as offering
% one extra constraint, i.e. on the total missing *energy*
% but the directional MPT constraint still requires an
% inclusive measurement
needs to rely on the precise measurement of detector quantities. Measuring \MET means summing all energy deposited in the detector within the hard scatter event, avoiding contamination from other sources that include simultaneous proton-proton interactions (pile-up). The first challenge for \MET measurements is to find a balance between measuring inclusively and avoiding contamination from these other sources, as \MET also encompasses the soft energy deposits not included in reconstructed objects such as jets or electrons. 
% For theorists, is it better to describe it as:
% We need to add up literally everything from the HS event
% but from that event *only*, thereby creating a large tension
% between inclusiveness and contamination? Then add that
% soft recoil can be important but in general is harder to
% measure with a good scale
One also needs to reject \MET contributions from additional proton-proton interactions within the same collision event, and events with large \MET originating from spurious non-collision backgrounds such as cosmic rays and detector noise. 
% Maybe just state the challenges in generic form here,
% then state that they are PU and NCB in the following boxes?

\textbf{Challenge: pile-up in \MET reconstruction.} 
Energy deposits from proton-proton interactions that are uncorrelated 
with the one of interest can have a significant contribution to the overall 
transverse momentum balance. This is why the combination of tracking information, 
event and object properties are used not only to remove
pile-up in the physics objects used to reconstruct \MET, but also in 
the overall event energy balance. [cite: https://cds.cern.ch/record/2205284/files/JME-16-004-pas.pdf, 
asked Emma and TJ for best reference]
The lack of tracking information in the trigger system is a limiting factor 
in selecting events with low \MET at the trigger level,
as the rates grow exponentially with the number of additional interactions. 
%from https://twiki.cern.ch/twiki/pub/AtlasPublic/MissingEtTriggerPublicResults/metxs_vs_mu.pdf

\textbf{Challenge: fake \MET rejection.} 
Large \MET causing fake contributions to the highest signal regions in LHC DM searches
originates from non-collision backgrounds, such as cosmic rays and detector noise. Specific
quality cuts, including tracking information and the structure of the energy deposited 
in the various calorimeter layers are applied to reject these events. 
The number of events passing the jet+\MET analysis selection before these quality cuts
is ten times larger than the expected SM contribution [cite monojet paper, but only 3.2/fb has fig?].  

\end{textbox}


%\subsubsection{Missing transverse momentum}
%\label{sub:MET} 

%Main points:
%\begin{itemize}
%\item The measurement of \MET relies on the precise measurement of all reconstructed physics objects. 
%\item Some description of \MET significance may be needed, but it may also be too academic. 
%\item Fake \MET is rejected using quality cuts.  
%\item Pile-up needs specific techniques because of the soft terms. 
%\item \MET at the trigger level is the driving reason why we can't go lower, see next section.
%\end{itemize}

%from ooutline

%- Mismeasured MET (combining instrumental effects and beam/cosmics background)				
%	- CDF				
%		- beam background: exploit track pointing to jet and calorimeter layers				
%		- QCD: shitty method from Mario (extrapolation changing the veto)				
%	- LHC:				
%		- beam backgrounds: like CDF, more refined				
%			- can have a % of how many events would have been				
%		- QCD: matrix method a la SUSY				
%	- Other backgrounds (diboson, top)				
%		- Small so using MC				
%		- LHC has validation regions				
%			- check ttbar				

%Valerio's talk for relevant plots 
%https://indico.cern.ch/event/466934/contributions/2590281/attachments/1489278/2314178/20170706_EPS_invisible particlesatATLAS.pdf

%MET significance: in VBF CMS search
%For the 8 TeV dataset, an additional requirement is set on an approximate missing transverse energy significance variable S(Emiss) defined as the ratio of Emiss to the square root of the scalar sum of the transverse energy of all PF objects in the event [62]. Selected events are required to satisfy S(Emiss) > 4?GeV.