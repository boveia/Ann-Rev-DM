\documentclass[review]{elsarticle}

\usepackage{lineno,hyperref}
%\usepackage{authblk}
\modulolinenumbers[5]

\journal{Physics of the Dark Universe}

%%%%%%%%%%%%%%%%%%%%%%%
%% Elsevier bibliography styles
%%%%%%%%%%%%%%%%%%%%%%%
%% To change the style, put a % in front of the second line of the current style and
%% remove the % from the second line of the style you would like to use.
%%%%%%%%%%%%%%%%%%%%%%%

%% Numbered
%\bibliographystyle{model1-num-names}

%% Numbered without titles
%\bibliographystyle{model1a-num-names}

%% Harvard
%\bibliographystyle{model2-names.bst}\biboptions{authoryear}

%% Vancouver numbered
%\usepackage{numcompress}\bibliographystyle{model3-num-names}

%% Vancouver name/year
%\usepackage{numcompress}\bibliographystyle{model4-names}\biboptions{authoryear}

%% APA style
%\bibliographystyle{model5-names}\biboptions{authoryear}

%% AMA style
%\usepackage{numcompress}\bibliographystyle{model6-num-names}

%% `Elsevier LaTeX' style
\bibliographystyle{elsarticle-num}

%%Bibliography additions (not compiling yet!)
\usepackage{etex}
\reserveinserts{20}
%,backref=true,maxcitenames=3,maxbibnames=3,
%\usepackage[hyperref=true,url=false,backend=bibtex,style=alphabetic,backref=false,firstinits=true,doi=false,eprint=true,language=USenglish]{biblatex}
% from http://tex.stackexchange.com/questions/176965/biblatex-sentence-case-and-math-mode-not-working-together
%\DeclareFieldFormat{sentencecase}{\MakeSentenceCase{#1}}
%\renewbibmacro*{title}{%
%	\ifthenelse{\iffieldundef{title}\AND\iffieldundef{subtitle}}
%	{}
%	{\ifthenelse{\ifentrytype{article}\OR\ifentrytype{inbook}\OR\ifentrytype{report}%
%			\OR\ifentrytype{incollection}\OR\ifentrytype{inproceedings}%
%			\OR\ifentrytype{inreference}}
%		{\printtext[title]{%
%				\printfield[sentencecase]{title}%
%				\setunit{\subtitlepunct}%
%				\printfield[sentencecase]{subtitle}}}%
%		{\printtext[title]{%
%				\printfield[titlecase]{title}%
%				\setunit{\subtitlepunct}%
%				\printfield[titlecase]{subtitle}}}%
%		\newunit}%
%	\printfield{titleaddon}}
% many of the following biblatex commands follow from examples by Ian Brock (ian.brock@cern.ch)
%\DeclareFieldFormat[article]{journaltitle}{#1\isdot}
%\DeclareFieldFormat[article]{journalsubtitle}{#1\isdot}
%\DeclareFieldFormat[article]{volume}{\textbf{#1}\isdot}
%\DeclareFieldFormat[article,inbook,incollection,inproceedings,patent,thesis,unpublished]
%{title}{\emph{#1\isdot}}
%\errorcontextlines=100
%\renewcommand*{\newunitpunct}{\addcomma\space}
%\renewbibmacro{in:}{}
%\renewcommand{\bibpagespunct}{\space}
%\DefineBibliographyStrings{USenglish}{%
%	page = {},
%	pages = {}
%}
%\DefineBibliographyStrings{UKenglish}{%
%	page = {},
%	pages = {}
%}
%%%

%%%%%%%%%%%%%%%%%%%%%%%
%Packages and formatting options
%%%%%%%%%%%%%%%%%%%%%%%

\usepackage{graphicx}
\usepackage[svgnames]{xcolor}
\usepackage{hyperref}
\usepackage{epstopdf}
\usepackage{xspace} %for single top
\usepackage{longtable} %for ttbar 
\usepackage{slashed}
\usepackage{amsfonts}
\usepackage[T1]{fontenc}
\usepackage{amsmath,amssymb}
\usepackage{slashed}

% begin feynman diagram setup
\usepackage{feynmp-auto}
% before setting default graphics include widths, save the default to properly scale feynman diagrams with their labels
\makeatletter
\let\ginnatwidth\Gin@nat@width
\let\ginnatheight\Gin@nat@height
\makeatother
\setkeys{Gin}{width=\linewidth,totalheight=\textheight,keepaspectratio}
%if you don't use the feynmandiagram environment for feynmf/mp figures, and you've changed the Gin keys above, then you need to manually adjust the unit length for feynman diagrams to get ratio of tex labels to graphics right. \unitlength = 1.21mm for tufte \linewidth and \textheight
\DeclareGraphicsRule{*}{mps}{*}{}
\newenvironment{feynmandiagram}[1][]{\setkeys{Gin}{width=\ginnatwidth,totalheight=\ginnatheight}\begin{fmffile}{#1}
		\begin{fmfgraph*}(100,70)\fmfpen{thick}}{\end{fmfgraph*}\end{fmffile}\setkeys{Gin}{width=\linewidth,totalheight=\textheight,keepaspectratio}}
% end feynman diagram setup

\usepackage{subfig}
\DeclareGraphicsRule{*}{mps}{*}{}

\usepackage{xhfill}% http://ctan.org/pkg/xhfill
\newcommand{\ditto}{~''~}

\setkeys{Gin}{width=\linewidth,totalheight=\textheight,keepaspectratio}
\usepackage{mathtools} % extended mathematics
\usepackage{booktabs} % book-quality tables
\usepackage{units}    % non-stacked fractions and better unit spacing
\usepackage{multicol} % multiple column layout facilities
\usepackage{fancyvrb} % extended verbatim environments
\usepackage{fancyhdr}
\usepackage{refcount}
\usepackage{calc}
\usepackage{lastpage}
%\usepackage{natbib} % for \citep
% create a dummy file (shell command "touch moderntex") to turn on some features that don't work on lxplus
\IfFileExists{moderntex}{
	\usepackage[protrusion=true,expansion=true,tracking=true,kerning=true,spacing=true]{microtype}
}{}
\fvset{fontsize=\normalsize} % default font size for fancy-verbatim environments

%%%Begin nice tables
\usepackage{booktabs,colortbl, array}
\usepackage{rotating}
%%End nice tables

%%%%%%%%%%%
% change typeface to something closer to Elsevier's Gulliver
% remove this before submission
\usepackage[bitstream-charter]{mathdesign}
\usepackage[T1]{fontenc}
%%%%%%%%%%%

% Prints an asterisk that takes up no horizontal space.
% Useful in tabular environments.
\newcommand{\hangstar}{\makebox[0pt][l]{*}}
\newcommand{\openepigraph}[2]{%
	%\sffamily\fontsize{14}{16}\selectfont
	\begin{fullwidth}
		\sffamily\large
		\begin{doublespace}
			\noindent\allcaps{#1}\\% epigraph
			\noindent\allcaps{#2}% author
		\end{doublespace}
	\end{fullwidth}
}

%%%%

\begin{document}

\begin{frontmatter}

\title{Dark Matter Benchmark Models for Early LHC Run-2 Searches: Report of the ATLAS/CMS Dark Matter Forum}
%\author{ATLAS+CMS Dark Matter Forum}

\input{authors.tex}
%% Group authors per affiliation:
%\author{The authors}
%\address{Radarweg 29, Amsterdam}
%\fntext[myfootnote]{Since 1880.}

%% or include affiliations in footnotes:
%\author[mymainaddress,mysecondaryaddress]{Elsevier Inc}
%\ead[url]{www.elsevier.com}

%\author[mysecondaryaddress]{Global Customer Service\corref{mycorrespondingauthor}}
%\cortext[mycorrespondingauthor]{Corresponding author}
%\ead{support@elsevier.com}

%\address[mymainaddress]{1600 John F Kennedy Boulevard, Philadelphia}
%\address[mysecondaryaddress]{360 Park Avenue South, New York}

\begin{abstract}
This document is the final report of the ATLAS-CMS Dark Matter Forum, a forum organized by the ATLAS and CMS collaborations with the participation of experts on theories of Dark Matter, to select a minimal basis set of dark matter simplified models that should support the design of the early LHC Run-2 searches. A prioritized, compact set of benchmark models is proposed, accompanied by studies of the parameter space of these models and a repository of generator implementations. This report also addresses how to apply the Effective Field Theory formalism for collider searches and present the results of such interpretations.
\end{abstract}

\begin{keyword}
	Dark Matter\sep Simplified Models \sep EFT \sep LHC
\end{keyword}

\end{frontmatter}

\input symbols.tex

\linenumbers
%\tableofcontents

\section{Introduction}
\label{sec:Introduction}
\input{tex/Introduction.tex}

\section{\texorpdfstring{Simplified models for all \MET+X analyses}{Simplified models for all MET+X analyses}}
\label{subsec:MonojetLikeModels}
\input{tex/MonojetLikeModels}


\section{Specific models for signatures with EW bosons}
\label{subsec:EWSpecificModels}
\input{tex/EWSpecificModels}

\section{Implementation of Models}
\label{app:MonojetLikeModels_Appendix}
\input{tex/MonojetLikeModels_Appendix.tex}

\section{Presentation of EFT results}
\label{sec:EFTValidity} 
\input{tex/EFTValidity.tex}

\section{Evaluation of signal theoretical uncertainties}
\label{sec:TheoryUncertainties} 
\input{tex/TheoryUncertainties.tex}

\section{Conclusions}
\label{section:conclusions}
\input{tex/Conclusions.tex}

\input{tex/backmatter.tex}

\appendix

\section{Appendix: Additional models for Dark Matter searches}
\label{app:EWSpecificModels_Appendix}
\input{tex/EWSpecificModels_Appendix.tex}

\section{Appendix: Presentation of experimental results for reinterpretation}
\label{app:Presentation_Of_Experimental_Results}
\input{tex/Presentation_Of_Experimental_Results.tex}

\section{Appendix: Additional details and studies within the Forum}
\label{app:Additional_details}
\input{tex/AdditionalDetails.tex}

\section*{References}

\bibliography{doc}

\end{document}