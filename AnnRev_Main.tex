% template.tex, dated April 5 2013
% This is a template file for Annual Reviews 1 column Journals
%
% Compilation using ar-1col.cls' - version 1.0, Aptara Inc.
% (c) 2013 AR
%
% Steps to compile: latex latex latex
%
% For tracking purposes => this is v1.0 - Apr. 2013

\documentclass{ar-1col}
\usepackage[numbers]{natbib}
\usepackage{xspace}
\usepackage{url}
\usepackage{slashed}

\setcounter{secnumdepth}{4}

% Metadata Information
\jname{Xxxx. Xxx. Xxx. Xxx.}
\jvol{AA}
\jyear{YYYY}
\doi{10.1146/((please add article doi))}


% Document starts
\begin{document}

% Page header
\markboth{A. Boveia, C. Doglioni}{Dark Matter Searches at Colliders}

% Title
\title{Dark Matter Searches at Colliders}

% Definitions
\newcommand{\chiDM}{\ensuremath{\chi}\xspace}
\newcommand{\mMed}{\ensuremath{M_{\rm{med}}}\xspace}
\newcommand{\mmed}{\mMed}
\newcommand{\gDM}{\ensuremath{g_{\chiDM}}\xspace}
\newcommand{\gdm}{\gDM}
\newcommand{\gl}{$g_{\mathrm{l}}$\xspace}
\newcommand{\gdmq}{\ensuremath{g_{\chiDM q}}\xspace}
\newcommand{\gq}{$g_{\mathrm{q}}$\xspace}
\newcommand{\ifb}{\ensuremath{\mathrm{fb}^{-1}}\xspace}
\newcommand{\mdm}{\ensuremath{m_{\chiDM}}\xspace}
\newcommand{\mDM}{\mdm}
\newcommand{\ghZprimeZprime}{\ensuremath{g_{hZ'Z'}}\xspace}
\newcommand{\gZPrime}{\ensuremath{g_{Z'}}\xspace}
\newcommand{\gZprime}{\ensuremath{g_{Z'}}\xspace}
\newcommand{\sinthetab}{\ensuremath{sin(\theta_B}\xspace}
\newcommand{\sinthetahS}{\ensuremath{sin(\theta_{hS})}\xspace}
\newcommand{\pt}{\ensuremath{p_\mathrm{T}}\xspace}
\newcommand{\pT}{\ensuremath{p_\mathrm{T}}\xspace}
\newcommand{\MET}{\ensuremath{\slashed{E}_T}\xspace}
\newcommand{\Zprime}{\ensuremath{\mathrm{Z}^\prime}\xspace}
\newcommand{\fb}{\ensuremath{\mathrm{fb}}\xspace}
\newcommand{\pb}{\ensuremath{\mathrm{pb}}\xspace}
\newcommand{\ifb}{\ensuremath{\mathrm{fb}^{-1}}\xspace}
\newcommand{\ipb}{\ensuremath{\mathrm{pb}^{-1}}\xspace}

%Authors, affiliations address.
\author{Antonio Boveia,$^1$ Caterina Doglioni, $^2$
\affil{$^1$The Ohio State University + address}
\affil{$^2$Lund University + address}}

%Abstract
\begin{abstract}
Abstract text, approximately 150 words. 
\end{abstract}

%Keywords, etc.
\begin{keywords}
keywords, separated by comma, no full stop, lowercase
\end{keywords}
\maketitle

%Table of Contents
\tableofcontents


% Heading 1
\section{INTRODUCTION}
\label{sec:intro}

%% Heading 2
\subsection{Second-Level Heading}
This is dummy text. This is dummy text. This is dummy text. This is dummy text.

% Heading 3
\subsubsection{Third-Level Heading}
This is dummy text. This is dummy text. This is dummy text. This is dummy text. 

% Heading 4
\paragraph{Fourth-Level Heading} Fourth-level headings are placed as part of the paragraph.

%Example of a Figure
\section{ELEMENTS\ OF\ THE\ MANUSCRIPT} 
\subsection{Figures}Figures should be cited in the main text in chronological order. This is dummy text with a citation to the first figure (\textbf{Figure \ref{fig1}}). Citations to \textbf{Figure \ref{fig1}} (and other figures) will be bold. 

\begin{figure}[h]
\includegraphics[width=3in]{SampleFigure}
\caption{Figure caption with descriptions of parts a and b}
\label{fig1}
\end{figure}

% Example of a Table
\subsection{Tables} Tables should also be cited in the main text in chronological order (\textbf {Table \ref{tab1}}).

\begin{table}[h]
\tabcolsep7.5pt
\caption{Table caption}
\label{tab1}
\begin{center}
\begin{tabular}{@{}l|c|c|c|c@{}}
\hline
Head 1 &&&&Head 5\\
{(}units)$^{\rm a}$ &Head 2 &Head 3 &Head 4 &{(}units)\\
\hline
Column 1 &Column 2 &Column3$^{\rm b}$ &Column4 &Column\\
Column 1 &Column 2 &Column3 &Column4 &Column\\
Column 1 &Column 2 &Column3 &Column4 &Column\\
Column 1 &Column 2 &Column3 &Column4 &Column\\
\hline
\end{tabular}
\end{center}
\begin{tabnote}
$^{\rm a}$Table footnote; $^{\rm b}$second table footnote.
\end{tabnote}
\end{table}

% Example of lists
\subsection{Lists and Extracts} Here is an example of a numbered list:
\begin{enumerate}
\item List entry number 1,
\item List entry number 2,
\item List entry number 3,\item List entry number 4, and
\item List entry number 5.
\end{enumerate}

Here is an example of a extract.
\begin{extract}
This is an example text of quote or extract.
This is an example text of quote or extract.
\end{extract}

\subsection{Sidebars and Margin Notes}
% Margin Note
\begin{marginnote}[]
\entry{Term A}{definition}
\entry{Term B}{definition}
\entry{Term C}{defintion}
\end{marginnote}

\begin{textbox}[h]\section{SIDEBARS}
Sidebar text goes here.
\subsection{Sidebar Second-Level Heading}
More text goes here.\subsubsection{Sidebar third-level heading}
Text goes here.\end{textbox}



\subsection{Equations}
% Example of a single-line equation
\begin{equation}
a = b \ {\rm ((Single\ Equation\ Numbered))}
\end{equation}
%Example of multiple-line equation
Equations can also be multiple lines as shown in Equations 2 and 3.
\begin{eqnarray}
c = 0 \ {\rm ((Multiple\  Lines, \ Numbered))}\\
ac = 0 \ {\rm ((Multiple \ Lines, \ Numbered))}
\end{eqnarray}

%% TOTAL WORD COUNT AT START OF PRUNING: 14289.

%% perl texcount.pl -inc AnnRev_Main.tex
%% File(s) total: AnnRev_Main.tex
%% Words in text: 14287
%% Words in headers: 166
%% Words outside text (captions, etc.): 568
%% Number of headers: 36
%% Number of floats/tables/figures: 6
%% Number of math inlines: 47
%% Number of math displayed: 0
%% Files: 8
%% Subcounts:
%%   text+headers+captions (#headers/#floats/#inlines/#displayed)
%%   247+29+0 (9/0/0/0) File: AnnRev_Main.tex
%%   369+0+0 (0/0/0/0) Included file: ./01_Introduction.tex
%%   4972+33+82 (8/1/11/0) Included file: ./02_Reactions.tex
%%   5843+61+297 (10/3/25/0) Included file: ./03_ExperimentalResults.tex
%%   1498+31+189 (7/2/8/0) Included file: ./04_Extrapolation.tex
%%   676+0+0 (0/0/0/0) Included file: ./05_FutureEvolution.tex
%%   335+4+0 (1/0/0/0) Included file: ./METSidebar.tex
%%   347+8+0 (1/0/3/0) Included file: ./TriggerSidebar.tex


%Goal: set the stage: why collider searches are needed and interesting. 

%- we know there is DM [review] and it is one of the best motivated observations of BSM physics
%-- relic
%- there are other searches and they haven't found anything yet, or perhaps they have (hints of signals), but nothing conclusive
%- at the LHC we have a tool that has been used to make discoveries of all SM particles so far
%- this article is about what we can learn, keeping in mind that the main running version of that tool we have is the LHC


Dark matter (DM) is perhaps the most persuasive experimental evidence for physics beyond the Standard Model of particle physics~\cite{Bertone:2016nfn}. 
If DM is indeed a particle~\cite{Steigman:1979kw}, it has gravitational interactions with normal matter.
It may have other, non-gravitational interactions, but these are relatively rare---dark matter is dark and has no electromagnetic charge.
It is also stable, or at least it decays with a lifetime comparable to that of the universe.
Finally, there is {\it a lot} of DM, about five times the standard matter (SM) described by the Standard Model of particle physics, with plenty of room to exceed the SM in its complexity.
The current abundance of dark matter in the universe, derived from measurements of the Cosmic Microwave Background~\cite{Ade:2015xua}, is one of the few quantitative measures of DM, or indeed of any physics beyond the Standard Model (BSM). The main consequences of cosmological observations for particle dark matter are listed in the Sidebar. 

%What we are missing

%A description of MET
%A simplified diagram of an s-channel process at colliders satisfying $Z_2$ symmetry is shown in panel (b) of Fig.~\ref{fig:monoX}.
%the following sentence: 
%The observation of a signal of visible or invisible particles at an LHC experiment that could be identified as being generated by one of the reactions described in this review cannot lead to claims that DM has been discovered. This is not a reason to discount searches for DM at the LHC, as such a signal would still be a groundbreaking discovery, regardless of its interpretation. Instead, we highlight the importance of the comparison of LHC results, where DM would be produced in the lab, with the results of complementary experiments that look for signals of DM coming from space. This comparison can only take place if the same theoretical model is used to interpret both results. This motivates the enumeration of possible models in this chapter. 

%Sidebar (50 words minimum, 200 words maximum) briefly discussing a fascinating adjacent topic; 
%insert below Literature Cited section, but indicate near which section in text the sidebar should be typeset
\begin{textbox}[!h]
\section{Particle properties of Dark Matter}

%Observations from astrophysics can inform experiments on DM particle targets, and whether newly discovered particles can be identified as DM. 
%We list here the most relevant consequences of these observation for the benchmark models used for collider DM searches described in this review. 

\textbf{Stability}
If DM is a particle, it does not seem to decay.
Conservation laws, such as R-parity in Supersymmetry (SUSY) or a $Z_2$ symmetry, can prevent the DM particle from decaying into any lighter even-parity SM particle.
Additionally, pairs of DM particles can be produced by the decay of other particles, charged under the same gauge group as the SM, or singly in the case the parent is a color triplet. 
%We also note that, while DM is stable on a cosmological scale, collider experiments are limited to the observation of particles with a lifetime that is longer than the time needed to escape the detector. 
%For this reason, we use the term "invisible particles" in collider reactions, rather than the term DM particles. 
%(i.e. DM candidate particles could still decay into other particles outside the detector and leave a signal of missing transverse momentum).

\textbf{Darkness} 
DM particles are effectively invisible to traditional collider experiments made of ordinary matter. However, the rest of the event is not. 
Invisible particles can be accompanied by one or more visible recoiling particles, leading to missing momentum in the transverse plane, whose magnitude is termed \MET. This is one of the main signatures of DM at colliders.

%\textbf{Observability of DM}
%Even though models of particle dark matter include SM couplings to satisfy cosmological observations under certain assumptions, these couplings need to be weak.
%~\footnote{We note that the only SM particle that satisfies this requirement of being sufficiently weakly interacting is the neutrino. However, neutrinos cannot make up the totality of DM as they are relativistic particles and cannot explain the galaxy structures that formed in the  universe~\cite{PlehnLecturesDM}.}.
%assumption of thermal freeze-out 
%These couplings determine the reach of collider searches, as they drive the production of new particles and their observability in the detectors. 
%so what
%if the couplings to the (constituents of) the colliding particles are enough to directly . Exceptions exist in the case of models where the SM and the DM interact only through a dark portal.
\end{textbox}


The field of particle physics is increasingly keen to understand what Dark Matter (DM) is, if it is indeed a particle. 
Some experiments, termed Direct Detection (DD) experiments, look for galactic DM colliding with underground targets made of Standard Model matter (SM)~\cite{0954-3899-43-1-013001}.
Others, termed Indirect Detection (ID) experiments, search for the products of annihilating dark matter concentrated within the gravitational potential wells of the Milky Way and elsewhere~\cite{Gaskins:2016cha}.
None of these experiments has yet found conclusive evidence of DM.
If the only interaction between DM and SM matter is gravitational, experiments will never see it.
Yet the search for particle DM started relatively recently, and plenty of room for optimism remains.

Colliders, one of the most successful tools of particle physics, have revealed much about SM matter.
This review will sketch how colliders contribute to the search for DM, focusing on the highest-energy collider currently in operation, the Large Hadron Collider at CERN.
Absent hints for the character of DM-SM interactions, it emphasizes what could be observed in the near future, the main experimental challenges presented, and how collider searches fit into the broader field.
Finally, it underlines a few areas to watch for the future LHC program.

%Define Standard Model (SM), Dark Matter (DM). There is also an "acronym" section we could use. 
%Need to decide whether Dark Matter or dark matter.

%Most recent reviews of DD and ID (in absence of anything better?)
%\cite{DMDD_NaturePhysics}
%\cite{DMID_NaturePhysics}
%I prefer these ones also because they have arXivs but they are older , . There is also a VERY old AR: \cite{doi:10.1146/annurev.nucl.54.070103.181244}


%Throughout this review, we will consider the relic density as a rough order-of-magnitude guide for our selection of models and results, rather than as an exact constraint. Most of the models considered that satisfy the relic density only consider a single particle. The DM sector may be much more complex than a single particle with a limited number of interactions. Nevertheless, if these simple examples dominate over others

%these simple examples may emerge in the searches at the early stages of the LHC  particles  
%Make point that we see simple things first (why simplified models is a good assumption). 

%make this into a sidebar




%Heading 1
\section{REACTIONS FOR INVISIBLE PARTICLE SEARCHES AT THE LHC}
\label{sec:02_Reactions}
%Throughout this review, we will consider the relic density as a rough order-of-magnitude guide for our selection of models and results, rather than as an exact constraint. Most of the models considered that satisfy the relic density only consider a single particle. The DM sector may be much more complex than a single particle with a limited number of interaactions. Nevertheless, if these simple examples dominate over others.  

%these simple examples may emerge in the searches at the early stages of the LHC  particles  

In this chapter, we will link the observations on DM to its particle properties. We then enumerate the possible reactions of DM at the LHC within certain grounding assumptions, building from simple to more complex models in terms of particle content. 

\subsection{Observations on DM as a guide for its particle properties}
\label{sec:DMObservations}

The observations mentioned in Section~\ref{sec:intro} require the dark matter particle to be stable on a cosmological timescale. This has important consequences for the prediction and observation of dark matter reactions at colliders. 

Firstly, a simple theoretical way to stabilize DM is the introduction
of a global $Z_2$ symmetry, as in Ref.~\cite{Batell:2010bp}. A realization of this
symmetry can be found in R-parity in the MSSM. %citation?
Under this symmetry, the parity of the DM particle is odd, while the parity of SM particles is even. 
$Z_2$-parity is multiplicative and conserved: this 
%is Z_2 parity a thing? I don't want to have it confused with the global SM parity
implies that an odd-parity DM particle (charge -1) cannot decay into any 
lighter even-parity SM particles (charge +1) and it is therefore stable. 
Additionally, DM particles will be produced in pairs from the decay of other particles
that are charged under the same gauge group as the SM. 

A simplified diagram of an s-channel process at colliders satisfying $Z_2$ symmetry is shown in Fig.~\ref{fig:sChannel}. If the particle mediating the SM-DM interaction is a SM particle, no additional particles beyond the DM need to be invoked, leading to the simplest DM production mode at the LHC. The only theoretically viable SM portal particles within the grounding assumptions of this revew are the Z and the Higgs bosons, described in Section~\ref{sec:HZPortalModels}. 

%TODO: add sidebar figure of s-channel. 
%this will become useful when we talk about s-channel mediators. maybe also make a point
%for the t-channel mediator?

Secondly, dark matter particles are invisible to detectors. 
However, the rest of the event is not: one can observe DM particles
produced in the event and escaping the detector 
due to their missing momentum in the transverse plane, if they recoil against one or 
more visible SM particles. 

%I don't like how this is linking up. 

%shared context: many possible new physics searches at the LHC
%problem: can't do them all
%solution: strong theoretical motivation, as well as observability
%exposition: particular case of DM

Collider experiments have a nearly unlimited choice of theoretically
motivated DM targets to search for. 
Theoretical arguments alone are not sufficient for a DM model to be tested at the LHC: 
couplings to SM particles need to feature in the model and be sufficiently large
to produce new particles and observe their signatures in the detectors. 

%everyone thinks of WIMPs, how strong is strong, how weak is weak? quantitative question of coupling, depends on model. in the introduction: need to talk about DM properties. Weak enough that there is no visible EM signal (no light emission or absorption). Relate those properties to what the particle physics properties need to be. Have a model in mind: s-channel mediator between DM and SM, weakness of interaction comes from particle being heavy or coupling being small. DMF models have order=1 couplings. 

Models of particle dark matter include SM couplings to satisfy cosmological observations in the freeze-out case. These couplings need to be weak enough that there is no visible signal of DM particles, as there is no evidence for DM interacting strongly with baryonic matter, nor for its emission or absorption of light. A typical DM-SM coupling satisfying relic density is of the order of XXX. 

The only SM particle that satisfies the requirement of being sufficiently weakly interacting is the neutrino. However, neutrinos cannot make up the totality of DM as they are not sufficiently massive to explain the galaxy structures that formed in the universe. %%CITE FENG AR, BERTONE'S BOOK
The upper bound on the neutrino content of DM is YYY. 

Unlike previous accelerators that either yielded large datasets (e.g. B-factories) or high center-of-mass energy (e.g. Tevatron), the LHC gives unprecedented access to both rare processes and high scale processes at the same time, planning to collect 3/ab by 2035 reaching the design center-of-mass energy of 14 TeV. For this reason, it is worth speculating whether the portal particles could be observed at the LHC for the first time. Models that include one or more very massive new particles beyond the SM in addition to the DM particle are also an LHC search target, and are described in Section~\ref{sec:BSMMediatorModels}. 

Portal models and models of simple BSM mediation are only motivated by the observation of DM. They keep the SM and the DM sectors separate, and make no claim to being a solution of other shortfalls of the SM. However, the coincidence that hierarchy problem, gauge coupling unification and DM particle nature could be solved with a single theory with observable consequences at the electroweak scale, has been one of the driving reasons to develop and consider SuperSymmetry (SUSY) as one of the main search targets for LHC searches. These models are discussed in Section~\ref{sec:SUSYModels}.

Finally, let us return on the concept of observability of the search target mentioned above. Even general purpose particle detectors may miss certain classes of phenomena, as the initial design choices privileged searches for the Higgs boson and for particles that generally decay promptly, as predicted by models discussed so far. However, there is tension when confronting data with portal models, BSM mediation models and supersymmetric models that are compatible with the standard freeze-out scenarios. This encourages us to look for other classes of models, especially those including particles with long lifetimes, as a way to shine the search lamppost beyond the classic WIMP scenario. Reaction including those particles and their connections to DM are sketched in Section~\ref{sec:LLPModels}

\section{Caveats and grounding assumptions}
\label{sec:GroundingAssumptions}

%%I suggest this part goes in the introduction, as it motivates enumeration of models in chapter 2 and comparisons in chapter 4. 
The observation of a signal of visible or invisible particles at an LHC experiment that could be identified as being generated by one of the reactions described in this chapter cannot lead to claim that DM has been discovered. This is because DM is stable on a cosmological scale, while LHC experiments are limited to the observation of particles with a lifetime that is longer than the time needed to escape the detector (i.e. DM candidate particles could still decay into other particles outside the detector and leave a signal of missing transverse energy). This is not a reason to discount searches for DM at the LHC, as such a signal would still be a groundbreaking discovery, regardless of its interpretation. This statement highlights the importance of the comparison of LHC results, where DM would be produced in the lab, with the results of complementary experiments that look for signals of DM coming from space. This comparison can only take place if the same theoretical model is used to interpret both results. This motivates the enumeration of possible models in this chapter. 

To define the scope of the reactions for invisible particles at colliders considered in this review, we make a number of grounding assumptions: 

\begin{enumerate}

\item We describe models where the DM particle interacts with SM particles, either directly or indirectly;
\item We restrict our list to models that include a $Z_2$ symmetry to stabilize DM;
\item We privilege models that respect Minimal Flavour Violation (MFV), which imposes that the flavor structure of couplings between DM and ordinary particles follows that of the SM.  %CITE?
%Citation from DMFs
%[49] J. Abdallah, H. Araujo, A. Arbey, A. Ashkenazi, A. Belyaev, et al., Simplified models for dark matter searches at the LHCSubmitted to Phys.Dark Univ. arXiv:1506.03116.
% [47] A. A. Petrov, W. Shepherd, Searching for dark matter at LHC with mono- Higgs production, Phys.Lett. B730 (2014) 178–183. arXiv:1311.1511, doi:10.1016/j.physletb.2014.01.051.
% [51] R. S. Chivukula, H. Georgi, Composite Technicolor Standard Model, Phys.Lett. B188 (1987) 99. doi:10.1016/0370-2693(87)90713-1.
% [52] L. Hall, L. Randall, Weak scale effective supersymmetry, Phys.Rev.Lett. 65 (1990) 2939–2942. doi:10.1103/PhysRevLett.65.2939.
% 2570 [53]
% A. Buras, P. Gambino, M. Gorbahn, S. Jager, L. Silvestrini, Universal unitarity triangle and physics beyond the Standard Model, Phys.Lett. B500 (2001) 161–167. arXiv:hep-ph/0007085, doi:10.1016/S0370-2693(01) 00061-2.
% 2575
% [54] G. D’Ambrosio, G. Giudice, G. Isidori, A. Strumia, Minimal Flavor Viola- tion: An effective field theory approach, Nucl.Phys. B645 (2002) 155–187. arXiv:hep-ph/0207036, doi:10.1016/S0550-3213(02)00836-2.
\item We primarily consider models where DM is a Dirac fermion, relying on existing theory material developed for early Run-2 searches. Other cases yield similar phenomenology for LHC searches, with some exceptions that we describe in this chapter. 
\item We privilege models that have a connection with thermal relic from freeze-out. There are other models from other cosmological histories (e.g. freeze-in) that can be considered and would lead to LHC phenomenology. 
%The Dawn of FIMP Dark Matter: A Review of Models and Constraints  - https://arxiv.org/pdf/1706.07442.pdf, Minimal Decaying Dark Matter and the LHC - https://arxiv.org/pdf/1305.6587.pdf
%
\end{enumerate}

\subsection{Higgs and Z boson portals}
\label{sec:HZPortalModels}

Even if we cannot observe DM itself at colliders, we can look for visible particles that are associated to Dark Matter. The LHC alone cannot solve the strong CP problem through observation of the axion, but it can still observe e.g. scalar resonances that appear in the theory. 

This raises the question of whether any of the SM particles could be associated to DM, for example in a similar fashion as the W and Z bosons mediate the weak interaction and produce neutrino pairs in the reaction. Models where the SM particle sector is coupled to the dark sector through an existing or a new particle are called \textit{portal models}. This kind of model leads to the most economical particle content for reactions at the LHC, as one only needs to add a neutral DM particle to the SM content if one of the SM particles is the portal particle. SM fermions cannot be portal particles under the assumption of a $Z_2$ symmetry, as they would allow the decay of DM. Photons, W bosons and gluons can't be portal particles either, as DM does not absorb nor emit light, nor it does it have electromagnetic or strong charge. The only viable SM portal particles remaining are the Z and the Higgs bosons. 

There are strong theoretical and experimental arguments to explore SM portal models at the LHC. 
%Theoretical
Processes involving mediators at the electroweak scale are among the first to be investigated, in DM theories that predict new weakly interacting particles. This kind of portals are also present in a number of other theories~\ref{Arcadi:2014lta}. 
%Experimental 
However, it is only the recent generations of collider and direct detection experiments that have started being able to probe the range of small couplings and relatively large scales required to observe this kind of models. 

The \textbf{Z portal} model, where the DM particle has vector and axial vector interactions with a Z boson, is a minimal extension of the SM as it only requires a single new particle to be added to the SM particle content. In $SU(2)_L x U(1)$ extensions of the SM, the axial and vector couplings of the Z boson to DM are generally required to be of the same order. If no other couplings are present, this model is not $SU(2)xU(1)$ invariant, unless couplings to the DM to the Higgs boson are added as well~\cite{Kahlhoefer:2015bea}. 
%The couplings between the Z and the DM can be vector, axial or mixed. 
In the minimal case where the couplings do not depend on the Lorentz structure of the interaction, 
%what I want to say: In general they are excluded in their minimal version because of their strong vectorial coupling necessary to respect relic abundance bounds.
large couplings are required for this model to satisfy the relic density. 
In the case of equal vector and axial couplings, this model is heavily constrained by LEP and direct detection experiments (see e.g. Refs.~\cite{Arcadi:2014lta,Escudero:2016gzx}). This model can still be viable wherever no relations between the vector and axial couplings are present. A review of Z portal models with different couplings can be found in Ref.~\cite{Arcadi:2014lta}. 
%Does one require a certain kind of couplings for symmetries? 
%Arcadi says: 
%In all these extensions, the axial coupling Aχ (see eq.(1)) of the Z boson to the dark matter is naturally of the order of magnitude of its vectorial coupling Vχ. The deep reason is that in a framework of SU(2)L × U(1) breaking the original SU(2)L condition (Vχ = Aχ) is only mildly modified by the dynamic of the breaking. 
%Maybe link this to the choices for the Z' model later on? 

The discovery of a SM-like Higgs boson~\cite{Aad:2012tfa,Chatrchyan:2012xdj} has sparked theoretical and experimental interest in \textbf{Higgs portal} models, where DM particles can interact with SM particles only through the Higgs boson (see e.g. Refs.~\cite{Patt:2006fw,Englert:2011yb,Djouadi:2011aa}). In Higgs portal models, DM couples to the SM operator connecting two Higgs fields and could dominate the interactions between SM and DM sectors.
%since the $H\dagger H$ operator has the lowest dimension in the SM. 
This interaction is renormalizable and leads to a UV-complete, minimal theory in the case of scalar and vector DM, while a self-consistent theory requires the presence of further particles mediating the interaction in the case of fermion DM~\cite{Freitas:2015hsa}. 

%From http://iopscience.iop.org/article/10.1088/1126-6708/2008/07/058/pdf
%Higgs-sector and Z′ interactions between the hidden sector
%and the SM states are special in that they involve gauge-invariant operators of dimension
%dO ≤ 4, and thus can be induced by physics at arbitrarily high scales with unsuppressed
%couplings. 

The properties of the Higgs boson are modified in the presence of decays to invisible particles. Precision measurements of the Higgs width and couplings offer a probe for these models complementary to direct searches for the invisible particles, as described in the next chapter. 
%This class of models is already constrained by electroweak precision measurements, but still viable if the DM mass is about half the Higgs mass. 

%We could have a picture of constraints here?

%Arcadi
%However, the last LUX results[11], combined with the invisi- ble width of the Higgs excluded the Higgs-portal scenario for dark matter mass below 200 GeV [2].

\subsection{Effective Field Theories and Simplified models of BSM mediators}
\label{sec:BSMMediatorModels}

Having completed the survey of the possible minimal DM models that only add a single new DM particle to the SM, we move to the next class of models, where the SM particle spectrum is complemented by the DM particle as well as by other BSM particles. In this case, the LHC search targets expand from the excess of missing transverse momentum to a wide variety of observable signatures from e.g. the decays of the new BSM particles. 

The huge body of theoretical literature on DM models featuring additional BSM particles drives the design of experimental searches in two complementary directions. In the case of self-consistent models of DM such as fully-developed SUSY models, all experimental handles are exploited for targeted searches that are sensitive to specific model features. These models will be described in the next section. However, the desire to make no assumptions on the DM phenomenology and to cast a net as wide as possible remains. The adoption of much simpler model as first LHC Run-2 DM benchmarks led to the design of more generic searches targeting the broad features of those models. The success of such simple, at times incomplete and not always theoretically sound models has been due to their ability to predict the key features and observables related to DM production at the LHC with only a limited number of new particles and theory parameters, factoring out the more complex processes that do not affect LHC phenomenology as they e.g. occur at higher energy scales. These simple models have generally been organized according to their interactions and observable consequences~\cite{Tait's papers, DMFReport}, used as building blocks for more complex theories in models of DM and elsewhere, see e.g ~\cite{Choudhury:2015lha,Gutschow:2012pw, Alwall:2008ag}, and employed for building a prioritized set of LHC search scenarios that is only loosely connected to specific theories of DM. 
Even in the case of simple models, this review lays grounding assumptions connected to what has been sought in the first LHC searches. In addition to the grounding assumptions discussed in Section~\ref{sec:GroundingAssumptions}, we restrict to models where the leading process is tree-level, leaving cases where the dominant contributions are of higher order for later study (see e.g. Ref.~\cite{Godbole:2015gma}). 

%\subsubsection{Effective Field Theories}

\textbf{Effective field theories} (EFTs)~\cite{Goodman:2010ku, Shoemaker:2011vi} are the simplest possible models of DM production at the LHC beyond those described in Section~\ref{sec:HZPortalModels}. A four-point interaction is used to describe the DM production at the LHC in a low-energy approximation of a full theory, similarly to what done when describing the weak interaction through a Fermi process before the introduction of the W and Z bosons~\cite{Fermi2008}. EFT operators were first widely employed to describe DM reactions at colliders at the Tevatron~\cite{Bai:2010hh,Beltran:2010ww}. They were found advantageous because of their model-independence, and since each of the operators encapsulated phenomenological characteristics of most known types of SM-DM interaction. 

The only parameter characterizing an EFT operator, in addition to the type of DM particle and to the type of SM-DM interaction, is the scale of the contact interaction $\Lambda$. In the case of a $s-$channel completion of the EFT, this interaction scale is proportional to the mass of the mediator particle. If the scale of the DM interaction is sufficiently low with respect to the mediator mass, the phenomenology is the same for the EFT as for its $s-$channel completion. 
This may not always be the case at the LHC, given the high center-of-mass energy collisions: a better description of both theory and phenomenology can be reached when explicitly including the new particles in the model considered~\cite{Buchmueller:2013dya,DeSimone:2016fbz,Berlin:2014cfa}. Certain EFT operators also may suffer from gauge invariance issues at the electroweak scale~\cite{Bell:2015sza}. 

\begin{marginnote}[]
If a completion of the EFT is not available, procedures describing how to truncate the events where the EFT description is not valid are available~\cite{Racco:2015dxa,Busoni:2014sya,Busoni:2013lha,Busoni:2014haa}. A recommendation on how to present EFT results from LHC searches can be found in~\cite{Abercrombie:2015wmb}.
\end{marginnote}

\textbf{Simplified models of BSM mediation} are a natural step beyond effective operators and still map well to the different types of interactions. 

Simplified models resolve the issue of whether a model-independent EFT description is valid at the LHC, even though they are not complete models themselves (e.g. not all the models used are gauge invariant~\cite{Kahlhoefer:2015bea}). Simplified models can be used for comparisons with non-collider DM searches within a clearly specified theory framework. Their use as benchmark models for Run-2 searches also highlights the strength of the LHC in searching for the visible decays of the mediator particle alongside its decays in DM particles, as detailed in the next chapter. 

In the simplified models chosen as benchmarks for the first LHC Run-2 searches, only one extra particle is added to the the DM and SM particle spectra. In general, this particle mediates the DM-SM interactions. If neutral, the mediator particle is singly-produced at the LHC, and decays in pairs of DM particles due to the Z2 symmetry as well as in pairs of SM particles. If the mediator is colored, it can lead to a $t-$channel exchange between an incoming LHC parton and the DM particle among other processes. Colored mediator phenomenology is akin to that of SUSY models with a squark exchange~\cite{}. 

For a comprehensive review of simplified models at the LHC, we refer to~\cite{Arcadi:2017kky}, while~\cite{Abercrombie:2015wmb} presents a prioritized list of simplified models that have been used in early LHC Run-2 searches following a joint experimental and theory effort called the Dark Matter Forum, building on the discussion of the communities detailed in Refs.~\cite{Yavin:14092893,Malik:2014ggr,Abdallah:2015ter}.  

$S-$channel resonances have been privileged 


\subsection{Supersymmetric models}
\label{sec:SUSYModels}

\subsection{Long-lived particle models}
\label{sec:LLPModels}

\section{EXPERIMENTAL RESULTS}
\label{sec:03_ExperimentalResults}
Now that we have a handle on the reactions of DM observable at collider experiments, we turn to a description of the searches and experimental constraints for DM at colliders, privileging LHC searches as they generally set the most stringent constraints. For a detailed description of the LHC and the ATLAS, CMS and LHCb experiments, we refer to~\cite{LHC2008,ATLAS2008,CMS2008}. %CD: blurb more? I'd say no. 
The first period of LHC running (2010-2012) at 7 and 8 TeV center-of-mass energy ($\sqrt{s}$) is termed Run-1, while the second period (2015-2018) is called Run-2. 
The categorization of these searches follows loosely the description of the benchmark models. We start describing searches for DM interacting through SM bosons~\ref{sec:results_ZHSearches}, then move to generic searches for signals with missing transverse momentum~\ref{sec:results_monoXSearches}, and outline the searches for complete models with DM candidates in Section~\ref{sec:results_SUSYSearches}. Throughout this chapter
%and in Section~\ref{sec:experimentalChallenges} %CD: removed in favour of sidebars
we will highlight the experimental challenges and the novel experimental techniques used to overcome them, motivated by the strong interest in dark matter searches. 
We then conclude with searches for long-lived particles within models of DM in
Section~\ref{sec:results_LLPSearches}. %CD: need to rewrite this sentence, but the idea is: if we hadn't had DM as a motivation motivation we wouldn't have done this difficult stuff. 

\subsection{Searches for DM in interactions mediated by SM-boson}
\label{sec:results_ZHSearches}

The invisible decays of the Z and Higgs boson are the main direct targets of searches for SM-boson-mediated interactions between SM and DM particles, if the DM particle is lighter than half the mass of the boson. Above this region, Direct Detection experiments are generally more sensitive than collider experiments. 
%CD: do we need to answer "what if not"? No one seems to care, but one could maybe think of using monojet off-shell (tiny tiny region) and precision constraints for the off-shell region too, a la dijet. Main point for the moment: DD covers this region so we don't have to. 
In the SM, the Z boson can decay to a neutrino-antineutrino pair, while the Higgs boson decays into a pair of Z bosons each decaying to neutrinos. Additional decays of the Z and Higgs boson to particles beyond the SM modify the properties of the vector boson, such as width and couplings. 

%MonoZ

%CD only mentioning below because it's like a monophoton
%Simple (but somehow messy) explanation in https://cds.cern.ch/record/1750933/files/CERN-THESIS-2013-330.pdf, Hugo's student
%Hugo did it, unpublished: https://www-cdf.fnal.gov/physics/ewk/2007/ZnunuWidth/
%CDF direct: 466 pm 42
\textbf{Decays of the Z boson into invisible particles} can be constrained using the invisible Z width. It can be measured directly in Z decays in association with a photon emitted as initial state radiation. Events are selected containing a single photon, missing transverse momentum and no other sizable event activity. This selection is also used for identifying events from possible DM reactions at colliders.
%LEP combined: 503 $\pm$ 16 MeV
The total Z width has been measured indirectly at LEP~\cite{ALEPH:2005ab} leading to a measurement of the number of light neutrino families compatible with cosmology; if the partial widths of the decays into visible particles are subtracted from the total width, the invisible width can be measured to 499.1 $\pm$ 1.5 MeV~\cite{Patrignani:2016xqp}. 
%The precision of the indirect measurement is better than that of the direct measurement, due to the higher statistics and the relative ease of selection and background subtraction for the visible Z decays. %CD: omitted, no space
%The main systematic uncertainty in this case comes from the theoretical uncertainties in the simulation. CD: Carena seems to think it is an uncertainty on fast simulation
New physics effects modify direct and/or indirect Z width~\cite{Carena:2003aj}.
\begin{marginnote}[]
Direct and indirect Z width measurements must agree if the decay of the Z to a pair of invisible new particles is to be the main mechanism responsible for the deviation from the SM values. 
\end{marginnote}%CD: I think this is important to mention in the same sense as the caveats on the s-channel resonances, but it can be omitted
%in this case, an analysis of the mass of the system recoiling against the photon would provide a handle to distinguish between different BSM processes. %CD: omitted because this is possibly too handwavy but how can we summarize 4 pages of Carena in a sentence?
%Carena quantitative: At present, measurements at LEP and CHARM II are capable of constraining the left-handed Z\nu\nu-coupling, 0.45 <~ g_L <~ 0.5,  while the right-handed one is only mildly bounded, |g_R| <= 0.2.
The LEP precision measurements~\footnote{Bounds on Z to invisible decays obtained from LHC searches are not yet competitive~\cite{deSimone:2014pda}.}, as well as direct detection experiments, rule out the majority of the Z-mediated DM scenarios~\cite{Arcadi:2014lta,Escudero:2016gzx}. The LEP invisible width is well below the width one would expect if vector and axial vector models of DM were realized, for all couplings satisfying the relic density with a DM mass below 25 GeV. Direct detection experiments such as Xenon1T~\cite{Aprile:2017iyp} 
%CD: take figure 2 of Escudero:2016gzx and compare with the results of Aprile:2017iyp
rule out most of the other simplified model scenarios compatible with freeze-out relic density up to multi-TeV DM masses. 
%DM mass above 6 TeV for the vector couplings, while for axial the plot is truncated. 

%%MonoH

\textbf{Invisible decays of the H boson} within the SM only contribute to less than 0.1\% of the total decay width. For this reason, an observation of even a small contribution to the Higgs width from invisible particles would signal the presence of new physics phenomena that could be linked to DM if 2\mdm $< m_H$~\footnote{For the case of heavier DM particles, see Ref.~\cite{Djouadi:2011aa}.}. 

Pre-LHC constraints on the invisible Higgs width are derived from measurements of the ZH production channel at LEP in searches for new neutral Higgs-like bosons, where only the visible decays of the Z are observed. This is a common procedure to select events in LHC DM searches. %CD: need a number! 
It is not feasible to directly or indirectly measure the total and partial Higgs widths at a hadron collider and then extract the invisible contribution as done for the Z at LEP, as some of the decays (e.g. gluons and lighter quarks) have too large a background to be measured, the experimental resolution even for leptonic decays is large compared to the intrinsic Higgs SM width, and the kinematics of the ZH process is not fully determined as in lepton colliders. %CD: this is ambiguous also because I am not sure I fully understand the first and third points completely - need AB's help, page 2 of Dobrescu/Lykken. 
%Lykken/Dobrescu, 1210.3342: Total theoretical SM width/mass for H125: 3.2 * 10^-5 MeV, due to small Yukawa of b quark and suppression of WW*. From rates and couplings,  can extract upper and lower limits on the exotic Higgs branching fractions, which come from the upper/lower limit on the total width. This paper ignores exp uncertainties. 
%Wagner Dark Side of the Higgs boson: omit because we don't care about non-SM Higgses
%The width can be extracted from the lineshape in the low-background channel $Z \rightarrow ZZ \rightarrow 4l$, assuming a SM width. This is what CMS has done. 
%If one does not want to assume the SM width, one can still extract the width
%above 190 GeV where the experimental resolution is better. 
%what we want to see is a larger total width with less normalization because of the invisible decays
Instead, searches at the LHC either attempt to directly observe the invisible decays of the Higgs boson, or compare measurements with precise theoretical calculations of SM parameters, to reveal
%unearth? there are no worms in this article so we should change this word
discrepancies signaling new physics or indirectly place constraints on new physics phenomena. 
Higgs to invisible LHC searches using Run-1 and Run-2 data~\cite{Khachatryan:2016whc,Aad:2015pla} employ and combine the $qq \rightarrow H qq$, $qq \rightarrow VH$, $gg \rightarrow HZ$, $gg \rightarrow Hg$ Higgs production modes. 
%CD: This section is just CMS for now. Need to add ATLAS, but also cut as we're using too much space for this:
%https://atlas.web.cern.ch/Atlas/GROUPS/PHYSICS/PAPERS/HIGG-2015-03/fig_09.png
In all cases, in addition to a requirement of sizable missing transverse momentum, auxiliary visible objects are used to select the events. 
%Loop-induced signals are important. 
%For more info on importance and calculation of loops: 1605.08039, but we run out of citations and space
The events are divided in exclusive categories targeting specific production modes. The associated boson (VH) searches target the decays of Z bosons to electrons, muons light or heavy flavour quarks, while the W bosons can decay into light-flavour jets. 
The $qq \rightarrow H qq$ production mode is dominated by Vector Boson Fusion (VBF) processes, where the Higgs boson is produced in association with two hadronic jets that have a large pseudorapidity ($\eta$)
%CD: assume eta?
separation in the detector, and a large invariant mass. This topology is used to select events and discriminate between signal and background. 
%The large QCD backgrounds are suppressed by requiring that the missing transverse momentum recoils against the jets in the event. 
%If the missing transverse momentum was in the direction of the jets, there would be a chance of it coming from mismeasured jets. %CD: hmmm written this in a rush
The jet+MET search, described in more detail in the next section, is reinterpreted for the $gg \rightarrow Hg$ mode. 

%The upper limit on the invisible BR from Higgs decays is 25%. 
%ATLAS Abstract
%Direct searches for invisible Higgs boson decays in the vector-boson fusion and associated production of a Higgs boson with W/Z (Z ? ??, W/Z ? jj) modes are statistically combined to set an upper limit on the Higgs boson invisible branching ratio of 0.25. The use of the measured visible decay rates in a more general coupling fit improves the upper limit to 0.23, constraining a Higgs portal model of dark matter.
%%Precision
Precision measurements of the Higgs boson properties and the comparison with SM theory also play a role in constraining the possible contributions to new physics, as decays into invisible particles would reduce the SM Higgs production and decay coupling strengths~\cite{Khachatryan:2016vau,Englert:2011yb,Aad:2015pla}. 
%For the Higgs boson, the upper limit on the branching fraction to visible and/or invisible non-SM particles only using precision measurements is 34\%
%In case we want to say what limits these
%The main limitation for the measurement of the invisible width of the Higgs at the LHC is due to QCD uncertainties the Higgs production cross-section, which limits the sensitivity of these searches to roughly 10\% of the SM value. 

The most stringent observed upper limit on the fraction of invisible decays of the Higgs boson, combining direct and precision measurements is 23\%.
%What does it mean for Higgs portal models: DD is always better
In the case of light fermion DM with scalar couplings to the Higgs, direct detection experiment rule out most of the parameter space where the model can provide the measured relic density~\cite{Escudero:2016gzx,Djouadi:2011aa}. Due to the suppression of the cross-section for DD in the pseudoscalar case, the model is still not constrained around a small region for DM masses corresponding to half the Higgs mass and above. %CD: maybe we have to say why this is the case - essentially rates are too small, see paper by Plehn. 

\subsection{Generic searches for DM with missing transverse momentum}
\label{sec:results_monoXSearches}

%Sidebar (50 words minimum, 200 words maximum) briefly discussing a fascinating adjacent topic; insert below Literature Cited section, but indicate near which section in text the sidebar should be typeset
\begin{textbox}[!h]
\section{Details of \MET reconstruction and fake \MET rejection}
This is why this is important, this is 10 words. 
This is why this is important, this is 10 words. 
This is why this is important, this is 10 words. 
This is why this is important, this is 10 words. 
This is why this is important, this is 10 words. 
This is why this is important, this is 10 words. 
This is why this is important, this is 10 words. 
This is why this is important, this is 10 words. 
This is why this is important, this is 10 words. 
This is why this is important, this is 10 words. 
This is why this is important, this is 10 words. 
This is why this is important, this is 10 words. 
This is why this is important, this is 10 words. 
This is why this is important, this is 10 words. 
This is why this is important, this is 10 words. 
This is why this is important, this is 10 words. 
This is why this is important, this is 10 words. 
This is why this is important, this is 10 words. 
This is why this is important, this is 10 words. 
This is why this is important, this is 10 words. 
This is why this is important, this is 10 words. 
This is why this is important, this is 10 words. 

%\subsubsection{Missing transverse momentum}
%\label{sub:MET} 

%Main points:
%\begin{itemize}
%\item The measurement of \MET relies on the precise measurement of all reconstructed physics objects. 
%\item Some description of \MET significance may be needed, but it may also be too academic. 
%\item Fake \MET is rejected using quality cuts.  
%\item Pile-up needs specific techniques because of the soft terms. 
%\item \MET at the trigger level is the driving reason why we can't go lower, see next section.
%\end{itemize}

%from ooutline

%- Mismeasured MET (combining instrumental effects and beam/cosmics background)				
%	- CDF				
%		- beam background: exploit track pointing to jet and calorimeter layers				
%		- QCD: shitty method from Mario (extrapolation changing the veto)				
%	- LHC:				
%		- beam backgrounds: like CDF, more refined				
%			- can have a % of how many events would have been				
%		- QCD: matrix method a la SUSY				
%	- Other backgrounds (diboson, top)				
%		- Small so using MC				
%		- LHC has validation regions				
%			- check ttbar				

%Valerio's talk for relevant plots 
%https://indico.cern.ch/event/466934/contributions/2590281/attachments/1489278/2314178/20170706_EPS_DMatATLAS.pdf

%MET significance: in VBF CMS search
%For the 8 TeV dataset, an additional requirement is set on an approximate missing transverse energy significance variable S(Emiss) defined as the ratio of Emiss to the square root of the scalar sum of the transverse energy of all PF objects in the event [62]. Selected events are required to satisfy S(Emiss) > 4?GeV.


\end{textbox}

WIMP DM particles at colliders escape detection, and their observation requires one or more visible objects in the same event. Searches that only rely on this feature are for the most part model-agnostic, as they only need to detect an excess of missing transverse momentum left by the DM particles recoiling against SM objects, without making any extra assumption on the DM particles or on their production mechanism. Similar search strategies have been employed as center-of-mass energy and dataset size increased, from LEP to Tevatron to the most recent LHC searches~\cite{Fox:2011fx,Beltran:2010ww,Bai:2010hh}. A generic event selection for excesses of \MET also provides an inclusive sample for more targeted searches as it will be discussed in Chapter~\ref{sec:05_Future}. 

We begin this section by describing the LHC searches for missing transverse momentum in association with one or more hadronic jets. The jet+\MET search allows us to illustrate many of the techniques used in invisible particle searches, and it is one of the most powerful to constrain BSM-mediated simplified models of dark matter. We then move on to outlining searches using different associated objects, and continue with searches for visible mediators that are the consequences of the DM production mechanism. Finally, we compare and discuss the sensitivity of invisible DM and visible mediator searches at the LHC. 

\subsubsection{Searches with jets}
%monojet

%%CD: I am not sure I would want to read this summary of monojet search. But maybe I'm just jaded. Anyway, if we can, we should make it more interesting / give it a slightly different spin than just a plain description. 

%%Intro and event selection

Events containing invisible particles can be identified and selected at colliders if initial state radiation (ISR) is present. For $e^+e^-$ colliders, the most frequently radiated object is a photon, while for hadron colliders gluon radiation dominates. These searches have been called "Mono-X", where X is the radiated object, although the radiation of a single object is only the leading process in a SM-DM $s-$channel interaction~\cite{Haisch:2013ata}. For this reason, the most recent LHC searches for MET with jets~\cite{Sirunyan:2017jix,Aaboud:2017phn} allow for events containing more than one jet in the final state. Since the presence of highly energetic invisible particles would manifest as an excess of events with a significant \MET, the main observable for this search is the number of events in \MET \textit{signal regions}, either exclusive (in bins of \MET) or inclusive (considering all events above a given \MET threshold). 
The discovery of a signal originating from one of the benchmarks DM models presents different challenges, depending on the DM particle mass and boost. If the mediator is heavy, any light DM particle will receive a boost and appear as an excess in the tails of the SM \MET distribution. If instead the DM particle pair originates from a light mediator, of the same mass range as the Higgs boson, it will manifest itself at low \MET. The low \MET suffers from a much higher rate of both instrumental and SM backgrounds. As a consequence, it is impossible to record and store all events with a low-\MET for further analysis, since at the data-taking stage (within the \textit{trigger} and data acquisition systems) it is difficult to obtain further handles to discriminate signal and background, and the sensitivity to low-\MET signals is compromised. This challenge will be discussed further for both visible and invisible particle searches in Sec.~\ref{sub:twoBody}.

\begin{marginnote}[]
\entry{Trigger}{a detector system that decides which LHC collision events are to be recorded for physics analysis. For a description of the trigger systems of the ATLAS and CMS experiments, see ~\cite{Aaboud:2016leb,Khachatryan:2016bia}.}
\end{marginnote}

%and  QCD subprocesses matter - too much detail too little space https://cds.cern.ch/record/159861/files/198507018.pdf

%in OOutline we wanted to quote example numbers, but there is a lot of eyeballing
%>1000 GeV: EM10: 226+/-16 events predicted, 245 observed
%WIMP mdm 400, mmed 1000: 0.2 (eyeballed)*200 GeV  

%up to 3 jets with pT>30 GeV for ATLAS
Events are selected to enter the \MET signal regions if they contain at least one jet in the central region of the detector ($\eta<2.4$) with \pt $>$ 250 GeV (ATLAS) or \pt $>$ 100 GeV (CMS) and \MET $>$ 250 GeV (ATLAS) or \MET $>$ 200 GeV (CMS). This selection ensures that all events with these characteristics are recorded by the trigger system for further analysis. A lepton veto is used to suppress background from leptonically decaying W bosons. 
%CMS excludes taus, while ATLAS does not. Too much detail imo. 
QCD background where large \MET originates from mismeasured jets is rejected by requiring that the $\phi$ direction of the missing transverse momentum vector does not align with the direction of the four-momentum of the jets with the highest \pt (leading jets). The remaining QCD background estimated from data amounts to a maximum of 0.4\% of the total background. The large number of events containing fake \MET due to non-collision background (e.g. cosmic rays, beam-gas interactions, calorimeter problems), shown in Fig.~\ref{fig:fakeMET} is rejected with specific quality criteria discussed in the relative Sidebar.%Sidebar? Textbox?

\begin{figure}[!htpb]
\includegraphics[width=0.5\textwidth]{figures/FakeMETTemp.png}
%Caption from the ATLAS monojet but they only have pT
%https://atlas.web.cern.ch/Atlas/GROUPS/PHYSICS/PAPERS/EXOT-2015-03/
\caption{The \MET distribution of the data events passing the [define selection] without any cleaning criteria applied on the leading jet. The Standard Model background indicated in the plots corresponds to the estimates obtained for the analysis signal region, including jet quality requirements. 
%The jet selection inefficiency of the cleaning selection is O(1\%), which is negligible compared to the observed excess in data. 
This demonstrates the necessity of a strong non-collision background suppression for this analysis. From~\cite{ToBeFound}.}
\label{fig:fakeMET}
\end{figure}

%CMS: up to 4 leading jets 
The CMS analysis also applies specific vetoes for photons and heavy flavour jets, to reject events with photon ISR or containing top quarks. The CMS analysis also includes a signal region targeting hadronic decays of the W and Z bosons using substructure techniques, which is considered separately in the case of ATLAS and will be discussed in Sec.~\ref{subsub:monoV}. 

\begin{marginnote}[]
\entry{Signal region}{a region of phase space for a search that is enriched in signal events. Event counts in this region are used to compare background-only prediction to data in search for discrepancies signaling new physics. 
%For example, a region with a high \MET and no other objects except for the ISR object is a signal region for a \MET+X search.
}
\entry{Control region}{a region of phase space for a search that is signal-free but with characteristics otherwise as close as possible to the signal region. Event counts in this region are used to estimate backgrounds in the signal region. 
%For example, a region requiring no \MET and a process mimicking that of the signal region is a control region for a \MET+X search.
}
\end{marginnote}


%%Backgrounds
The main background contributions that remain after the event selection are
invisible decays of the Z boson into neutrinos (approximately 55-70\% of the total background) 
%numbers in Livia's talk: https://indico.cern.ch/event/682235/contributions/2817876/attachments/1576792/2490208/DMWG-2017_V2.pdf and in Francesca's talk:
%https://indico.cern.ch/event/682235/contributions/2817877/attachments/1576793/2490236/171218_atlas_ungaro.pdf
and leptonic decays of the W boson where the lepton is not
reconstructed (approximately 20-35\% of the total background), in association with jets. 
%
In order to reduce theoretical and experimental uncertainties on the main V+jet backgrounds, the number of events in the signal region from each of these backgrounds are derived from data in signal-free \textit{control regions} selecting V+jet processes where the W and Z bosons decay into visible particles ($Z\rightarrow ll, W\rightarrow l\nu+jets$, where $l$ = $e, \mu$). 
%not sure we should say that the bin by bin estimates are used, here it's ambiguous
The event selection follows that of the signal region, substituting a lepton requirement to the lepton veto. The visible decay products in events selected for the control regions are subtracted from the total transverse momentum balance, providing an estimate of the contribution of these backgrounds in the signal region. CMS also uses a $\gamma$+jet control region where the photon is subtracted following the same procedure, to increase the statistical precision of the background estimate. 
%
The distribution of events in the main observable used for the search, the shape of the \MET distribution, is simulated and reweighted for each of the control regions using the most recent perturbative calculations for NLO QCD and QED~\cite{Lindert:2017olm}.  The estimation of the number of $Z\rightarrow \nu\nu$events from the $\gamma$+jet and  $W\rightarrow l\nu$+jets control region needs a specific treatment due to the difference in the processes. This is particularly important for a consistent treatment of the different processes used in the background estimation and of the main theoretical uncertainties, and it will be discussed further in the following textbox. %CD: how do we call the thing
The full information on the theoretical and experimental uncertainties and their correlations from this procedure is used in a simultaneous fit to control and signal regions, to determine the overall background estimate in each of the \MET regions considered. 
%and leads to an improvement of 40 to 50% in the search according to PhilHarris and Livia, but I wouldn't add that unreferenced
Backgrounds from top processes in ATLAS are estimated using a dedicated control region with a requirement of a $b-$jet that is included in the fit, while CMS takes this background from simulation. Smaller diboson backgrounds are estimated from simulation. 

%Sidebar (50 words minimum, 200 words maximum) briefly discussing a fascinating adjacent topic; insert below Literature Cited section, but indicate near which section in text the sidebar should be typeset
\begin{textbox}[!h]
\section{Precision estimation of background for \MET+X searches}
In order to relate the number of events in the jet+\MET
signal regions (where $Z\rightarrow \nu\nu$ dominates) and control
regions (where events with jets produced in association with
$Z\rightarrow ll$, $W\rightarrow l\nu$ and $\gamma$ are used to maximise
the statistical power of the background estimation),
one needs to rely on a precise theory  
prediction of the ratio of the V+jets cross-sections. 
This is why this is important, this is 10 words. 
This is why this is important, this is 10 words. 
This is why this is important, this is 10 words. 
This is why this is important, this is 10 words. 
This is why this is important, this is 10 words. 
This is why this is important, this is 10 words. 
This is why this is important, this is 10 words. 
This is why this is important, this is 10 words. 
This is why this is important, this is 10 words. 
This is why this is important, this is 10 words. 
%\subsubsection{Precise background estimation}
%\label{sub:precision}

%from ooutline
%- LHC results				
%	citations for most recent
%	- Differences between ATLAS and CMS:				
%		- CMS starts with only Z, ATLAS uses Z and W, CMS uses everything including gamma				
%		- theoretical issues with using Z and W				
%		- Pozzorini paper: shit is complicated, W and Z are one thing but if you want to do photon it's a different story				
%			- dependence of result on analysis cuts				
%			- QCD correction				
%			- EW corrections				
\end{textbox}

%%Experimental uncertainties
The systematic uncertainties on the background estimate for the jet+MET search range from 2 to 7\% (CMS) and 2 to 10\% (ATLAS), depending on the \MET region. The main uncertainties are due to the identification of leptons (CMS) and the understanding of the jet and \MET calibration (ATLAS). 

%%What it means for the models we talked about
Since no significant excess is found in any of the signal regions, limits are set on the parameter space of Higgs portal models described in Sec.~\ref{sec:HZPortalModels} and simplified models described in Sec.~\ref{sec:BSMMediatorModels}, namely where the SM-DM interaction is mediated by $s-$channel vector (V), axial vector (AV), scalar (S) and pseudoscalar (P) and colored scalar mediators.  
%CD: Maybe we put this in the reactions chapter? It is quite an important statement
Since the simulation of the entire parameter space for these models by the experiments is computationally intensive, the Dark Matter Forum had agreed on a limited set of benchmark parameters to be tested~\cite{Abercrombie:2015wmb}, privileging those that change the LHC kinematics of the search (e.g. give a harder \MET spectrum) rather than those that only change the cross-section of the process and can therefore be reinterpreted from the search results. For example, the kinematics and cross-section of the vector and axial vector mediators is very similar at the LHC, while the DD and ID cross-sections change. 
The parameter values used as benchmarks (e.g. couplings) have been selected considering the sensitivity of early Run-2 searches, precision constraints and general simplicity arguments. As described more in detail in Section~\ref{sub:comparisonVisibleInvisible}, results are given in the \mdm, \mmed  plane fixing the couplings to \gq=0.25 and \gdm=1.0 for vector and axial vector mediated models, \gq=\gdm=1.0 for scalar and pseudoscalar models and \gdmq=1.0 for colored scalar models. The simplified models employed by the experimental collaborations are known at NLO~\cite{Neubert:2015fka,Haisch:2013ata,Backovic:2015soa}. 

%monoH come from CMS but maybe ATLAS has a better reinterpretation. 
The most stringent 95\% C.L. observed (expected) upper limits on the invisible branching fraction from jet+\MET searches are 53\% (40\%).  TODO look for ATLAS?
%(CMS, combining jet and vector boson radiation categories). 
%V/AV come from CMS search, ATLAS is less sensitive as it's 1.55 TeV
Vector and axial vector mediators are excluded by LHC searches at values of \mdm up to 700 and 400 GeV respectively with \mmed up to 1.8 TeV. This choice of model and couplings produces a relic density that is lower than the Planck measurement and it is still unconstrained by LHC searches for \mdm$>$0.3 TeV at \mdm$=$1.8 TeV for the vector mediator, and for 0.65$<$\mdm$<$0.75 TeV at \mdm=1.8 TeV for the axial vector mediator\footnote{Here and in the following, we quote observed limits at 95\% C.L. and refer to the bibliography for expected limits and 90\% C.L. limits.}. 
%CD: Not sure this is interesting for anyone? A bit complicated to project a 2D plot in words
%pseudoscalar comes from CMS
The LHC limit on the pseudoscalar mediator mass is lower due to the Yukawa-like couplings suppressing the cross-section with respect to spin-1 mediators, and it is 0.4 TeV in the CMS search for \mdm up to roughly 150 GeV. 
Jet+\MET searches are not yet sensitive to scalar mediators with the chosen couplings. 
%t-channel comes from ATLAS
%CMS
%Colored scalar mediators with masses up to 1.4 TeV at values of \mdm = 60 GeV are excluded.
%ATLAS
%CD TODO: check what parameters are people using?
Colored scalar mediators with masses up to 1.7 TeV at values of \mdm = 10 GeV are excluded for \gdmq=1 and \mdm=100 GeV. Considering this exclusion limit, this model still provides a viable DM relic density for \mmed \mdm above roughly 500 GeV at \mmed=1.7 TeV\footnote{The ATLAS and CMS results do not use the same parameters, here we report the ATLAS result.}.

Other benchmark scenarios such as compressed SUSY scenarios, 
%maybe explain?, 
squark pair production, 
%who ordered that
non-thermal singly-produced DM, 
and Large Extra Dimensions (ADD) are also constrained by the ATLAS and CMS searches, in some cases providing the most stringent constraints to date. 

%%Reinterpretation - this is the only thing that I think fits well here
The jet+\MET search can constrain a wide variety of reactions for invisible particles. Therefore, various approaches have been taken to allow model-builders and phenomenologists to easily reinterpret its results. As for most LHC searches, the published experimental data from the ATLAS and CMS collaborations is provided on the HEPData platform~\cite{Maguire:2017ypu}. Additionally, a  simplified likelihood function~\cite{Collaboration:2242860}, which under certain assumptions approximates the full likelihood using a reduced set of information, is provided for the CMS result~\cite{Sirunyan:2017jix} and has been used for reinterpretation~\cite{Pobbe:2017wrj}. Moreover, the ATLAS Collaboration has used the ratio of cross sections of events containing a jet and \MET and events containing a jet produced in association with an opposite-sign same-flavour dilepton pair from the decay of a Z/$\gamma*$ boson~\cite{Aaboud:2017buf}, corrected for detector effects. 
%in a fiducial phase space - saving space, leaving it unsaid
This is an observable sensitive to the anomalous production of events with jets and \MET, and uses many of the techniques from the the jet+\MET search described above to estimate background. The constraints derived are comparable to those of the jet+\MET search with the equivalent dataset. Unlike most other searches for new physics described in this review, detector effects are already accounted for (\textit{unfolded}) when presenting results, so that there is no need to implement a detector simulation to reinterpret this search. 

\subsubsection{Searches with photons and vector bosons}
\label{subsub:monoV}
%monophoton, monoV

Searches for invisible particles produced in association with a jet are the most sensitive among the searches employing an object radiated in the initial state, due to the large signal rates from the radiation of a gluon as opposed to the radiation of a photon or a W/Z/Higgs boson. The jet+\MET final state is however also affected by the largest SM and instrumental background, and only covers signals producing a high \MET to comply with data-taking limitations at the trigger level, due to the high-rate backgrounds producing signal-like signatures. It is therefore worth considering other objects as ISR, as those searches will be subject to different backgrounds, different kinds of systematic uncertainties, lower \MET thresholds, and can provide confirmation in case of an excess in the jet+\MET final state~\cite{Birkedal:2004xn,Petriello:2008pu}. 
%Gershtein:2008bf 2nd monophoton paper, save cites
Moreover, the sensitivity hierarchy of \MET+X searches does not necessarily privilege the jet+\MET final state if there is a direct new physics coupling between a vector boson and the DM, as in the case of the EFT model mentioned in Section~\ref{sub:EFT}, or if the radiated object is a new particle~\cite{Autran:2015mfa}. 
%This latter signal motivates searches in the \MET+generic resonance final state CD: keep for later. 

ATLAS and CMS have pursued searches for missing transverse momentum produced in association with a photon%monophoton - if we're running low on citation space, remove CMS monophoton
\cite{Aaboud:2017dor,CMS-PAS-EXO-16-014},%less lumi, unpublished, ,
vector boson decaying hadronically %monoV, had, 2015+2016 (CMS) and 2015 (ATLAS)
\cite{Sirunyan:2017jix,Aaboud:2016qgg} or leptonically %monoV, lep, 2015+2016
\cite{Aaboud:2017bja, Sirunyan:2017qfc}. 

One of the advantages of these search signatures over the jet+\MET one
is the lower event selection threshold, thanks to the additional handles provided by 
either the photon object or the leptonic decays. As an example, the lowest \MET 
value for the search is 100 GeV for the leptonic Z+\MET search~\cite{Sirunyan:2017qfc} %CMS https://arxiv.org/pdf/1711.00431.pdf
as opposed to 200 GeV for the jet+\MET search~\cite{Sirunyan:2017jix}. 

%CD: this can be cut except for the first sentence? 
The event selection and the background estimation strategy depends on the final state, but generally mirror those of the jet+\MET search. 
The \textbf{photon+\MET searches} use a photon to trigger the events to be recorded for analysis, and selects events containing an isolated photon above 150 GeV and no leptons. The number of events from Z decays to neutrinos in association with a photon can be estimated in events where the lepton veto is inverted and the contribution of visible Z and W boson decays is removed from the transverse momentum balance of the event, and transferred to the signal region. 
%Don't want Zgamma resonances so restrict mllgamma < 1 TeV, but maybe too much info
Jets and leptons faking photons are estimated directly from data, and the $\gamma$+jet background where the jet is mismeasured and produced \MET is suppressed by the requirement that the photon and the direction of the \MET vector do not overlap in the azimuthal plane. 
The \textbf{W/Z+\MET searches} where the vector boson (V) decays to a quark-antiquark pair specifically select events where the decay products from the high-\pt{} boson are collimated, to better discriminate signal and background. QCD jets will not present any \textit{substructure}, while the decay products of vector bosons grouped into large-radius jets have a typical two-prong pattern from the hadronization of the quark-antiquark pair. The dominant background is still Z decays to neutrinos in association with jets, followed by W decays where the lepton is not identified, and top quark decays. The shapes of these backgrounds are estimated using simulation, while the normalization is determined in control regions, similarly to the jet+\MET search. In the CMS search, the V+\MET backgrounds are estimated in a simultaneous fit together with the jet+\MET backgrounds. The main uncertainty for this search (up to 9\%~\cite{Sirunyan:2017jix} and 13\%~\cite{Aaboud:2016qgg} %ATLAS, CMS is 9\% due to the tagging requirements
) is due to the modeling of the substructure observables. 
\begin{marginnote}[]
\entry{Jet substructure}{a set of techniques employed to extract information from the radiation pattern of a jet, by analyzing its constituents or its reconstruction history. For a review, see ~\cite{Larkoski:2017jix}.}
\end{marginnote}
The \textbf{Z+\MET searches} where the Z boson decays leptonically~\footnote{Leptonic decays W bosons have also been employed in the past for this kind of searches, but due to the additional experimental challenges (e.g. the presence of an additional invisible particle, the neutrino in the W decay) and the reduced sensitivity with respect to the hadronic decays, they have not been specifically pursued as DM searches for the LHC Run-2.} are sufficiently general to be sensitive to simplified models of DM with a Z radiation, as well as to Higgs decaying into new invisible particles and produced in association with a Z~\cite{Sirunyan:2017qfc, Aaboud:2017bja}. The event selection includes a constraint on the dilepton invariant mass, which limits the backgrounds to diboson, leptonically decaying top quarks, Drell-Yan production and a small amount of triboson processes. The estimation of the main $ZZ\rightarrow 2\nu 2l$ background (about 60\% of the total backgrounds) uses simulation, as the data sample that could be used to constrain the normalization as in the jet+\MET search is statistically-limited. The main uncertainty for this search (10\% on the background estimation) comes from the theoretical uncertainties on this background. The CMS search uses a Boosted Decision Tree (BDT) applied to events with the missing transverse momentum between 100 and 130 GeV, to enhance the sensitivity to invisible Higgs decays. 

The photon+\MET searches provide the next-to-most stringent constraints after the jet+\MET search, 
up to \mmed $<$ 1200 GeV for vector and axial vector mediators with \gdm=1.0 and \gq=0.25 for \mdm=100 GeV,
in the region where the mediator can decay to DM. %Not mentioning off-shell, otherwise it's a closed region. 
The ATLAS photon+\MET search also sets limit on the EFT model where the DM couples directly to the photon, 
excluding EFT scales between 150 and 750 GeV for \mdm=100 GeV, assuming the maximal coupling value allowed by perturbativity. 
%The excluded region decreases to 150-600 GeV for a coupling value of 3. 
Searches in the \MET+hadronic Z final state constrain vector and axial vector mediator
masses of \mmed $<$ 650 GeV for \mdm=100 GeV\footnote{As a side note that is useful to compare results from different LHC datasets, 
Run-1 V+\MET searches used a version of the vector simplified model
that enhanced W radiation because of the constructive interference 
due to different up- and down-quark mediator couplings,
but that was not gauge invariant~\cite{Bell:2015sza,1475-7516-2016-01-051}.}. 
Searches in the \MET+leptonic Z final state provide constraints on 
\mmed $<$ 650 GeV for \mdm=100 GeV for vector and axial vector mediators. 
W/Z+\MET searches are also uniquely sensitive to the radiation of a boson 
from the mediator particle in the case of colored scalar models~\cite{Bell:2012rg}, 
but the Run-2 searches do not present this interpretation. 
%CD: maybe we add a link to the 8 TeV search but it's worse than monojet 

\subsubsection{Search signatures including the Higgs boson}
%monoH, H to invisible detailed earlier on

%CD: we already said that before
The newly discovered Higgs boson is of particular interest for DM searches at the LHC. 
Higgs radiation is kinematically and PDF suppressed, but searches for a mono-Higgs have
other strong theoretical motivations. 
Higgs portal models are the simplest incarnation of theories where the coupling
between the dark sector and the SM is realized through a Higgs boson. Higgs couplings
to at least a new scalar are necessary for the gauge invariance of simplified models described in
the previous chapters towards more complete theories. 
%The rates of signatures including Higgs bosons are small, but 
Gauge symmetries link Higgs+\MET signatures and signatures including W, Z bosons
or jets as well as two-body mediator searches~\cite{Liew:2016oon}. 
%so results from Higgs searches are not to be taken in isolation.  

The search strategy depends on the decay mode of the Higgs boson. 
With the current LHC dataset (2015+2016 Run-2, 36 $fb^{-1}$), only the 
$H \rightarrow \gamma\gamma$ and $H \rightarrow b\bar{b}$ decay channels 
have been used to search for DM, due to their relative experimental simplicity
and high rates. Other decay channels such as $ZZ, WW$ and $\tau\tau$ are 
expected to contribute to DM searches as well in the future. 

Searches in the decay channel 
$H \rightarrow \gamma\gamma$ in association with \MET~\cite{CMS-PAS-EXO-16-054,Aaboud:2017uak}
are sensitive to a variety of benchmark models regardless of the small branching fraction, thanks to the 
high precision in the reconstructed Higgs boson mass and the ability to probe low \MET 
thresholds compared to other Higgs decay channel as the trigger rates are low. 
The SM background is estimated using a fit to the diphoton mass distribution, in events categorized
according to their missing transverse momentum for CMS (50$<$\MET$<$130 GeV and \MET$>130$ GeV)
or according to specifications optimised for different signal categories.%this is useless but the analysis is needlessly complicated  
The main uncertainty for the $H \rightarrow \gamma\gamma$ searches using the 2015+2016
LHC Run-2 dataset is statistical. 
In the search where the Higgs boson decays into two bottom quarks~\cite{Aaboud:2017yqz}
in association with \MET$>$150 GeV, 
all backgrounds except for the QCD background are estimated using MC simulation
and constrained in dedicated control regions. This search also employs jet substructure
techniques for events with \MET$>$500 GeV,
to discriminate boosted Higgs decays from QCD processes. 
The main systematic uncertainty for the lower \MET signal region is the modelling 
of the V+jets background, while higher \MET signal region is still statistically limited
with the 2015+2016 LHC dataset.

In absence of discrepancies between data and background,
limits are set on the baryonic Higgs benchmark model outlined in
Sec.~\ref{sub:simplifiedModels} with 
\gq=1, \gdm=1, \ghZprimeZprime/$m_{Z}$=1, \sinthetab=0.3, 
%CMS: mZ'=10-10000 GeV, mDM=1-1000 GeV
%CD: it would be nice to say what this can be reinterpreted to but we have no space
and on a Z'-2HDM model with $tan\beta$=1, \gZPrime=0.8 and \mdm=100 GeV~\footnote{ 
In the case of the Z'-2HDM model, CMS and ATLAS set different masses for the 
new Higgs bosons, 
%https://docs.google.com/presentation/d/10R9XJaoMDEhXKhd_Wx9yMXEaPl4uXR8IcmuTeLancvg/edit#slide=id.g1f308da957_0_17
%ATLAS fixes both to 300 GeV, CMS fixes to mA0 
%For the record:
%CMS: A and Z' varied between 300-800 and 600-2500 GeV respectively
%ATLAS: mZ? = 400 to 1400 GeV, mA0 = 200 to 450 GeV
%The masses of the neutral CP-even scalar (H0) and the charged scalars (H�) from Z?-2HDM model are set to 300 GeV. The DM mass m? is set to 100 GeV 
%CMS: 
%Two-Higgs-doublet-Z' signals with a pseudoscalar mass of 300 GeV are excluded at 95\% CL for Z' masses below 900 GeV
%Baryonic Z' models with a dark matter mass of 1 GeV are excluded at 95% CL for Z' masses below 800 GeV
so the constraints are not directly comparable. 
This has been rectified in the coming iteration of these analyses.}. 

%CD: I kinda want to say this but we have no space so why bother
%the mandala boson is shit, misguided attempts at combining Higgs discovery with every other excess

\subsubsection{Searches with heavy-flavor quarks}
%ttbar+MET
%reinterpretation of SUSY

Generic searches employing one single additional object produced in association with \MET
are powerful tools to probe simple models of DM. More complex models, however, bring 
more handles for discovery: the first step in this direction can be taken with searches using 
scalar and pseudoscalar models as benchmarks, where information about the production mechanism 
(e.g. the mediator is produced in association with two heavy flavor quark, complementing the
gluon-fusion production mode of the \MET+jet searches) is exploited in 
the search strategy. 

The searches in~\cite{Aaboud:2017rzf,CMS-PAS-EXO-16-051}
are optimized for DM scalar and pseudoscalar mediators
selecting events in the semileptonic and fully hadronic top quark decay channels,
as well as events containing one or two bottom quarks, in association with \MET. 
The dominant backgrounds in~\cite{Aaboud:2017rzf} are estimated separately
using MC in each of the signal regions,
and their normalization constrained using control regions in a simultaneous fit.
The main uncertainties for these searches are, depending on the signal region, 
theoretical and MC simulation related uncertainties, jet energy scale and resolution. 
and uncertainties related to the identification of heavy flavor quarks. 
Signatures including \MET and two heavy flavor quarks are similar to 
signatures of third generation quark superpartners, leading to dedicated
DM signal regions being included in SUSY searches or used
for reinterpretation~\cite{Aaboud:2017aeu,CMS-PAS-SUS-17-001}. 
In SUSY-like searches, the dominant $\t\bar{t}$ backgrounds
are heavily suppressed using variables that combine visible and invisible
mass~\cite{Lester:1999tx} targeting the model sought. 
This step uses information that is model-specific,
but increases the sensitivity to specific processes. The remaining
small backgrounds are estimated using simulation. 

The sensitivity of searches of \MET associated to top quarks
is comparable for the two strategies. 
For a choice of \mdm=1 GeV, pseudoscalar mediator masses of 10-50
GeV~\cite{AAaboud:2017aeu} and scalar mediator masses up to 100
GeV~\cite{CMS-PAS-SUS-17-001} are excluded. 
%CD: it would be nice to find out why this difference in sensitivity? 
The increased LHC dataset 
will allow these searches to be sensitive for other DM masses. 
%CD: this sentence is shit but i'm tired
Signatures with $b\bar{b}$ pairs are less sensitive to
scalar and pseudoscalar mediators that do not explicitly
privilege bottom quarks. Mediator masses for the b-flavored colored scalar 
model discussed in~\cite{Agrawal:2014una} are excluded up to 1.1 TeV
for \mdm=35 GeV. 

%Not spending more than one sentence on monotop, is that ok?
Other searches in the heavy flavor+\MET category are those only
including only one top or bottom quark
(also called mono-top or mono-bottom searches)~\cite{CMS-PAS-EXO-16-051},
and place constraints on models that include singly-produced DM candidates
described in~\cite{Boucheneb:2014wza}. 

%The main backgrounds for these searches are single top or misidentified $t\bar{t}$ processes. 
%The search where the top quark decays hadronically
%employs substructure techniques to tag the boosted top quark decays.  
%These searches place constraints on models of DM (resonant, non-resonant). 

\subsubsection{Two-body mediator searches}
\label{sub:twoBody}
%Dijet and dilepton
%Mention TLA

Mention LianTao's paper where baryonic / 2HDM monoHiggs is also constrained. 

This is two body mediator searches. 

%Sidebar (50 words minimum, 200 words maximum) briefly discussing a fascinating adjacent topic; insert below Literature Cited section, but indicate near which section in text the sidebar should be typeset
\begin{textbox}[!h]
\section{Challenges in selecting events at the detector level (trigger)}
%Trying to approximate 10 words per line
The LHC collides protons every 25 $\mathrm{ns}$, producing 40 billion 
of events per second. This amount of data cannot be 
recorded in its entirety, and not all events are interesting
for the experiments' physics programmes. A trigger system is used
to choose whether an event is selected for further analysis. 
Its first level is realized in hardware and only uses
partial detector information for fast decisions in a time of
the order of $\mathrm{\mu s}$, while its second level is software-based
and uses more refined algorithms and information to make a
decision in $\mathrm{ms}$. Since the rates of SM physics processes decrease
with the transverse momentum of the objects involved, and processes
with a high momentum transfer have a higher chance of containing
interesting features or new particles, the trigger system records
events above a certain threshold e.g. in leading jet \pt or in event \MET. 
Only a fraction of events that do not satisfy these thresholds is recorded. 
Simultaneous proton-proton interactions occurring within the detector
readout time cannot be completely disentangled from the hard process
of interest. This \textit{pile-up} increases the likelihood of passing 
the minimum threshold to record events. For this reason, the increase in the
LHC instantaneous luminosity by virtue of increasing the number of simultaneous
collisions leads to increases in the thresholds to keep manageable event recording rates,
and penalizes searches for signals with low MET or \pt,
unless different data selection and recording techniques are employed.
\end{textbox}

\subsubsection{Comparison of sensitivity of visible and invisible LHC searches}
\label{sub:comparisonVisibleInvisible}
%It would be nice to make tables of lowest mediator/DM searches+refs, as a poor-person approximation of a summary plot we can't make because ATLAS data not public. 

%%%SUMMARY TABLE FOR MONOX SEARCHES
\begin{table}[h]
\tabcolsep7.5pt
\caption{Summary of searches for BSM mediators at the LHC}
\label{tab:BSMSearchesSummary}
\begin{center}
\begin{tabular}{@{}l|c|c|c|c@{}}
\hline
Signature & Model& \mmed limit & \mdm limit  & Cit.\\
 &  & (\mdm=100 GeV) & (\mmed=100 GeV)  &  \\
%{(}units)$^{\rm a}$ &Head 2 &Head 3 &Head 4 &{(}units)\\
\hline
Jets+\MET & $s-$channel, AV$^{\rm a}$ & Column3 & Column4 & \cite{Sirunyan:2017jix,Aaboud:2017phn} \\
Z (lep)+\MET 1 & $s-$channel, AV$^{\rm a}$ & Column3 & Column4 & \cite{Aaboud:2017bja, Sirunyan:2017qfc}\\
Photon+\MET & $s-$channel, AV$^{\rm a}$ & Column3 & Column4  & \cite{Aaboud:2017dor,CMS-PAS-EXO-16-014} \\

Jets+\MET & colored scalar & Column3 & Column4 & \cite{Sirunyan:2017jix,Aaboud:2017phn} \\
Jets+\MET & pseudoscalar & Column3 & Column4 & \cite{Sirunyan:2017jix,Aaboud:2017phn} \\
Photon+\MET & Column 2 & Column3 & Column4 & Column\\
W,Z (had)+\MET 1 & Column 2 & Column3 & Column4 &Column\\
W,Z (lep)+\MET 1 & Column 2 & Column3 & Column4 &Column\\
Higgs+\MET &Column 2 & Column3 & Column4 &Column\\
\hline
\end{tabular}
\end{center}
%\begin{tabnote}
$^{\rm a}$ Coupling values: \gq=0.25, \gdm=1.0; $^{\rm b}$second table footnote.
%\end{tabnote}
\end{table}



%The CMS analysis also scans the coupling-mass plane by fixing the ratio between \mdm and \mmed to ensure perturbativity - nice but maybe we put it later. 
%CMS sentence: Quark couplings down to 0.05 for mediator masses at 50 GeV are excluded for the spin- 1 simplified models as shown in Fig. 12. 

\subsection{Searches for SUSY DM}
\label{sec:results_SUSYSearches}

Left for AB. 

\subsubsection{Searches for sparticles}

\subsubsection{pMSSM scans}

%mention GAMBIT

\subsubsection{Searches for electroweakinos}

%CD: commented all of this out in favour of three sidebars
%\subsection{Experimental challenges for DM searches at the LHC}
%\label{sec:experimentalChallenges}

%Now that we have seen the searches for invisible particles and their sensitivity to DM models, we will cover the experimental challenges more in detail, as those will be the key to fully exploit %bleurgh
%the Run-2 and Run-3 datasets. 

\subsection{Searches for DM in association with long-lived particles}
\label{sec:results_LLPSearches}

%beginning with the searches that illustrate many of the experimental 
%
%
% and then signals of MET. 
%
% description has a historical and ordering
%We begin with introducing 
%
%start with introducing the searches, mostly  
%
% of DM searches, we will turn to a [categorization] of the searches done so far. After a summary of searches for interactions through SM bosons, we turn to the experimental results 


%As described in the previous chapter, DM itself is not visible at colliders and has
%to be observed indirectly in association with other visible particles. 
%
%Signals of DM can come from MonoX and diX searches. 
%
%State of the art MC:
%
%Vector and scalar models are known at NLO~\cite{Neubert:2015fka,Haisch:2013ata}
%NLO corrections for vector and scalar models in monoZ and monojet
%
%t-channel
%
%From Millie's talk (see Reaction OmniOutliner):
%Since the DM-mediator-quark vertices allow for simultaneous FS partons with different hard scale, particular prescription needs to be used for the generation of samples with different partons that splits samples in number of mediators. Interference between the diagrams is neglected following Papucci et al. 

%%%Bunch of text from ERC that could be useful for TLA part

%This is what I wrote in ERC, not sure if it's usable
%Not knowing the exact nature of these new particles, it is not possible
%to pin down at which energy scales they are going to be produced, 
%and at which rates. This calls for generic searches that are sensitive 
%to a broad range of theoretical benchmarks, and whose results can be easily
%re-interpreted in different frameworks. The absence of any New Physics discoveries 
%during the first LHC run also points to the need for the study of more elaborate 
%and rare processes that can only be achieved with new analysis techniques. 

%At the LHC, proton beams with a center of mass energy of 13 \tera\electronvolt \ will collide up to 
%every 25 \nano\second \ at the four interaction points, where experiments are installed. 
%The ATLAS (\textbf{A} \textbf{T}oroidal \textbf{L}HC \textbf{A}pparatu\textbf{S}) experiment 
%is a general purpose detector located at the Interaction Point 1.
%From the Spring of 2010 to Fall 2012 the LHC has recorded a total of 25 \invfb of collision data 
%at 7 and 8 \tera\electronvolt. 
%
%It is the high interaction rate of proton-proton collisions at the LHC that 
%allows the collection of the large dataset necessary to search for rare processes. 
%The design and operation of the ATLAS detector is described in 
%References~\cite{DetectorPaper}. The key subsystems employed in this projects are the calorimeters, 
%where energy deposits from the collimated sprays of particles originated by 
%quarks and gluons are detected. The energy deposits are used as inputs of 
%\textit{jet algorithms}~\cite{Salam:2009jx}: 
%the experimental output used for the analysis is a reconstructed jet that, after calibration, 
%contains information on the kinematics of the original physics process. 
%The system and technologies employed to select and collect interesting data are the crucial components for the 
%innovative Trigger Level Analysis searches outlined in WP2 and WP3, 
%and they are described in more detail in Section~\ref{subsub:datacollection}. 
%
%The LHC will restart operations in Summer 2015, increasing the center of mass energy to 13 TeV. 
%The dataset collected over the course of the planned three years of operations (called \textit{Run 2}, 
%ongoing until mid-2018) corresponds to 100 \invpb~\cite{KEKRossi}. 
%Upgrades to further increase the data rate and the center of mass energy to 14 TeV will be 
%undertaken during the course of 2016, and completed during the 1.5 year technical stop (called LS2).
%LS2 will last until the beginning of 2020, leading to the third LHC run 
%that will collect an additional 200 \invpb of data. 
%
%During LS2, the LHC experiments will undergo major upgrades to their hardware (called Phase-I upgrades). 
%In the final two years of this project that coincide with the LHC Run 3, 
%I will exploit the new gFEX trigger board~\cite{Bartoldus:1602235} to extend the Trigger-Level Analysis 
%method to obtain an increased sensitivity for four-jet final states. 
%
%\subsubsection{Data collection and data analysis in ATLAS at the LHC}
%\label{subsub:datacollection}
%
%The data rate delivered by the LHC (up to 40 million collisions per second in nominal conditions) exceeds the 
%current capabilities for recording events offline, both in terms of recording speed and storage space. 
%The rare signal events sought are buried in an overwhelming number of background events.
%It is therefore necessary to have a \textit{trigger} system that in a very short timescale 
%selects the interesting collision events based on the presence of high transverse momentum physics objects 
%(muons, electrons, photons, jets and tau leptons). 
%
%The ATLAS data collection rate at the end of the trigger selection still
%surpasses other ``Big data'' 
%challenges both in academic research and in industry. 
%As an example, ATLAS and the three other main detectors at the LHC produced and recorded 
%13 petabytes of data just in 2010~\cite{NaturePetabyte}. It is clear
%that with such large dataset new ideas and new analysis techniques are needed,
%especially in the case of small deviations over large backgrounds. 
%One such idea consists of moving the data analysis as close
%to the data taking as possible, ultimately removing the need for storage at all. 
%This proposal moves the first steps for the ATLAS detector in this direction, 
%proposing to only retain a subset of the event information, and eventually 
%to move an entire search to be done in real-time as soon as the data
%is collected. 
%
%\paragraph{The ATLAS Trigger system}
%
%The ATLAS trigger system is subdivided in two levels: Level-1 and High Level Trigger (HLT)
%The first, fast selection is made at L1 using coarse detector information. 
%Its logic is mostly hardwired in the readout electronics, 
%given that the decision needs to be made in less than 2.5 $\mu$ \second. 
%The bandwidth available for data transfer to the HLT and the HLT computing power
%limit the rate that could be accepted by the L1 to 75 kHZ from the initial 40 MHz 
%provided by the LHC~\cite{Sfyrla:1510140}. The events accepted by the L1 trigger 
%are passed on to the HLT trigger, a software subsystem. The longer latency allows 
%to perform a more refined reconstruction of the objects that are used for the selection.
%Jets reconstructed within the HLT system will be called \textit{trigger jets} in the following,
%while \textit{offline jets} are those reconstructed after the event has passed the full trigger chain.
%During the 2012 data taking, ATLAS recorded and reconstructed data at a rate of 
%400 Hz, leaving an additional 200 Hz of recorded data for later reconstruction
%(\textit{delayed} data).  
%
%Since the rate for certain signatures (e.g. multi-jet events that are backgrounds of the 
%searches outlined in this proposal) would saturate the limited bandwidth that needs to be shared 
%by all triggers, some triggers are \textit{prescaled}. 
%This means that only a fraction of the events accepted 
%are effectively passed onto the next level.
%A prescale of 1 means that all events 
%selected by the trigger are accepted, while larger prescales mean that only a fraction 
%1/prescale is accepted. This in turn harms the sensitivity of searches, as signal events
%will be rejected by the trigger system as well. 

%Blurb from ERC

%Hadronic jets are a particularly promising final state for both the Dark Matter mediators 
%produced at the LHC, but also for new, unknown particles that might be created when crossing 
%the threshold of a new energy scale such as in the upcoming LHC run. 
%Not knowing the exact nature of these new particles, it is not possible
%to pin down at which energy scales they are going to be produced, 
%and at which rates. This calls for generic searches that are sensitive 
%to a broad range of theoretical benchmarks, and whose results can be easily
%re-interpreted in different frameworks. The absence of any New Physics discoveries 
%during the first LHC run also points to the need for the study of more elaborate 
%and rare processes that can only be achieved with new analysis techniques. 

%A wide range of theoretical models attempt to incorporate Dark Matter. 
%Many of these models postulate the presence of a new massive subatomic particle
%that interacts only feebly with SM particles as a Dark Matter candidate.
%The presence of these Weakly Interacting Massive Particles (WIMPs) can be inferred in 
%in a variety of experiments. \textit{Direct Detection} 
%experiments detect the interaction between an incoming 
%Dark Matter particle and target nuclei within the detector, by measuring nucleon recoil.
%\textit{Indirect Detection} experiments detect the fluxes of SM particles that 
%are produced from the annihilation of DM particles~\cite{Bertone:2004pz, Bauer:2013ihz}. 
%Dark Matter searches at colliders are supported by the consistency
%of the thermal relic density and the annihilation rates of WIMP candidates with masses
%in the GeV-TeV range, compatible with the energy regime of the LHC~\cite{Bertone:2004pz}. 
%The complementarity of those searches in terms of the WIMP-SM interaction of interest 
%is shown in Figures~\ref{fig:complementarity} (a-c). 
%
%\begin{wrapfigure}{L}{0.7\textwidth}
%% \begin{figure}[!h]
%\centering
%    \includegraphics[width=0.65\textwidth]{figures/SimplifiedModels}
%  \caption[Complementarity of DM searches]{\label{fig:complementarity} Sketch showing the complementarity 
%  between different experiments searching for Dark Matter (a-c). The difference between
%  the EFT approach and the simplified model approach is depicted for collider searches in (c-d).}
%% \end{figure}
%\end{wrapfigure}
%
%Indirect detection experiments such as 
%FERMI~\cite{Hooper:2010mq}, AMS 
%~\cite{PhysRevLett.113.121101} and DAMA~\cite{Bernabei:2003za}
%have observed tantalizing signals 
%with a possible Dark Matter explanation.
%There some tension between direct detection experiments: the signal-like excesses 
%and characteristics of events in the CDMS and CoGENT experiments~\cite{Agnese:2013rvf,Hooper:2010uy}
%are not confirmed by other experiments such as LUX~\cite{Akerib:2013tjd}. 
%
%The flagship searches for Dark Matter at the LHC and at the Tevatron 
%exploit the recoil of undetected pair-produced WIMPs against a jet radiated 
%by one of the initial-state quarks or gluons. Such searches have yet to find 
%evidence for WIMPs (see e.g. Refs. ~\cite{Aad:2013oja,ATLAS:2014wra,Aad:2014vka,Aad:2014vea,Aad:2014tda,ATLAS-CONF-2012-147}
%for the ATLAS Collaboration). 
%
%Results from all these experiments need to be connected in a coherent framework for a 
%successful program of study of Dark Matter in the coming years. 


\section{EXTRAPOLATION OF COLLIDER RESULTS}
\label{sec:04_Extrapolation}
A wide variety of reactions may produce {\IP}s at colliders, and, if the mediators of the interaction are light enough to be produced on-shell, collider experiments are particularly suited to discovering and characterizing the interactions responsible.  Meanwhile, connecting a collider experiment's discovery or non-discovery of {\IP}s to dark matter requires direct and indirect detection experiments, where galactic dark matter collides with a terrestrial target, or extragalactic dark matter annihilates.

Making this connection requires that one assumes a particle physics model.
Within a given model and under well-specified assumptions, the information obtained in a collider experiment can be related to the information obtained in direct, indirect, and astrophysical probes, and vice versa.
One can then compare and contrast the different types of information, e.g., to understand where a DM discovery in current DD searches could be further explored with mediator studies at the LHC, and where, amongst the multitude of possible signals, collider searches might focus effort.

In the following, we outline a strategy adopted by the ATLAS and CMS experiments when making comparisons with astrophysical observations (e.g., where is a model consistent with the present DM density in the universe), DD, and ID results. We discuss the assumptions made in the relic density calculation and in relating reactions for {\IP}s to reactions of DM.

\subsection{Comparing LHC constraints from visible and invisible searches with non-collider results}

The ATLAS and CMS collider results typically appear as constraints on production cross sections of specific processes, which are then interpreted as statements about the fundamental parameters of a simplified model (masses, couplings), Within the model, information on the parameters can then be extrapolated to statements about the non-collider observable of interest---for example, the WIMP-nucleon scattering cross section for DD, or the thermal relic density.
In Ref.~\cite{Boveia:2016mrp}, the LHC Dark Matter Working Group provides an in-depth discussion of how to perform these extrapolations.
Recently, most general LHC search results have selected a single, specific set of BSM-mediated simplified models for extrapolation, due to their immediate relevance with the first 13~TeV collision data. But many other models can be used~\footnote{For reinterpretation of LHC results and their comparisons to DD and ID for scalar and pseudoscalar mediators, also in the context of 2HDM, see e.g.~\cite{Athron:2017kgt,Banerjee:2017wxi,Bell:2016ekl}}, and published searches typically provide some form of model-agnostic results for this purpose.


As an example, CMS has extrapolated the parameter exclusions obtained by a recent set of searches to the spin-independent WIMP-nucleon cross section of a direct detection experiment.
The result, Fig.~\ref{fig:SICMS}, with a selection of DD results also shown for comparison, illustrates some general features of many such comparisons.
For spin-independent scattering, LHC searches for heavy {\IP}s are generally in the position of confirming non-observation of DM in DD searches, whereas LHC invisible particle searches are sensitive to arbitrarily light {\IP}s while DD searches are not, and at intermediate DM masses both LHC and DD experiments have great potential for a discovery and could verify each other's claims.
For spin-dependent DD scattering (e.g., an axial-vector-mediated model), because the LHC signals are relatively insensitive to the Lorentz structure of the interaction while the DD signals are suppressed, similar plots show that LHC searches play a more powerful role relative to the DD searches over a wide range of invisible particle masses. 

In the same way, one can compare collider and ID results using simplified model benchmarks. In traditional comparisons, only one DM annihilation state at a time has been used for the comparison of collider and ID results (e.g. $b\bar{b}$, see for example~\cite{Agrawal:2014una}) but one can also compare ID and LHC results for models annihilating to multiple final state fermions.~\cite{Carpenter:2016thc}.

Recently, some DD and ID collaborations have adopted the benchmark simplified models being used by ATLAS and CMS, see e.g.\cite{PhysRevLett.118.251301,Balazs:2017hxh}.
IceCube and other experiments have used constraints from a MSSM scan, see e.g.~\cite{Aartsen:2016zhm}. The pMSSM is also a good framework to highlight the complementarity of LHC, direct and indirect detection experiments, as shown in e.g. Ref,~\cite{Cahill-Rowley:2014twa} and discussed later. %CD: m

\begin{marginnote}[]
\entry{The LHC Dark Matter Working Group}{provides  for the translation of LHC limits to DD and ID in~\cite{Boveia:2016mrp}, as well as reference sets of simplified models, calculations of relic density, and other tools to standardize such comparisons~\cite{githubDMWG}.} 
\end{marginnote}. 

It must be underlined that the exclusion regions obtained in this way will depend strongly on the assumption of the model.
The extrapolations are done with full knowledge that the simplified model is merely a crude guess, and one must be careful not to over-generalize.
More so, neither this procedure nor the simplified models themselves account for effects outside the model, such as interference and mixing with SM boson and quarkonia resonances, or the evolution of the operators in the model from the LHC collision energies to other energy scales~\cite{DEramo:2014nmf}.
Moreover, all experimental results, be it from DD, ID or collider, are affected by experimental and theoretical uncertainties that are not displayed here.


\begin{figure}[!htpb]
\includegraphics[width=0.9\textwidth]{figures/SI_CMSDD_Summary}
\caption{The 90\% CL constraints from the CMS experiment in the \mdm-spin-independent DM-nucleon plane, for a vector mediator, Dirac DM and couplings \gq = 0.25 and \gdm = 1.0, compared with DD experiments. From~\cite{CMSSummary}.}
\label{fig:SICMS}
\end{figure}

\subsection{Relic density considerations}
Absent a signal in non-collider experiments, the ability of a model to link its {\IP}s with the observed DM abundance is key to distinguishing it from other types of models of physics beyond the Standard Model.
Making this link, however, requires extrapolating from the present day to the early universe along an increasingly tenuous chain of assumptions.
For simplified models, this is especially problematic, because the model is designed to describe collider-scale processes; it may not even contain the interactions relevant in the early universe.
Nevertheless, it is interesting to examine where a model can make the link, even if in limited situations~\cite{Busoni:2014gta,Catena:2017xqq}.
For example, for the general simplified models in Sec.~[fix], one can compute with programs such as MadDM and MicrOMEGas~\cite{Backovic:2015cra,Barducci:2016pcb} the DM abundance for a standard thermal relic assuming that the interaction described by the simplified model is the one responsible for setting the relic density.
Often, e.g. Fig.~\ref{fig:sensitivityComparison}, ATLAS and CMS supplement their results with contours indicating where within a model this procedure obtains the correct dark matter density of $\omega_c = 0.12 h^2$.
Where the model cannot reproduce the correct abundance, it is an indication that either the model requires additional ingredients beyond those included in the simplified model, or that the chain of assumptions is incorrect~\cite{Bernal:2017kxu}.


\section{FUTURE EVOLUTION OF COLLIDER SEARCHES}
\label{sec:05_Future}
%Higgs and Z portals
%For future: mention future e+e- colliders in VH production. 


\clearpage

% Summary Points, one sentence each
\begin{summary}[SUMMARY POINTS]
\begin{enumerate}
\item Summary point Note. These should be full sentences.
\item Summary point 1. Both broad and targeted searches at the LHC are necessary
\item Summary point 2. The LHC is a mediator-producing machine
\item Summary point 3. DM is a good motivation to implement novel techniques 
\item Summary point 4. DM connected people in the DMF and DMWG, LHC community effort, so the reader can get involved or have questions
\item Summary point 5. Complementarity is important to elucidate DM 
\item Summary point 6. Nevertheless, these benchmark models have motivated novel search techniques
to look for low-coupling, low-mass resonances below the TeV scale that would
have otherwise not been explored in early Run-2 data due to experimental difficulties make same point for dark photons
\end{enumerate}
\end{summary}

% Future Issues
\begin{issues}[FUTURE ISSUES]
\begin{enumerate}
\item Summary point Note. These should be full sentences.
\item Future issue 1. HL-LHC and precision searches
\item Future issue 2. Dark sectors and DM particles near the range of the SM are worth looking for, not just high-mass EW scale WIMPs
\item Future issue 3. Look out for community efforts, try to include DD and ID and low-mass experiments 
\end{enumerate}
\end{issues}

%Disclosure
\section*{DISCLOSURE STATEMENT}
The authors are not aware of any affiliations, memberships, funding, or financial holdings that
might be perceived as affecting the objectivity of this review. 

% Acknowledgements
\section*{ACKNOWLEDGMENTS}
We thank Suchita Kulkarni, Valerio Ippolito, Christian Ohm and Frederik Ruehr for the help and advice in preparing this manuscript. 
We also thank Teng Jian Khoo, Ulrich Haisch, for useful discussion.
Add DOE and ERC grant info. 
% References
%
% Margin notes within bibliography
%\section*{LITERATURE\ CITED}
%To download the appropriate bibliography style file, please see \url{http://www.annualreviews.org/page/authors/author-instructions/preparing/latex}. 
%Please see the Style Guide document for instructions on preparing your Literature Cited.
%The citations should be numbered in order of appearance. For example:

\bibliography{AnnRev_Main}
\bibliographystyle{ar-style5}

\end{document}
