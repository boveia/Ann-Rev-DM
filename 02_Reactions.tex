\begin{marginnote}[]
In this chapter, we will link the observations on DM to its particle properties. We will then enumerate the possible reactions of DM at the LHC, building from simple to more complex models in terms of particle content.  
%\entry{Term A}{definition}
%\entry{Term B}{definition}
%\entry{Term C}{defintion}
\end{marginnote}

The observations mentioned in Section~\ref{sec:intro} require the dark matter particle to be stable on a cosmological timescale. This has important consequences for the prediction and observation of dark matter reactions at colliders. 

Firstly, a simple theoretical way to stabilize DM is the introduction
of a global $Z_2$ symmetry~\cite{Batell:2010bp}. A realization of this
symmetry can be found in R-parity in the MSSM. %citation?
Under this symmetry, the parity of the DM particle is odd, while the parity of SM particles is even. 
$Z_2$-parity is multiplicative and conserved: this 
%is Z_2 parity a thing? I don't want to have it confused with the global SM parity
implies that an odd-parity DM particle (charge -1) cannot decay into any 
lighter even-parity SM particles (charge +1) and it is therefore stable. 
Additionally, DM particles will be produced in pairs from the decay of other particles
that are charged under the same gauge group as the SM. A simplified diagram of an 
s-channel process at colliders is shown in Fig.~\ref{fig:sChannel}. This is the simplest 
DM production mode at the LHC, as it does not invoke additional particles beyond the DM. 
%TODO: add sidebar figure of s-channel. 
%this will become useful when we talk about s-channel mediators. maybe also make a point
%for the t-channel mediator?

Secondly, dark matter particles are invisible to detectors. 
However, the rest of the event is not: one can observe DM particles
produced in the event and escaping the detector 
due to their missing momentum in the transverse plane, if they recoil against one or 
more visible SM particles. 

%I don't like how this is linking up. 

%shared context: many possible new physics searches at the LHC
%problem: can't do them all
%solution: strong theoretical motivation, as well as observability
%exposition: particular case of DM

Collider experiments have a nearly unlimited choice of theoretically
motivated DM targets to search for. 
Theoretical arguments alone are not sufficient for a DM model to be tested at the LHC: 
couplings to SM particles need to feature in the model and be sufficiently large
to produce new particles and observe their signatures in the detectors. 

%everyone thinks of WIMPs, how strong is strong, how weak is weak? quantitative question of coupling, depends on model. in the introduction: need to talk about DM properties. Weak enough that there is no visible EM signal (no light emission or absorption). Relate those properties to what the particle physics properties need to be. Have a model in mind: s-channel mediator between DM and SM, weakness of interaction comes from particle being heavy or coupling being small. DMF models have order=1 couplings. 

Models of particle dark matter include SM couplings to satisfy cosmological observations in the freeze-out case. These couplings need to be weak enough that there is no visible signal of DM particles, as there is no evidence for DM interacting strongly with baryonic matter, nor for its emission or absorption of light. A typical DM-SM coupling satisfying relic density is of the order of XXX. 

The only SM particle that satisfies the requirement of being sufficiently weakly interacting is the neutrino. However, neutrinos cannot make up the totality of DM as they are not sufficiently massive to explain the galaxy structures that formed in the universe. %%CITE FENG AR, BERTONE'S BOOK
The upper bound on the neutrino content of DM is YYY. 
%cold vs hot dark matter, relativistic vs non-relativistic? no space to discuss. 

Even if we cannot observe DM itself at colliders, we can look for visible particles that are associated to Dark Matter. The LHC alone cannot solve the strong CP problem through observation of the axion, but it can still observe e.g. scalar resonances that appear in the theory. 

This raises the question of whether any of the SM particles could be associated to DM, for example in a similar fashion as the W and Z bosons mediate the weak interaction and produce neutrino pairs in the reaction. 

%Only the 

%It is the first time in the history of colliders that there is enough luminosity 


