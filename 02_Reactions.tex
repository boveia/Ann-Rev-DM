In this chapter, we describe reactions of DM from the perspective of experimental collider physicists, focusing on a selection of simple models that provide distinct and testable LHC signatures, without the ambition of theoretical completeness. For other perspectives on the models used for early LHC searches for DM, see~\cite{Kahlhoefer:2017dnp,Abercrombie:2015wmb,Arcadi:2017kky,Feng:2010gw}. 

The large body of theoretical literature on DM models featuring additional BSM particles drives the design of experimental searches in two complementary directions. In the case of self-consistent models of DM such as fully-developed SUSY models, all experimental handles are exploited for targeted searches that are sensitive to specific model features. These models will be described in the next section. However, the desire to make no assumptions on the DM phenomenology and to cast a net as wide as possible remains. The adoption of much simpler model as first LHC Run-2 DM benchmarks led to the design of more generic searches targeting the broad features of those models. The success of such simple, at times incomplete and not always theoretically sound models has been due to their ability to predict the key features and observables related to DM production at the LHC with only a limited number of new particles and theory parameters, factoring out the more complex processes that do not affect LHC phenomenology as they e.g. occur at higher energy scales. As proven by the history of SM discoveries, this simple approach can be used to discover the most prominent DM-SM interaction processes in the wake of the LHC start-up. 

Even if we cannot observe Dark Matter itself at colliders, we can look for visible particles that are associated to the interactions between Dark Matter and normal matter. The details of the models specify the nature and amount of accompanying particles. A similar situation arises in the SM. Colliders cannot detect neutrinos directly but can detect particles coupled to the W and Z bosons, the nature of which are determined by the details of the weak interaction. Since the nature of the interaction between SM and DM remains a mystery, we do not know these determinative details. We narrow our choice of models to those satisfying the requirements discussed in the introduction, and add further assumptions to restrict the scope of this review to models that are more widely studied:

\begin{enumerate}
\item We restrict our list to models that include a $Z_2$ symmetry to stabilize DM;
\item We describe models where the DM particle interacts with SM particles, either directly or indirectly;
\item We privilege models that have a connection with thermal relic from freeze-out. We remark however that there are other models from other cosmological histories (e.g. freeze-in) that can be considered and would lead to interesting LHC signatures~\cite{Bernal:2017kxu}. 
%Citation from DMFs
%[49] J. Abdallah, H. Araujo, A. Arbey, A. Ashkenazi, A. Belyaev, et al., Simplified models for dark matter searches at the LHCSubmitted to Phys.Dark Univ. arXiv:1506.03116.
% [47] A. A. Petrov, W. Shepherd, Searching for dark matter at LHC with mono- Higgs production, Phys.Lett. B730 (2014) 178–183. arXiv:1311.1511, doi:10.1016/j.physletb.2014.01.051.
% [51] R. S. Chivukula, H. Georgi, Composite Technicolor Standard Model, Phys.Lett. B188 (1987) 99. doi:10.1016/0370-2693(87)90713-1.
% [52] L. Hall, L. Randall, Weak scale effective supersymmetry, Phys.Rev.Lett. 65 (1990) 2939–2942. doi:10.1103/PhysRevLett.65.2939.
% 2570 [53]
% A. Buras, P. Gambino, M. Gorbahn, S. Jager, L. Silvestrini, Universal unitarity triangle and physics beyond the Standard Model, Phys.Lett. B500 (2001) 161–167. arXiv:hep-ph/0007085, doi:10.1016/S0370-2693(01) 00061-2.
% 2575
% [54] G. D’Ambrosio, G. Giudice, G. Isidori, A. Strumia, Minimal Flavor Viola- tion: An effective field theory approach, Nucl.Phys. B645 (2002) 155–187. arXiv:hep-ph/0207036, doi:10.1016/S0550-3213(02)00836-2.
\item We primarily consider models where DM is a Dirac fermion, relying on existing theory material developed for early Run-2 searches. Departures from these assumptions are discussed in~\cite{Abercrombie:2015wmb}. 
%Other cases yield similar phenomenology for LHC searches, with some exceptions that we describe in this chapter.
%The Dawn of FIMP Dark Matter: A Review of Models and Constraints  - https://arxiv.org/pdf/1706.07442.pdf, Minimal Decaying Dark Matter and the LHC - https://arxiv.org/pdf/1305.6587.pdf
\end{enumerate}

For the most part, models are chosen to mimic the pattern of flavor violation found in the SM (Minimal Flavor Violation, or MFV~\cite{DAmbrosio:2002vsn}). 

%In the following we enumerate the possible reactions of DM at colliders within these assumptions.

\subsection{Higgs and Z boson portals}
\label{sec:HZPortalModels}

We start from models that do not add any other particle to the SM except the DM and later continue to models with more complex particle content and interactions. 

Models where the SM particle sector is coupled to the dark sector through an existing or a new particle are called \textit{portal models}. Models where the SM particle sector is coupled to the dark sector through an existing or a new particle are called portal models. If one adds only a neutral DM particle to the SM content, one arrives at Z or Higgs portal models, which economically relate DM to the SM fermions through a neutral gauge boson. No other SM particle can serve as portal at leading order. 

%Models where the SM particle sector is coupled to the dark sector through an existing or a new particle are called \textit{portal models}. This kind of model leads to the most economical particle content for reactions at the LHC, as one only needs to add a neutral DM particle to the SM content if one of the SM particles is the portal particle. SM fermions cannot be portal particles under the assumption of a $Z_2$ symmetry, as they would allow the decay of DM. Photons, W bosons and gluons can't be portal particles either, as DM does not absorb nor emit light, nor it does it have electromagnetic or strong charge. The only viable SM portal particles remaining are the Z and the Higgs bosons. 

There are compelling theoretical and experimental arguments to explore SM portal models at modern colliders. 
The electroweak (EW) scale seems to be a special energy scale in nature, where the weak bosons, Higgs boson, and top quark masses lay. Moreover, it is natural to have mediators at this scale within DM theories predicting new weakly interacting particles~\cite{Cotta:2012nj,Arcadi:2014lta}.
However, only the recent generations of collider and direct detection experiments have had sufficient energy and luminosity to search for faintly-coupled mediators with EW-scale masses.

In \textbf{Z portal} models the DM particle couples to a Z boson. While this is perhaps the simplest model that one can construct involving DM and no other new particles, there are strong constraints on this model from LEP and from direct detection experiments, as discussed in Sec.~\ref{sec:results_ZHSearches}. 

The discovery of a SM-like Higgs boson
%~\cite{Aad:2012tfa,Chatrchyan:2012xdj} 
has now made it possible to test \textbf{Higgs portal} models, where the interaction between DM and SM particles is dominated by processes involving exchanges of the Higgs boson (see e.g. Refs.~\cite{Patt:2006fw,Englert:2011yb,Djouadi:2011aa}). For scalar or vector DM, this interaction is renormalizable and leads to a UV-complete theory. If DM is a fermion, further particles mediating the interaction are required to make the theory self-consistent~\cite{Freitas:2015hsa,Escudero:2016gzx,deSimone:2014pda}. 
For fermionic DM heavier than a few TeV, the model is not perturbative anymore. 
%and references therein? In case there is more room for H portal citations, should cite also: 
%https://arxiv.org/pdf/1304.2417.pdf
%From Escudero:2016gzx
%We point out that for fermionic dark matter heavier than several TeV, perturbative unitarity is lost, and higher dimension operators such as those ones considered in ref. [10] may become relevant for the phenomenology. It is interesting to note that within the context of the MSSM, a bino-like neutralino (with a subdominant higgsino fraction) can possess the characteristics found within this scenario [11].
%The couplings of the Higgs to DM can be scalar or pseudoscalar. This may need to be discussed later. 
%Another completion of the Dirac DM:
%S. Baek, P. Ko, and W.-I. Park, Search for the Higgs portal to a singlet fermionic dark matter at the LHC, J. High Energy Phys. 1202 047, (2012), [arXiv:1112.1847].
%From http://iopscience.iop.org/article/10.1088/1126-6708/2008/07/058/pdf
%Higgs-sector and Z′ interactions between the hidden sector
%and the SM states are special in that they involve gauge-invariant operators of dimension
%dO ≤ 4, and thus can be induced by physics at arbitrarily high scales with unsuppressed
%couplings. 
Since the properties of the Higgs bosons are modified by decays to DM, precision measurements of the Higgs width and couplings offer a probe for these models complementary to direct searches for the invisible particles, as described in the next chapter. 
%This class of models is already constrained by electroweak precision measurements, but still viable if the DM mass is about half the Higgs mass. 

%We could have a picture of constraints here?

%Arcadi
%However, the last LUX results[11], combined with the invisi- ble width of the Higgs excluded the Higgs-portal scenario for dark matter mass below 200 GeV [2].

\subsection{Effective Field Theories and Simplified models of BSM mediators}
\label{sec:BSMMediatorModels}

In the Z and Higgs portal models, the relationship between the mass of the mediators and the collision energy has important consequences for collider signals. If the collision energy is large in comparison to the mass of the mediators, then the mediators are produced on-shell and resonant effects are apparent. If on the other hand the collision energy is lower than the mediating particle mass, a situation arises similar to the Fermi model of weak interactions.
 
%~\cite{Fermi2008}. 

\subsubsection{Effective Field Theories}
\label{sub:EFT}

For the situation where the collision energy is much lower than the mediator mass, 
the physics of the interaction appears much simpler. In such a contact interaction,
a single rate parameter and the spin structure of the interaction describe the collider phenomenology.
One may hope that the situation is this simple for DM at modern colliders, as all unknown
interaction details are conveniently integrated out. This can encompass cases where new particles are much heavier than the SM bosons that mediate the interaction. In this case, we describe the collider production of invisible particles through Effective Field Theories (EFTs)~\cite{Goodman:2010ku, Shoemaker:2011vi}. 
A sketch of an EFT process at colliders is shown in panel (a) of Fig.~\ref{fig:monoX}.
The only parameter characterizing an EFT operator, in addition to the type of DM particle and to the type of SM-DM interaction, is the scale of the contact interaction, which control the overall rate of the process. For example, in the case of a $s-$channel completion of the EFT, this interaction scale is proportional to the mass of the mediator particle, which is in turn related to the momentum transfer. 

If instead the mediator mass is within the reach of the typical momentum transfer in the collision, the model needs to include further details of the SM-DM interactions~\cite{Fox:2011pm},
%~\cite{Buchmueller:2013dya,DeSimone:2016fbz,Berlin:2014cfa}, 
such as the mass of the mediator particle. If those details are unknown (e.g. the EFT completion is not known), one can use the EFT as a framework to provide results for reinterpretation once a completion is known~\cite{Racco:2015dxa,Busoni:2013lha} and to motivate exploration of distinct kinematic regions and signatures. 

%\begin{marginnote}[]
%\entry{EFT truncation}{If a completion of the EFT is not available, procedures describing how to truncate the events where the EFT description is not valid are available~\cite{Racco:2015dxa,Busoni:2014sya,Busoni:2013lha,Busoni:2014haa}. A recommendation on how to present EFT results from LHC searches can be found in~\cite{Abercrombie:2015wmb}.}
%\end{marginnote}

\subsubsection{Simplified models}
\label{sub:simplifiedModels}

For the situation where the collision energy is near or higher than the mediator mass, 
specifying the details
of the phenomenology at the collider energy scale is preferable to [truncating the EFT] if those details are known. 
The phenomenology in this case may also include complementary avenues to study the mediating interaction.
For example, when the SM-DM interaction is mediated by a \Zprime (additional neutral boson), producing 
this \Zprime at a hadron collider means it may often decay into the particles that produced it. 
These visible decays can provide more information about the interaction than the invisible decays alone, 
in the same fashion that the on-shell decays of the Z provide more information about the weak interactions than 
neutral current scattering at low energies.

The hard part is figuring out what these details should be. If, as discussed in the introduction,
we assume that only a small number of new particles are important for the collider phenomenology, simplified models can be developed for tree-level pair
production of invisible particles~\cite{Alwall:2008ag, Agrawal:2010fh, Alves:2011wf, Choudhury:2015lha}, and %, Gutschow:2012pw
organized according to their signatures~\cite{Abercrombie:2015wmb}. 
To form a full theory, these simplified models often have to be augmented with additional physics to satisfy theoretical requirements~\cite{Kahlhoefer:2015bea},
but for the purposes of describing the relevant collider phenomenology they are sufficient. 

%and employed for building a prioritized set of LHC search scenarios that is only loosely connected to specific theories of DM. 
%used as building blocks for more complex theories in models of DM and elsewhere, see e.g , an
%Even in the case of simple models, this review lays assumptions on what is covered, similarly to what has been done for the first LHC searches. In addition to the assumptions discussed in the introduction to this chapter, we restrict to models where the leading process is tree-level, leaving cases where the dominant contributions are of higher order for later study (see e.g. Ref.~\cite{Godbole:2015gma}). 

\begin{figure}[!htpb]
\includegraphics[width=\textwidth]{figures/MonoX.pdf}
\caption{Sketches of (a) the basic Standard Model (SM) - Dark Matter (DM) interaction at colliders in an effective field theory (EFT), (b) its extension as a basic simplified model where a new mediator particle is exchanged (including an additional energetic object radiated from one of the initial state quarks) and (c) the same simplified model where the mediator decays back into SM quarks. The coupling constant characterizing the mediator-quark interaction strenght is denoted as \gq, while the mediator-DM coupling constant is denoted as \gdm. From~\cite{monoXfig}.}
\label{fig:monoX}
\end{figure}

The Z and Higgs portal model discussed above are examples of such simplified models, where only a single particle, the DM particle, has been added to the SM. 
Simplified models of SM-DM mediation with one additional new boson, the mediator, are a natural step beyond this. 

At the start of the present data-taking Run-2 of the LHC, the ATLAS and CMS Collaborations, together with experts on particle theory, compiled a standard set of such models for invisible particle pair production~\cite{Abercrombie:2015wmb}. These represent a list of high-priority simplified models for ATLAS and CMS Run-2 searches, building upon a great deal of effort in the wider dark matter community (e.g.~\cite{Fox:2011pm,Yavin:14092893,Malik:2014ggr,Abdallah:2015ter}) 
\begin{marginnote}[]
\entry{The ATLAS/CMS Dark Matter Forum}{provided reference implementations of the models in Ref.~\cite{Abercrombie:2015wmb} at [Add reference.]} 
\end{marginnote}. 

These models include neutral mediator particles that are singly-produced at the LHC, and decay in pairs of DM particles as well as in pairs of SM particles. If the mediator is colored, it can lead to interactions between an incoming LHC parton and the DM particle. The phenomenology of colored mediators of DM is akin to that of SUSY models with a squark exchange~\cite{Papucci:2014iwa,An:2013xka,Bell:2012rg}. New mediator particles decaying in two bodies
are a simple, attractive benchmark to be directly produced at a particle collider that has just increased its center-of-mass energy. Models of BSM mediation can be classified according to the spin of the mediator: spin-1 vector or axial vector mediators (\Zprime), scalar mediators (termed $\phi$ in the following) and spin-2 vector mediators. This review does not cover in detail spin-2 mediators as they have not yet been adopted as benchmark models by LHC searches, even though they produce diboson signatures not present in other models~\cite{Kraml:2017atm,Han:2015cty}. For additional decay signatures, 'dark sectors' with more additional particles, and the statistical power of far larger datasets, many more simplified models become interesting. For a comprehensive review of WIMP simplified models of DM at the LHC, we refer to~\cite{Arcadi:2017kky}. [cite co-annihilation codex? other inventories?]

In the following, we give an overview of the important reactions presently used for ATLAS and CMS invisible particle searches from Ref.~\cite{Abercrombie:2015wmb}.

%%%s-channel
\textbf{Massive color-neutral spin-1 bosons with vector or axial-vector couplings} are nearly ubiquitous in BSM theories. 
It is therefore interesting to consider the \Zprime as the mediator particle between SM and DM~\cite{Shoemaker:2011vi},
as to connect simplified models of dark sector interactions with a wider classes of models.
A relevant characteristic of this model for LHC phenomenology is that if the \Zprime couples to quarks (as it needs to, in order to be produced at the LHC). 
then it must have visible decays back into quarks. This opens a new avenue for searches of this new
particle in dijet signatures with sensitivity in early LHC data, complementing searches for excesses
of missing transverse momentum. 

%Parameters
The simplest incarnation of this kind of model only needs to couple to DM and initial state partons. 
It is specified by the Lorentz structure of the \Zprime
couplings to SM fermions and DM (vector, axial vector or mixed), their magnitude (\gq for quarks, \gl for leptons, and \gDM), the mass of the DM particle
\mdm and the mediator mass \mmed. The Lorentz structure of the \Zprime couplings
does not significantly change the LHC phenomenology, but has a much larger effect in signals in non-collider searches. A coupling (or loop-level coupling) to SM partons is required to produce the mediator at the LHC, and it is also required to produce nuclear recoils in underground DM experiments. Lepton decays, if not included explicitly at tree level, arise through the quark coupling at loop level (see Ref.~\cite{Albert:2017onk} and references therein). Decays of the spin-1 mediator into neutrinos are also required by gauge invariance, and add an invisible decay channel that can enhance signatures of missing transverse momentum, depending on the size of the couplings~\cite{Albert:2017onk}. 
 %leading to a mixing between the Z' and the Z

%%changes the comparison of LHC results to DD and ID searches.
%Axial vector are the most widely used LHC benchmarks as DD rates are suppressed. 

%Additions: Higgs and Z' models
Vector and axial vector mediator models can include couplings of the \Zprime to the Higgs boson, 
so that the \Zprime boson can acquire mass through a new baryonic Higgs $h_B$~\cite{Berlin:2014cfa}.
This collapses to the simpler vector model in the limit of very heavy \Zprime mass. 
When the interaction between the \Zprime and the SM Higgs is relevant, 
it can be parameterized using the \Zprime-SM Higgs coupling \ghZprimeZprime and the mixing angle
between the SM Higgs and the baryonic Higgs, \sinthetab. 
%which in turn is a function of the vacuum expectation value of $h_B$. 
%This model can also lead to mono-Z signals, if the \Zprime is allowed to interact with the Z and the photon through kinetic mixing, 
%but this interaction has so far been neglected in LHC Run-2 searches. 
%CD: is there a paper by Vichi on this topic? Can't find it
A \Zprime can also be embedded in a Type-II Two-Higgs Doublet Model (2HDM)~\cite{Berlin:2014cfa,Liew:2016oon}.
%, where
%it does not decay directly into DM but instead decays into a new pseudoscalar $A^0$ and
%a SM Higgs boson with a coupling \gZprime~\cite{Berlin:2014cfa,Liew:2016oon}. Here the model parameter
%space is more complex, even when considering the alignment limit (one of the Higgs from the doublets is the SM Higgs).
%This model includes a mixing between the \Zprime and the SM Z boson that is proportional to
%the ratio between the vacuum expectation values of the new Higgs doublets ($tan \beta$). 

%Problems
In certain region of this parameter space, especially at low mediator masses,
the leptophobic Z' model can satisfy the relic density constraints~\cite{Chala:2015ama}.
However, if taken in isolation, these models are non-renormalizable, and the axial vector
model violates perturbative unitarity in certain regions of the off-shell parameter space,
if $\mdm^2\gdm^2/(\pi \mmed^2)<1/2$~\cite{Chala:2015ama,Kahlhoefer:2015bea,Boveia:2016mrp}. 

%%%%%
%%%%%
%%%%%

% If the mediator is a real scalar or a pseudoscalar singlet, it can have tree-level interactions with DM.

\textbf{Color-neutral scalar and pseudoscalar bosons} are the BSM analogue of the Higgs portal to DM. In comparison to the \Zprime models, a BSM (pseudo-)scalar mediator~\cite{Buckley:2014fba},  
has some additional peculiarities. 
Under the assumption that the simplified model has the same pattern of flavor violation as the SM, to evade flavor constraints, the couplings of the (pseudo-)scalar bosons to fermions are mass-dependent, similar to the Higgs boson. This has three consequences for
(pseudo-)scalar mediated models, familiar from Higgs physics: firstly,
loop-induced couplings to gluons~\cite{Haisch:2015ioa,Mattelaer:2015haa}
and associated production with heavy flavor quarks~\cite{Buckley:2014fba}
are important for their production,
secondly their cross-sections are much smaller
compared to those of vector mediators; thirdly, the visible decays
of these mediators are dominated by heavy-flavour quarks. 
Even though the production of (pseudo-)scalar mediator is
suppressed with respect to their (axial-)vector counterparts
at the same masses, the associated heavy quarks provide experimental handles that make those models testable at the Run-2 LHC.
%Single top signatures are also in reach of LHC searches~\cite{Pinna:2017tay}, as they are kinematically favored even though their production cross-section is generally lower with respect to the associated production of a pair of top quarks. %CD: this is interesting and new, but may be cut if we need to cut. 
%Remaining flavor constraints~\cite{Dolan:2014ska}.

%Parameters
(Pseudo)scalar models are fully defined given the masses of the DM particle and
of the mediator, the nature of the $\phi$-DM couplings (\gdm) and $\phi$-fermion (\gq) couplings.
Following the convention in~\cite{Abercrombie:2015wmb}, \gq is a pre-factor to the Yukawa couplings
of the mediator to fermions and it is considered equal for all quarks in early LHC searches. 
The LHC kinematics of models with scalar and pseudoscalar mediators assuming the same couplings is nearly
degenerate.
%Since the pseudoscalar model has been favoured for the interpretation of the DAMA and galactic center excess~\cite{Arina:2014yna,Agrawal:2014una} and collider searches are favored when comparing to DD~\cite{Banerjee:2017wxi}, LHC searches have privileged this choice and considered the associated scalar boson as decoupled at higher energies. 

%CD: this needs a feynman diagram otherwise one gets confused
When introducing additional scalars, one must take special care to consider the relationship between the new scalar and the Higgs boson. Large mixing with the Higgs can lead to strong constraints. In the model in Ref.~\cite{Berlin:2014cfa}, the scalar can couple to DM through a Higgs portal~\cite{Berlin:2014cfa}. The coupling between the scalar and the Higgs is $b$ (set to unity for simplicity, and the mixing angle is \sinthetahS, constrained to be below 0.4 by Higgs precision measurements. The values of these two parameters don't affect the kinematics significantly. 
Couplings to other SM particles, notably EW gauge bosons, can also be added as a consequence of electroweak symmetry breaking~\cite{Bauer:2016gys,Englert:2016joy}. The signatures of these models include signals of Higgs and vector bosons plus missing transverse momentum at tree level and invisible decays of the Higgs boson if the DM particle is lighter than the SM Higgs.
%CD: check back on emails with DiFranzo

%Where it is applicable and what are the pitfalls
If the mediators are pure SM singlets, the model is not invariant under $SU(2)_L$~\cite{Bell:2016ekl}. 
%Maybe add Uli/No/etc's papers here?
To restore gauge invariance, mixing with the Higgs sector is crucial. 

%%%t-channel, colored
%General
\textbf{Colored scalar and pseudoscalar bosons} allow interactions between SM particles carrying color and DM particles carrying $Z_2$ charge~\cite{Bai:2013iqa, Papucci:2014iwa, An:2013xka, Bell:2012rg}. Models including colored mediators have a broader set of multi-jet signatures and kinematic features with respect to the neutral mediator models, including the radiation of vector bosons by the mediator~\cite{Bell:2012rg}. 

%Parameters
In colored (pseudo-)scalar models, the mediator must be heavier than the DM particle to ensure DM stability. 
%The mediator mass is set equal for all mediators due to the MFV assumption,
%Requirement of the width: mChi^2+mq^2<=Mmed^2
The coupling between DM and quarks \gdmq in LHC searches have been set to be universal but only to the first two quark generations, violating MFV. However, if only right-handed, down-type quarks are considered, flavour constraints still do not exclude a significant part of the parameter space~\cite{Abercrombie:2015wmb}. 
%CD: the previous sentence does not have a source, and the Bell model only uses left-handed quarks. This is a question for Millie and the theorists present tomorrow. 

This simplified model, with exchange of a scalar colored under SU(3), is analogous to squarks in the MSSM where only squarks and neutralino are light. In the MSSM, the coupling between DM and the squark is constrained to be small. In the more general case, without the necessity of fitting into a SUSY framework, this coupling need not be small. For example, if the DM in this model is a standard thermal relic, the couplings required to obtain the correct dark matter density are generally higher than what used by SUSY models. Models with three generations of mediators can satisfy flavor constraints and the SM gauge symmetry~\cite{Ko:2016zxg}. 
%Citation to MG's studies?
%Couplings to vector bosons also allow the mediator to radiate a W or Z, leading to collider signatures that can be targeted by searches looking for this radiation~\cite{Bell:2012rg}. 

%Problems 
%If one tunes the parameters and couplings of this model to describe gamma ray excess and motivate searches with a single $b-$quark in the final state~\cite{Agrawal:2014una}, leading to a similar phenomenology as the MSSM with a light bottom squark and neutralino. 
%Top-flavored models also exist in literature but have not been used as benchmarks in LHC searches. 

\subsubsection{Less simplified models}
\label{sec:LessSimplifiedModels}

%The initial list of models recommended to the experimental collaboration by the Dark Matter Forum described above is a first set of simple, mostly tree-level processes targeting early Run-2 searches. 

Simplified models can be a useful guide when designing generic searches. However, targeting the simplest models one at a time does not cover the full complexity of possible collider signatures that arise in more complete models. On the other hand, relying too heavily on a small sample of complete models risks focusing searches too narrowly on an unrepresentative set of signatures.

There are a large number of "less-simplified" models in literature that attempt to find a middle ground between very generic and simple models and fully developed models. Because the variety of these models grows quickly with the number of ingredients, and there is not a broad consensus on which models should be a priority, very few of them have been explicitly considered by LHC searches. In this review, we will only sketch the main characteristic of a small selection of models that lead to distinct signatures with respect to the simplified models described so far. 

\textbf{Co-annihilation} models add one extra particle to the dark sector, generally close in mass to the DM particle. Examples can be found in Refs.~\cite{Buschmann:2016hkc,Baker:2015qna,Khoze:2017ixx}. The interaction between these two states drives the cosmological history~\cite{Ellis:1999mm,PlehnLecturesDM}, as processes involving both types of DM particles can efficiently annihilate DM into SM particles. The LHC phenomenology of co-annihilation simplified models includes missing transverse energy and multiple hadronic jets accompanied by multiple resonant or non-resonant hadronic jets, but it can be very rich, sharing signatures with a diverse array of existing BSM searches looking for unrelated ideas (e.g., searches for lepto-quarks). In some cases, these signatures are untested by any current LHC search~\cite{Buschmann:2016hkc}. 

Most of the models discussed above are constructed assuming Minimal Flavor Violation (MFV), i.e. the same pattern of flavor violation that is found in the Standard Model, to ensure that the models are compatible with a vast and unwieldy assortment of experimental constraints on flavor-violating processes. Nevertheless, viable \textbf{non-mimimal flavor-violating models} can be constructed, but one then must understand the impact of the many constraints \cite{Blanke:2017tnb}. Mediators that couple to dark matter and a top quark are one category of flavor-violating model that remains least constrained by low-energy measurements~\cite{DHondt:2015nat}. These yield a distinct 'mono-top' LHC signature. 

Other~\textbf{models with multiple mediators} with small couplings to SM particles have been developed to escape existing LHC constraints~\cite{Duerr:2016tmh}. 

[add more models that we find interesting if we have time. Other types of initial states, gluphilic, and lepton DM that we can't produce at the LHC but at LEP.]
%Dark terminator needs Majorana particle, not mentioned although the idea of vector + scalar is interesting beyond LianTao's 2HDM
%OOutline said gluphilic, but we cited it above and we need to save space

%JUNK: However, as we will see in Sec.~\ref{sec:LessSimplifiedModels}, this complexity does not translate into significant changes in the LHC kinematics of the simplest models, but rather adds extra signatures for LHC searches. Meaning: it can be rescaled. 

%Problem: we don't know whether scalar sector of the SM is only Higgs or there's a more complex Higgs sector.

The recent discovery of the Higgs boson places LHC at the forefront of the exploration of the SM scalar sector. Ultimately we don't yet know whether this scalar sector is limited to the SM Higgs boson. The model of SM-DM interaction with a single scalar mediator may not encode all important features of the more complicated phenomenology of more \textbf{complex scalar sectors}. Extended scalar sector models often involve signatures with \MET accompanied by the Higgs or other SM bosons.

One step beyond the simple scalar mediator model is to take mixing between the single scalar mediator and the SM Higgs boson, dictated by gauge invariance, into account~\cite{Bauer:2016gys,Berlin:2014cfa}. 

A much larger step beyond this is to consider an extended Higgs sector, like found in supersymmetry, such as a Two-Higgs Doublet Model (2HDM) where one or more of the scalars acts as the SM-DM mediator~\cite{Bauer:2017ota,Ipek:2014gua,No:2015xqa,Goncalves:2016iyg,Bell:2016ekl}. In these models, the new scalar or pseudo-scalar mediator mixes with the Higgs partners rather than with the SM Higgs, so that the model remains compatible with Higgs measurements. Some 2HDM models developed for LHC searches focus on one Yukawa structure (Type-II). The particle content of this model includes two CP-even bosons (one of which is the SM Higgs boson), two CP-odd bosons (of which one is the pseudoscalar DM mediator), two charged Higgs bosons, and the DM particle. Masses and couplings of these models are chosen to respect vacuum stability~\cite{No:2015xqa}, electroweak and flavour constraints, and to reproduce the observed dark matter abundance.

%Depending on the parameter chosen, this model can satisfy the relic DM density, in general with values of \mdm above 100 GeV.   
%%Overkill?
%[1] M. Bauer, U. Haisch, F. Kahlhoefer, CERN-TH-2017-011, DESY-17-010 [arxiv:1701.07427 [hep-ph]].
%[2] S. Ipek, D. McKeen and A. E. Nelson, Phys. Rev. D 90, no. 5, 055021 (2014) [arXiv:1404.3716 [hep-ph]].
%[3] J. M. No, Phys. Rev. D 93, no. 3, 031701 (2016) [arXiv:1509.01110 [hep-ph]].
%[4] D. Goncalves, P. A. N. Machado and J. M. No, arXiv:1611.04593 [hep-ph].
%[5] N. F. Bell, G. Busoni and I. W. Sanderson, arXiv:1612.03475 [hep-ph].
%^The richer span of experimental signatures permits to expose uncovered regions in the parameter space of the model, as well as to highlight the complementarity between final states. 

%Sam's LianTao's 2HDM checks
%https://docs.google.com/presentation/d/10R9XJaoMDEhXKhd_Wx9yMXEaPl4uXR8IcmuTeLancvg/edit#slide=id.g217998804d_0_47

\subsection{Supersymmetric models and other complete theories}
\label{sec:SUSYModels}
NEEDS WORK

So far, we have considered models that are simple extensions to the SM. There is no claim of completeness for these models,
so their combined phenomenology cannot be considered a comprehensive list of the ways in which DM would manifest in a collider experiment.
A complementary way to attempt to repair [Italianism, synonym needed] to this issue is to look for theories that have been built with
a different philosophy with respect to simplified models, e.g. by attempting to solve other theoretical problems of the SM, 
and concentrate on theories that provide a good DM candidate. 

%AB sentence which I liked better than mine
%Supersymmetry (SUSY), besides solving many theoretical problems of the Standard Model, often provides a dark matter candidate. 
Superymmetry (SUSY) is one of these theories.
The literature on this subject is far too broad to cover here, and more complete reviews of supersymmetric DM models can
be found in \cite{Feng:2010gw,Ellis:2010kf}. Instead, we will broadly sketch models that are relevant to
the experimental progress to date and point out a few areas where we expect future developments. 

Supersymmetric relic dark matter, the archetype for the WIMP idea, has a long
history \cite{doi:10.1016/0550-3213(84)90461-9}.
Of these, the most viable and well-studied has been neutralino dark matter. 
%alternative: sneutrino DM in MSSM, but it is excluded by LEP and DD and doesn't fit relic
The presence of this stable, neutral supersymmetric spin 1/2 fermion that
is lighter than all other particles in the model (Lightest Supersymmetric Particle, or LSP) %so that it fits the bill of our grounding assumptions
is a consequence of R-parity conservation~\cite{Farrar:1978xj}. %here we can also cite Ellis
%, a key feature of many SUSY theories 
%featuring supersymmetric partners of SM particles. 
%maybe we could cut, but we need it if we want to mention that make clear that gravitino needs to be RPV to be observed
In the Minimal Supersymmetric Standard Model (MSSM),
%, where superpartners have the same mass as their SM counterparts
%this could be a sidebar where one explains that SUSY needs to be broken to have masses that are not SM
there are four neutralinos, each a mixture of SM boson partners: a wino, a bino, and two higgsino fermion states.
The lightest neutralino may be called 'bino-like,' 'wino-like,' or 'higgsino-like'
in regions of MSSM parameter space where one of these components dominates the mixture.
The collider rates and phenomenology of these models depend on the composition of the chosen scenario,
and on the remaining particle spectrum. 
LHC signatures feature missing transverse momentum from the neutralino and a high multiplicity of other objects
(leptons, jets) produced from the cascade decays of heavier supersymmetric objects. The phenomenology
of SUSY scenarios is different from most of the simplified models in the previous section. 
If sufficiently energetic, the cascade products can serve as additional experimental handles 
to target specific benchmark points. 

JUNK: If the next-to-lightest SUSY particle (NLSP) is much heavier than the LSP, the LSP will be boosted and produce
a large amount of missing transverse momentum. If instead the mass spacing between NLSP and LSP is 
small (compressed scenario) there will be only room for a limited amount of \MET in the event.

Another DM candidate in gauge- or gravity-mediated supersymmetric models, is the gravitino, 
%MSUGRA: gravitino is too weakly interacting? unclear
a 3/2-spin particle superpartner of the graviton. 
Gravitino interactions are suppressed by the Planck scale (10$^{18}$ GeV) before SUSY breaking.
%the caveat is that some stuff happens during SUSY breaking and improves the rates, but 10^18 is a shitton
This has consequences both on their viability as a thermal relic and on their phenomenology. 
In gauge-mediated SUSY, the gravitino is a good DM candidate assuming a
non-standard cosmological history~\cite{Steffen:2007sp}. 
%For AB: I am not sure this is enough...
%and a mass below the GeV~\cite{}. 
%https://arxiv.org/pdf/hep-ph/9801417.pdf
%While very light gravitinos (eV masses)
%will not contribute significantly to the total mass density of the universe, if
%the gravitino LSP is too heavy (m g >? few keV) its relic mass density in 3/2
Similarly to the neutralino case, the identity and masses of the heavier state determine the
gravitino LSP signatures. However, the gravitino interactions are very weak, posing problems for 
its detection in direct and indirect detection experiments that would be needed to confirm 
a collider discovery. Observation at colliders is also difficult due to the low rates of direct production,
but additional experimental features from the long-lifetime of coannihilating long-lived NLSPs
can be used to discriminate between rare signals and backgrounds. 
%here and above we are specifically thinking of tau because it gives a good relic, maybe worth mentioning later

%Even though the gravitino does not easily constitute a good
%thermal relic candidate, it provides different signatures 
%i found Bolz:2000fu that says it's a good one

Since the MSSM is a complete theory with more than 100 independent parameters, 
versions of SUSY models implementing simplifying hypotheses (e.g. universality of certain
particle masses or parameters) with specific parameter choices 
are developed and used as predictive benchmarks for DM searches. 
SUSY searches have also adopted a simplified
model approach, decoupling the particles that determine the collider
phenomenology (generally LSP and NLSP)
from the rest of a heavier particle spectrum~\cite{Alves:2011wf}. 
Conversely, models that extend the MSSM also exist. Examples are models where a right-handed neutrino
is added to the SM leading to a SUSY DM candidate with signatures privileging leptons~\cite{Arina:2015uea},%this motivates monoW
and R-parity violating models, which have connections to dark sectors in so-called \textit{Asymmetric Dark Matter} realizations, 
or models that add to the minimal particle spectrum of the MSSM superpartners. 

\begin{marginnote}[]
\entry{Asymmetric Dark Matter}{Models where DM particles and antiparticles
are not produced in equal amounts, in the same fashion as matter and antimatter
for baryons. For a review, see ~\cite{Zurek:2013wia}.}
\end{marginnote}

%Aside from dark matter comprised of neutralinos, DM could be consistuted from gravitinos. Gravitino DM would not be a thermal relic (because only gravitational strength interactions, cite?). At colliders, it could be produced by the decay of heavier SUSY states. Similar to the neutralino cases, the identity and mass of the heavier states (i.e. the NLSP) determine the signatures, but for the gravitino the favored signatures are different. 
%- NLSP could be stop i.e. tops + MET signature
%- NLSP could be stau, metastable charged particle at the LHC, relatively long-lived (non-prompt) lepton signatures
%- NLSP could be sneutrino and so forth.

The absence of SUSY signals after the first LHC data, as detailed in Sec.~\ref{03_ExperimentalResults}
has also generated interest in theories that can evade those constraints. 
%but also give up some of SUSY's desirable features. 
An example is neutral naturalness, a theoretical framework that features a mirror copy of the SM,
with no equivalent of the SUSY colored partners whose mass scales have been
pushed beyond the "natural" TeV scale. Models realizing "neutral naturalness"~\cite{Craig:2014aea}
include dark sector particles with non-standard signatures and Dark Matter candidates similar to the
Asymmetric Dark Matter models mentioned above~\cite{Garcia:2015toa},
and it will be testable at colliders beyond the LHC. 

Other BSM theories including DM particle candidates that are not covered in this review are
extra dimensions~\cite{Hooper:2007qk}, and DM as sterile neutrino~\cite{Adhikari:2016bei}.
%If we want to summarize we can look here: http://www.slac.stanford.edu/econf/C040802/papers/L002.PDF
%For AB who has book: See Bertone's book for other non-SUSY candidates at the EW scale? 
%\cite{FengAR} %x-devonthink-item://F9E6F4B6-265D-48CF-B8CC-48B17D28C0DC

\subsection{Long-lived particle models}
\label{sec:LLPModels}

%Connection to SUSY

Another class of models found within supersymmetry and beyond it are models where the cascade decays of a heavier particle (the NLSP in SUSY) to a lighter particle (the DM LSP in SUSY) are suppressed. The suppression can be so large that the particle travels a macroscopic length within the detector before it decays. One way to achieve this suppression in SUSY is to require the NLSP to decay through a heavy intermediary. Split supersymmetry models are a subset of SUSY models where the gluino must decay through a heavy, off-shell squark~\cite{Masiero:2004ft}. The heavier the mass of the squark, the longer-lived the gluino. This is akin to the W-mediated decay of the pion. 
Alternatively, the NLSP decay can be heavily suppressed by some power of the mass difference with the LSP. The mass difference between LSP and NLSP also makes a significant difference in the DM co-annihilation rates and therefore the DM abundance~\cite{Ellis:1999mm}. Finally, another way to achieve long-lived decays is with parameterically small couplings, as in the case of gauge-mediated supersymmetry models where the long-lived NLSP decays to its SM partner plus the gravitino~\cite{Dimopoulos:1996vz}, with a SM analogue in the Cabibbo-suppressed B-meson decays. Because of the prevalence of these mechanisms within and without SUSY, it is important to look for long-lived cascade decays.

Besides SUSY, one can find long-lived signatures within the generic simplified models in Sec. \ref{sub:simplifiedModels} with small enough couplings. If one assumes thermal freeze out, the coupling of the mediator to DM pairs is bounded from below to obtain a sufficiently large annihilation cross section. However, it's anybody's guess what mechanism in the early universe was responsible for the observed DM density. In alternate scenarios, such as "dark freeze out" where DM can annihilate directly to BSM mediators but not viceversa, the mediator couplings to the SM can be far smaller than allowed in standard thermal freeze out~\cite{Pospelov:2007mp,Das:2010ts}. The couplings of the mediator therefore can be small enough that it is long-lived. Despite these small couplings, the mediator can be produced at colliders with appreciable cross section provided that it is light enough. A large number of models have been proposed in this direction. Two notable examples are dark vector boson models, where DM interacts with the SM via a vector boson originated by a U(1)' dark symmetry equivalent to the SM's U(1) but with much smaller couplings~\cite{Holdom:1985ag} such as those originated by kinetic mixing. 
The mediator can also be a dark scalar boson (a "dark Higgs") that only couples to the SM, akin to a Higgs portal~\cite{Curtin:2014cca}. %or via mixing with a heavy pseudoscalar in 2HDM with a coupling $k$. Who cares, maybe also who cares about the mixing even. 
%The reason why it is interesting is because ID is suppressed but it's also very hard to discover at colliders.
In both these cases, the dark boson mediator can be light and long-lived~\cite{Pospelov:2007mp}, and its visible decays into SM particles or associated production with a SM boson provide the main collider handle for observation~\cite{Curtin:2014cca}. These scenarios can also be probed by complementary beam dump and fixed target experiments~\cite{Battaglieri:2017aum}. %maybe in results section?
%, including present and future colliders, as this model includes decays of the dark boson into SM fermions as well as into DM particle, and include Higgs decays. 
%Z width, contact interactions
%Relic connection
%There is still a large possible dark boson and DM candidate mass range that is still compatible with thermal freeze-out~\cite{Das:2010ts}. 
Another mechanism for generating a relic density that is consistent with very weak SM interactions such as those of dark photons is the freeze-in scenario, see e.g. Refs.~\cite{Co:2015pka,Bernal:2017kxu}. 
%Further examples of the third case can be connected to the simplified models described in Sec.~\ref{sub:simplifiedModels}. If the only connection between the DM and SM is the new mediator particle, and e.g. when DM can annihilate directly to BSM mediators and not viceversa (as \mdm $>$ \mmed), then the couplings of the mediator to the SM can be arbitrarily small.

%too LLP? "we need this, an example of a way to do it is here"
Motivated by the wide variety of theoretically possible, experimentally challenging long-lived particle signatures, various efforts are ongoing to derive a prioritized set of benchmark models for collider searches in a similar fashion as Ref.~\cite{Abercrombie:2015wmb}. Among the various possibilities, we note the bottom-up approach adopted in~\cite{Buchmueller:2017uqu}, choosing masses and couplings for the models described in~\ref{sub:simplifiedModels} so that they must include a long-lived particle. The categorization of the models by production operator and final state permits to adopt a more systematic set of benchmarks for this kind of signatures. These models can then be mapped onto more complete theories. However, no attempts have yet been made to connect the models in~\cite{Buchmueller:2017uqu} to cosmological history. 
Simplified co-annihilation models with long-lived particles have also been proposed~\cite{ElHedri:2017nny}. 

%To illustrate this mechanism, in Section 2 we will construct several models with fermionic, scalar and vector particles as mediators. We show that if mWIMP < mmediator, the parameter space of such models is highly constrained, as the coupling of the mediator to the SM must necessarily be sizable to ensure the required annihilation cross-section. Yet if the reverse is true, mWIMP > mediator, there are no strict requirements on the size of the mixing except for the lifetime of the mediator state, which in some instances can be satisfied for (mixing)2 of the mediator with the SM as low as 10−23.

%when the couplings between the partner particle and either SM or DM particles are small, as in the SM example of the Cabibbo-suppressed B-meson decays, or in the example of 
%produced with appreciable cross-sections at the LHC. 

%If the mass difference is lower than 100 MeV, the NLSP may be long-lived as the available phase space
%for its decay into LSP and SM particles is limited, %this is a repetition but if I were a student I would like to know why
%leading to signatures of long-lived particles~\cite{Chen:1995yu}. 

%MET in the limit of long-lived
%As discussed in the introduction to this chapter, the LHC does not directly detect DM, but rather uses visible objects to signal the presence of non-interacting, long-lived particles that escape detection. If the particle does not decay, then it is a good DM candidate. In many non-WIMP, dark sector models, one can postulate the existence of DM particles as well as other particles with lifetimes not long enough to be cosmologically stable~\cite{Strassler:2006im}. Those particles escape conventional detection by collider experiments, as they e.g. decay half-way through the detector.
% and may lead to signatures of missing transverse momentum.
%this is ambiguous: they can also lead to MET if they decay invisibly?
%However, these dark sector particles usually do not carry sufficient energy to be observed by traditional searches, so experiments must devise methods that specifically target those non-standard decays. These will be discussed in Chapter~\cite{sec:03_ExperimentalResults}.
%JUNK: The small mass splitting between the two particles forces the decay of the next-to-lightest particle into the lightest particle to be kinematically suppressed, in turn leading to a sizable lifetime for the next-to-lightest particle~\cite{Khoze:2017ixx}. The late decays of the coannihilation partner give an additional experimental handle that can be used for LHC searches, as described in the next chapter.  


