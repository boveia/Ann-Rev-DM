\begin{textbox}[!h]
\section{PARTICLE PROPERTIES OF DARK MATTER}

\textbf{Stability}
If DM is a particle, it does not seem to decay.
Conservation laws, such as R-parity in Supersymmetry (SUSY) or a $Z_2$ symmetry, can prevent the DM particle from decaying into any lighter even-parity SM particle.
Additionally, pairs of DM particles can be produced by the decay of other particles, charged under the same gauge group as the SM, or singly in the case the parent is a color triplet. 

\textbf{Darkness} 
DM particles are effectively invisible to traditional collider experiments made of ordinary matter. However, the rest of the event is not. 
Invisible particles can be accompanied by one or more visible recoiling particles, leading to missing momentum in the transverse plane, whose magnitude is termed \MET. This is one of the main signatures of DM at colliders.

\end{textbox}
