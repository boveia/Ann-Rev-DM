%channeling @elonmusk

We are optimistic that, from collider approaches to underground experiments to observatories, the variety of powerful searches for particle dark matter will make much progress toward its discovery in the next decade.
%The complementarity amongst the types of searches does what ???
At colliders, the tools essential to discovering and understanding the fundamental particles of the Standard Model are once again being applied.
Our understanding of collider search targets is also rapidly improving.
Along with SUSY benchmarks motivated by the hierarchy problem, targets directly motivated by DM observations are encouraging a new generation of experimentalists to branch out into directions that so far are sparsely explored.
Finally, the LHC is just at the beginning to take data at a new 13~TeV collision energy, with the goal of a dataset that exceeds that presented here by a factor of 100.
These are exciting times.





%% for a discovery leading the way to future colliders and future experiments to investigate the particle nature of DM within the experimental reach of the LHC -- these are exciting times indeed. 


%%  - searches for dark matter are still a growing area



%% While particle dark matter has not been discovered in any search so far, many types of searches exist




%% %% From 1960 to the Higgs discovery, colliders have been essential tools to discover fundamental particles.
%% %% A similar expectation existed for the LHC restart: SuperSymmetry, with a Dark Matter candidate and a rich spectrum of particles mirroring the SM, seemed on the verge of being discovered.
%% %% This has not been the case so far.
%% Nevertheless, it is at this point that searches for new particles at collider, complemented by next generation DD and ID experiment and increasingly precise maps of our universe, become most interesting.
%% With only loose guidance from theoretical arguments and current observations, we are prompted to hedge our bets by searching in an increasing number of signatures, benefitting from novel techniques motivated by the search for dark matter, and by the prospect that 99\? of the LHC data still has to be collected and analyzed.


%% Dark matter (DM) is perhaps the most persuasive experimental evidence for physics beyond the Standard Model of particle physics~\cite{Bertone:2016nfn}. 

%% The field of particle physics is increasingly keen to understand what Dark Matter (DM) is, if it is indeed a particle. 
%% Some experiments, termed Direct Detection (DD) experiments, look for galactic DM colliding with underground targets made of Standard Model matter (SM)~\cite{0954-3899-43-1-013001}.
%% Others, termed Indirect Detection (ID) experiments, search for the products of annihilating dark matter concentrated within the gravitational potential wells of the Milky Way and elsewhere~\cite{Gaskins:2016cha}.
%% None of these experiments has yet found conclusive evidence of DM.
%% If the only interaction between DM and SM matter is gravitational, experiments will never see it.
%% Yet the search for particle DM started relatively recently, and plenty of room for optimism remains.

%% Colliders, one of the most successful tools of particle physics, have revealed much about SM matter.
%% This review will sketch how colliders can contribute to the search for DM, focusing on the highest-energy collider currently in operation, the Large Hadron Collider at CERN.
%% Absent hints for the character of DM-SM interactions, it emphasizes what could be observed in the near future, the main experimental challenges presented, and how collider searches fit into the broader field.
%% Finally, it underlines a few areas to watch for the future LHC program.
